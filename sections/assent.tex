\section{Assent}

\subsection{Introduction to Interpretation}

\subsubsection{Subjectivity and Objectivity}

\begin{enumerate}
    \item % TODO introductory note 368-69
\end{enumerate}

\paragraph{Restatement First \S\ 227}

\begin{enumerate}
    \item % TODO 369
\end{enumerate}

\paragraph{\emph{Lucy v. Zehmer}}

\begin{enumerate}
    \item % TODO 370
\end{enumerate}

\paragraph{\emph{Keller v. Holderman}}

\begin{enumerate}
    \item % TODO 374
\end{enumerate}

\paragraph{\emph{Raffles v. Wichelhaus}}

\begin{enumerate}
    \item % TODO 374
\end{enumerate}

\paragraph{Simpson, ``Contracts for Cotton to Arrive: The Case of the Two 
Ships \emph{Peerless}}

\begin{enumerate}
    \item % TODO 375
\end{enumerate}

\paragraph{\emph{Frigaliment Importing Co. v. B.N.S. Intern. Sales Co.}}

\begin{enumerate}
    \item % TODO 376
\end{enumerate}

\paragraph{\emph{Oswald v. Allen}}

\begin{enumerate}
    \item % TODO 380
\end{enumerate}

\paragraph{\emph{Falck v. Williams}}

\begin{enumerate}
    \item % TODO 380
\end{enumerate}

\paragraph{\emph{Colfax Envelope Corp. v. Local No. 458-3M}}

\begin{enumerate}
    \item % TODO 381
\end{enumerate}

\paragraph{Intent: \emph{Embry v. Hargadine, McKittrick Dry Goods Co.}}

Contracts generally require a meeting of the minds, but intent is irrelevant 
if the other party could not reasonably know the other's intent.

\begin{enumerate}
    \item Embry supported the company's sales team. After his one-year 
    contract was up, Embry tried to get McKittrick to agree to another 
    contract. After a brief conversation in McKittrick's office where 
    McKittrick said, ``Go ahead, you're all right. Get your men 
    out~.~.~.~''\footnote{Casebook p. 382.} Embry understood his words as 
    assenting to a new one-year contract. McKittrick didn't.
    \item The key issue was whether both parties intended to create a 
    contract.
    \item Contracts generally require a meeting of the minds, but intent is 
    irrelevant if the other party could not reasonably know the other's 
    intent.
    \item Held for Embry.
\end{enumerate}

\paragraph{Restatement Second \S\S\ 20, 201}

\begin{enumerate}
    \item % TODO supp
\end{enumerate}

\paragraph{CISG Art. 8}

\begin{enumerate}
    \item % TODO supp
\end{enumerate}

\paragraph{UNIDROIT Arts. 4.1, 4.2, 4.3}

\begin{enumerate}
    \item % TODO supp
\end{enumerate}

\paragraph{Principles of European Contract Law Arts. 5.101, 5.102}

\begin{enumerate}
    \item % TODO supp
\end{enumerate}

\paragraph{CISG and Intent: \emph{MCC-Marble Ceramic Center, Inc. v. 
Ceramica Nuova D'Agostino}}

CISG allows the court to discern the parties' subjective intent, even absent 
outward manifestations.

Would this case have come out differently if US law controlled?

\begin{enumerate}
    \item MCC contracted to buy tiles from D'Agostino. They signed a form 
    contract.
    \item MCC brought suit for D'Agostino's failure to fill orders. D'Agostino 
    argued that it was not obligated to fill the orders because MCC had 
    defaulted on earlier payments.
    \item The form contract gave D'Agostino the right to cancel if MCC failed 
    to pay. MCC argued that it failed to pay because some of the tiles were 
    unsatisfactory---but the contract also had a clause requiring written 
    notice of complaints, which MCC did not give.
    \item MCC argued that it had an oral agreement with D'Agostino that the 
    clauses in the form contract would not apply.
    \item CISG allows inquiry into subjective intent, ``even if the parties 
    did not engage in any objectively ascertainable means of registering this 
    intent.''\footnote{Casebook p. 387.}
    \item Three affiants testified to MCC's intent to nullify the clauses. 
    While the affidavits may have been conclusory, they at least presented a 
    triable issue of fact, so summary judgment was inappropriate.
\end{enumerate}

\paragraph{\emph{Mayol v. Weiner Companies, Ltd.}}

\begin{enumerate}
    \item Mayol contracted to buy a piece of property which Weiner sold on 
    behalf of the owner. The contract included a clause granting possession 
    ``on or before November 1, 1979 subject to tenant's rights.'' Mayol paid a 
    \$1,000 deposit.
    \item It turned out that the tenant had a right to purchase. Upon learning 
    this, Mayol breached and sued to recover his deposit.
    \item Mayol had asked about the lease but Weiner had not told him about 
    the tenant's purchase option until after the sale contract was complete. 
    The court held for Mayol, reasoning that he had no reason to believe that 
    he was buying property subject to a purchase option, and the seller had no 
    reason to think that Mayol believed so, either.
\end{enumerate}

\paragraph{Objective and Subjective Elements in Interpretation}

\begin{enumerate}
    \item Classical contract law largely disregarded the parties' intent.
    \item There are four principles of interpretation in modern contract 
    law:\footnote{Casebook p. 394--95.}
    \begin{enumerate}
        \item The more reasonable meaning prevails If both parties attach 
        different subjective meanings to an expression  and they 
        are not equally reasonable.
        \item But if the two meanings are equally reasonable, neither 
        prevails.
        \item If the parties attach the same meaning, that meaning prevails 
        even if it is unreasonable.
        \item If A and B attach different meanings, and A knows B's meaning 
        but B doesn't know A's, B's meaning prevails even if it is less 
        reasonable.
    \end{enumerate}
\end{enumerate}

\paragraph{\emph{Berke Moore Co. v. Phoenix Bridge Co.}}

\begin{enumerate}
    \item Mutual understanding is not private and is therefore valid.
\end{enumerate}

\subsubsection{Problems of Interpreting Purposive Language}

% TODO 397-406

\subsubsection{Usage, Course of Dealing, and Course of Performance}

% TODO 406-414

\subsection{Offer and Revocation}

\subsubsection{What Constitutes an Offer}

% TODO 413-424

\subsubsection{Lapse, Rejection, and Counter-Offer}

\paragraph{\emph{Akers v. J.B. Sedberry, Inc.}}

\begin{enumerate}
    \item % TODO 
\end{enumerate}

\paragraph{\emph{Effect of the Rejection of an Offer}}

\begin{enumerate}
    \item When does an offeree terminate his power of acceptance?
    \item Generally, it's when the offeror would reasonably understand that 
    the offeree has taken the offer off the table.
\end{enumerate}

\paragraph{Qualified vs. Absolute Acceptance: \emph{Ardente v. Horan}}
~\\\\
A qualified acceptance, subject to a condition, does not create a contractual 
obligation if the other party does not satisfy the condition.

\begin{enumerate}
    \item Ardente tried to buy property from Horan. He sent a \$20,000 deposit 
    with a letter asking the sellers to leave behind certain furnishings 
    (patio furniture, etc.). Horan declined and returned the deposit. Ardente 
    sued for specific performance.
    \item The issue was whether Ardente's deposit and letter constituted a 
    contract, hinging on the condition in the letter. Was it a qualified 
    acceptance, subject to a condition, or an ``absolute acceptance together 
    with a mere inquiry concerning a collateral matter''?\footnote{Casebook 
    p. 432.}
    \item The court held that Ardente's letter was a qualified acceptance, 
    binding only if Horan met its conditions. Thus, ``it operated as a 
    rejection of the defendants' offer and no contractual obligation was 
    created.''\footnote{Casebook p. 432.}
\end{enumerate}

\paragraph{\emph{Rhode Island Dep't of Transp. v. Providence \& Worcester 
R.R.}}

\begin{enumerate}
    \item % TODO 433
\end{enumerate}

\paragraph{\emph{Price v. Oklahoma College of Osteopathic Medicine and Surgey}}

\begin{enumerate}
    \item % TODO 434
\end{enumerate}

\paragraph{Mirror-Image Rule}

\begin{enumerate}
    \item % TODO 435
\end{enumerate}

\paragraph{\emph{Livingstone v. Evans}}

\begin{enumerate}
    \item % TODO 436
\end{enumerate}

\paragraph{\emph{Culton v. Gilchrist}}

\begin{enumerate}
    \item % TODO 436
\end{enumerate}

\paragraph{Effect of the Offeror's Death or Incapacity Before Acceptance}

\begin{enumerate}
    \item % TODO 436
\end{enumerate}

\subsubsection{Revocation}

\paragraph{\emph{Dickinson v. Dodds}}

\begin{enumerate}
    \item % TODO 438
\end{enumerate}

\paragraph{Contract Practice}

\begin{enumerate}
    \item % TODO 441
\end{enumerate}

\paragraph{What Constitutes Receipt of Written Acceptance}

\begin{enumerate}
    \item % TODO 441
\end{enumerate}

\paragraph{\emph{Ragosta v. Wilder}}

\begin{enumerate}
    \item % TODO 442
\end{enumerate}

\paragraph{Offers for Unilateral Contracts}

\begin{enumerate}
    \item % TODO 446
\end{enumerate}

\paragraph{Reliance and Revocability: \emph{Drennan v. Star Paving Co.}}
~\\\\
Promissory estoppel prevents a subcontractor from revoking its offer once the 
contractor has acted upon the subcontractor's promise.

\begin{enumerate}
    \item Facts:
    \begin{enumerate}
        \item Drennan was preparing a bid for a school construction job. His 
        final bid was \$317,385. He had to provide a 10\% bond as a guarantee 
        that he would enter the contract.
        \item On July 28, 1955 Drennan received between 50 and 75 
        subcontractor bids. Star Paving Co. put in a bid for \$7,131.60 for 
        paving work.
        \item Drennan's was the winning bid. The next day, he stopped by Star 
        Paving's offices, where one of its engineers told him they had 
        mistakenly underbid. The had intended to bid \$15,000 and refused to 
        do the job for anything less.
        \item Drennan found another company, L \& H, to do the paving for 
        \$10,948.60. Drennan then sued Star for the difference, or \$3,817.00.
    \end{enumerate}
    \item Star argued that ``there was no enforceable contract between the 
    parties on the ground that it made a revocable offer and revoked it before 
    plaintiff communicated his acceptance to defendant.''\footnote{Casebook p. 
    449.}
    \item Drennan argued that ``he relied to his detriment on defendant's 
    offer and that defendant must therefore answer in damages for its refusal 
    to perform.''
    \item The trial court held that Star made a definite offer and that 
    Drennan relied on Star's bid in computing his own bid. It awarded 
    \$3,817.00 in reliance damages to Drennan.
    \item Earlier cases held that the subcontractor is not bound because there 
    was no binding promise of an irrevocable offer and no consideration.
    \item On appeal, the question was whether Drennan's reliance made Star's 
    offer irrevocable.
    \item Judge Traynor:
    \begin{enumerate}
        \item Star's offer was a promise. Since Star was silent on revocation, 
        the court had to decide whether the promise was revocable.
        \item Restatement \S\ 90 hold that justifiable reliance can 
        make a promise binding, even absent bargained-for 
        consideration.\footnote{Casebook p. 450. This case was decided in 
        1958. The ALI published the Restatement in 1932 and Restatement Second 
        in 1979.}
        \item Star should have known that Drennan would have relied on its 
        promise in computing its bid---indeed, it wanted him 
        to.\footnote{Casebook p. 450--51.}
        \item Star also argued that its bid was the result of a mistake, so it 
        was entitled to revoke it. The court held that it would be revocable 
        only if Drennan had not relied on it. But here, Star's mistake misled 
        Drennan, causing further detriment and creating an additional reason 
        to enforce the promise.
        \item Star's last argument was that Drennan failed to mitigate 
        damages. The court dismissed this argument, finding that he had.
    \end{enumerate}
\end{enumerate}

\paragraph{Critiques of \emph{Drennan}: \emph{Pavel Enterprises, Inc. v. A.S. 
Johnson Co. 452}}

\begin{enumerate}
    \item Subcontractors are bound to the general, but the general is not 
    bound to the subcontractor, creating incentives for the general contractor 
    to act unethically:
    \begin{enumerate}
        \item \emph{Bid shopping}: using the lowest bid to negotiate lower 
        bids from others.
        \item \emph{Bid chopping}: pressuring the subcontractor to make a 
        lower bid.
        \item \emph{Bid peddling}: a subcontractor waits until other bids are 
        in and then undercutting them, avoiding the cost of estimating his own 
        bid.
    \end{enumerate}
    \item Most courts have followed \emph{Drennan}, but at least one as 
    deviated.\footnote{Casebook p. 453.}
\end{enumerate}

\paragraph{Dodge, ``Teaching the CISG in Contracts''}

\begin{enumerate}
    \item Common law: ``an offer is freely revocable, even if the offeror has 
    promised to hold it open, unless that promise is supported by 
    consideration or reliance.''\footnote{Casebook p. 453.}
    \item UCC: merchants can make a ``firm offer'' (i.e., an irrevocable 
    offer) without the need for consideration. The offeror must be a merchant, 
    etc.\footnote{Casebook p. 453.}
    \item CISG Art. 16 allows an offeror to make an irrevocable offer without 
    these restrictions.
\end{enumerate}


\paragraph{Limiting \emph{Drennan}: \emph{Preload Technology, Inc. v. A.B. \& 
J. Construction Co., Inc.}}

\begin{enumerate}
    \item When the general contractor engages in the practices warned against 
    in \emph{Pavel Enterprises} (bid shopping, etc.), \S\ 90 reliance may not 
    be available.
\end{enumerate}

\paragraph{Restatement Second \S\ 87(2): Option Contract}

\begin{enumerate}
    \item ``An offer which the offeror should reasonably expect to induce 
    action or forbearance of a substantial character on the part of the 
    offence before acceptance and which does induce such action or forbearance 
    is binding as an option contract to the extent necessary to avoid 
    injustice.''
    \item An option contract is an offer in which the offeror promises to keep 
    the offer open for a certain period of time. For instance, a seller grants 
    a buyer the option to buy his house for \$1,000 anytime during the next 
    month.
    \item The distinction between \S\S\ 45 and 87 is that an offeree who has 
    begun performance can recover expectation damages, while an offeree who 
    has not begun performance can only recover reliance 
    damages.\footnote{Casebook p. 455.}
\end{enumerate}

\subsection{Modes of Acceptance}

\subsubsection{Acceptance by Act}

\paragraph{Promise to Bequeath: \emph{Klockner v. Green}}
~\\\\
A promise becomes binding when the offeree acts on the offeror's request.

\begin{enumerate}
    \item Richard and Francis Klockner were stepson and stepgranddaughter of 
    the decedent, Edyth Klockner. Edith Klockner left a will devising her 
    property to her husband, who predeceased her, so upon her death her 
    property would have passed to her heirs by the rules of intestate 
    succession.
    \item However, Edyth Klockner promised to leave her real property to 
    Richard and her personal property to Francis. She had her attorney draw up 
    a will to this effect, but she never signed it---``stymied by her own 
    superstition.''\footnote{Casebook p. 464.}
    \item Richard and Francis testified that they would have taken care of 
    Edyth even if she hadn't promised to leave them her property. The trial 
    court held that there was no contract because there was no offer and 
    acceptance nor consideration. The appellate court held that the statute 
    of frauds barred enforcement because there had been no reliance.
    \begin{enumerate}
        \item Statute of frauds: at common law, it held that contracts are not 
        enforceable if they are not in writing.
    \end{enumerate}
    \item The question on appeal was whether Edyth Klockner entered into a 
    valid, binding contract with Richard and Francis to bequeath her property 
    to them.
    \item Held: Edyth's promise became binding when Richard and Francis acted 
    upon it. It need not have been their sole motivation. Edyth also received 
    the benefit of her bargain.
\end{enumerate}

\paragraph{\emph{De Cicco v.  Schweizer}}

\begin{enumerate}
    \item % TODO 466
\end{enumerate}

\paragraph{\emph{Simmons v. United States}}

\begin{enumerate}
    \item % TODO 465
\end{enumerate}

\paragraph{\emph{Stephens v. Memphis}}

\begin{enumerate}
    \item % TODO 466
\end{enumerate}

\paragraph{Performance of a Condition as Acceptance: \emph{Carlill v. Carbolic 
Smoke Ball, Inc.}}
~\\\\
Performing the conditions counts as acceptance of the offer. For 
instance, if the Carbolic Smoke Ball manufacturers promise a reward for anyone 
who uses the product and gets sick, anyone who performs those conditions has 
accepted the offer and can recover the reward.

\begin{enumerate}
    \item The manufacturers of ``The Carbolic Smoke Ball'' placed a newspaper 
    ad promising a cash reward for anyone who became sick after using their 
    product three times daily for two weeks. The plaintiff did and got sick. 
    She sued for the reward.
    \item The trial court held that she could recover the value of the reward.
    \item Was this a promise or ``mere puff''?\footnote{Casebook p. 467.} It 
    was a promise.
    \item Was it binding? The defendants argued that the promise was made to 
    nobody in particular. The court held that the ad was an offer to pay the 
    reward to ``anyone who will perform these conditions, and the performance 
    of the conditions is the acceptance of the offer.''\footnote{Casebook p. 
    467.}
    \item The defendants also argued that there was no notice requirement. The 
    court held that notice was irrelevant.
    \item The defendants argued, finally, that there was no consideration. The 
    court disagreed, finding that any use of their product conferred a 
    benefit. Moreover, consumers are inconvenienced when they use the Carbolic 
    Smoke Ball.
    \item Affirmed.
\end{enumerate}

\subsubsection{Subjective Acceptance}

\paragraph{\emph{International Filter Co. v. Conroe Gin, Ince, \& Light Co.}}

\begin{enumerate}
    \item International Filter manufactured water purifying machinery. Conroe 
    made ice, etc. International Filter sent Conroe a proposal for the sale of 
    two water tanks. The proposal dictated that it ``becomes a contract when 
    accepted by the purchaser and approved by an executive officer of 
    International Filter Company~.~.~.''\footnote{Casebook p. 471.}
    \item Conroe wrote back with ``Accepted.'' The president of International 
    Filter wrote ``O.K.'' International then sent Conroe a confirmation.
    \item Conroe later tried to back out. International Filter sued for 
    performance.
    \item Conroe argued (1) that International's ``O.K.'' did not amount to an 
    approval by an executive and (2) that International did not notify Conroe 
    of its acceptance of the contract.\footnote{Casebook p. 472.}
    \item The trial court found for Conroe. Affirmed on appeal.
    \item The court here held that International was not required to 
    communicate its acceptance. As long as an executive at International 
    approved the order, notice of the approval to Conroe was not necessary.
    \item Moreover, even if notice had been required (which it wasn't), 
    International's ``O.K.'' would have been good enough.
    \item Reversed.
\end{enumerate}

\subsubsection{Acceptance by Conduct}

\paragraph{Silence as Assent: \emph{Polaroid Corp. v. Rollins Environmental 
Services (NJ), Inc.}}

\begin{enumerate}
    \item Rollins operated a hazardous waste disposal facility. Rollins 
    disposed of hazardous waste from Polaroid and Occidental.
    \item Hazardous waste disposal agreements between Polaroid/Occidental and 
    Rollins included an indemnification for liability for spills. Occidental 
    included its indemnity clause in its purchase orders. Rollins accepted 
    these purchase orders several times before it objected.
    \item Polaroid and Occidental sued for a determination that Rollins was 
    obligated to indemnify them against liability for hazardous waste spills. 
    \item The trial court held that Polaroid and Occidental had valid 
    contracts with Rollins indemnifying them from liability.
    \item The New Jersey Department of Environmental Protection requested 
    compensation of \$9,224,189 for cleanup.
    \item Rollins argued that it was not obligated to indemnify Occidental 
    because it refused the purchase orders containing the indemnity clause. 
    The court here disagreed, holding that Rollins had silently assented to 
    the terms of the contract. ``~.~.~.~when an offeree accepts the offeror's 
    services without expressing any objections to the offer's essential terms, 
    the offeree has manifested assent to those terms.''\footnote{Casebook p. 
    478.}
    \item Affirmed.
\end{enumerate}

\subsubsection{The Effect of Using a Subcontractor's Bid}

\paragraph{\emph{Holman Erection Co. v. Orville E. Madsen \& Sons, Inc.}}

\begin{enumerate}
    \item Madsen was a general contractor. Holman submitted a bid as a 
    subcontractor, which Madsen included in its general bid. Madsen won the 
    bid and awarded the subcontract to a different company. Holman sued for 
    lost profits, arguing that inclusion of Holman's bid in Madsen's general 
    bid created a binding contract.
    \item The trial court held that there had been no contract. It granted 
    Madsen's motion for summary judgment.
    \item ``Does the act of listing Holman in the general bid constitute an 
    acceptance of Holman's offer to do the work when no other communication 
    occurred after the offer and prior to the substitution of a different 
    subcontractor? We think not.''\footnote{Casebook p. 480.}
    \item Holman argued that listing it as a subcontract counted as acceptance 
    because (1) there is no other reasonable explanation, (2) it is unfair to 
    bind Holman without binding Madsen, and (3) Madsen knew that its general 
    bid was public record, so it should have expected that Holman would have 
    seen it.
    \item The court considered several reasons for binding general contractors 
    to subcontractors upon listing their bids,\footnote{Casebook p. 482.} but 
    ultimately held that the justifications for ``unequal treatment of 
    generals and subcontractors'' were more persuasive.\footnote{Casebook p. 
    483.}
    \item Affirmed.
\end{enumerate}

\paragraph{\emph{Southern California Acoustics, Inc. v. C.V. Holder, Inc.}}

\begin{enumerate}
    \item % TODO 485
\end{enumerate}

\subsubsection{Silence As Acceptance}

\paragraph{\emph{Vogt v. Madden}}

\begin{enumerate}
    \item Vogt had an oral sharecrop agreement with Madden to farm Madden's 
    land in 1979 and 1980. The issue was whether there was an agreement to 
    farm the land in 1981. The crops had been doing poorly. Vogt proposed 
    growing beans. Madden did not respond. Vogt took Madden's silence as an 
    acceptance. Madden did not, and he leased the land to another tenant for 
    1981.
    \item The trial court held for Vogt.
    \item Can silence constitute acceptance?
    \begin{enumerate}
        \item By default, silence does not constitute acceptance.
        \item Restatement Second \S\ 69 recognizes two exceptions: (1) when 
        the offeree silently takes offered benefits and (2) ``where one party 
        relies on the other party's manifestation of intention that silence 
        may operate as acceptance.''\footnote{Casebook p. 493.}
        \item Neither exception applied here.
    \end{enumerate}
    \item Madden's silence was not assent, so there was no contract. Reversed.
\end{enumerate}

\paragraph{\emph{Laurel Race Courses v. Regal Const. Co.}}

\begin{enumerate}
    \item % TODO 495
\end{enumerate}

\paragraph{\emph{Cole-McIntyre-Norfleet Co. v. Holloway}}

\begin{enumerate}
    \item % TODO 495-497
\end{enumerate}

\paragraph{\emph{Kukusa v. Home Mut. Hail-Tornado Ins. Co.}}

\begin{enumerate}
    \item % TODO 498
\end{enumerate}

\paragraph{\emph{Hobbs v. Massasoit Whip Co.}}

\begin{enumerate}
    \item % TODO 498-499
\end{enumerate}

\paragraph{\emph{Louisville Tin \& Stove Co. v. Lay}}

\begin{enumerate}
    \item % TODO 499
\end{enumerate}

\paragraph{\emph{Austin v. Burge}}

\begin{enumerate}
    \item % TODO 499-500
\end{enumerate}

\paragraph{Negative-Option Plans}

\begin{enumerate}
    \item % TODO 500
\end{enumerate}

\paragraph{The Significance of Unjust Enrichment and Loss through Reliance in 
Cases Where Silence Is Treated as Acceptance}

\begin{enumerate}
    \item % TODO 500-501
\end{enumerate}

\subsubsection{Acceptance by Electronic Agent}

% TODO 501-503

\subsection{Implied-in-Law and Implied-in-Fact Contracts; Unilateral Contracts 
Revisited}

\paragraph{\emph{Nursing Care Services, Inc. v. Dobos}}

\begin{enumerate}
    \item % TODO 504-506
\end{enumerate}

\paragraph{\emph{Sceva v. True}}

\begin{enumerate}
    \item % TODO 506-07
\end{enumerate}

\paragraph{Terminology}

\begin{enumerate}
    \item % TODO 507
\end{enumerate}

\paragraph{Implied-in-Fact and Implied-in-Law Contracts}

\begin{enumerate}
    \item % TODO 508-10
\end{enumerate}

\paragraph{Morrison, ``I Imply What You Infer Unless You Are a Court''}

\begin{enumerate}
    \item % TODO 510-11
\end{enumerate}

\paragraph{\emph{Day v. Caton}}

\begin{enumerate}
    \item % TODO 511-13
    \item % TODO note 513-14
\end{enumerate}

\paragraph{\emph{Bastian v. Gafford}}

\begin{enumerate}
    \item % TODO 514-15
\end{enumerate}

\paragraph{\emph{Hill v. Waxberg}}

\begin{enumerate}
    \item % TODO 515
\end{enumerate}

\paragraph{\emph{Ramsey v. Ellis}}

\begin{enumerate}
    \item % TODO 515-16
\end{enumerate}

\paragraph{\emph{Paffhausen v. Balano}}

\begin{enumerate}
    \item % TODO 516
\end{enumerate}

\paragraph{\emph{Pine River State Bank v. Mettille}}

\begin{enumerate}
    \item % TODO 531
\end{enumerate}

\paragraph{Modification of Employee Handbooks}

\begin{enumerate}
    \item % TODO 531-33
\end{enumerate}

\paragraph{The Effect of Disclaimers in Employee Handbooks}

\begin{enumerate}
    \item % TODO 533-34
\end{enumerate}

\paragraph{Pettit, ``Modern Unilateral Contracts''}

\begin{enumerate}
    \item % TODO 534-35
\end{enumerate}

\subsection{Preliminary Negotiations, Indefiniteness, and the Duty to Bargain 
in Good Faith}

% TODO 536ff

\subsection{The Parol Evidence Rule and the Interpretation of Written 
Contracts}

% TODO 584

