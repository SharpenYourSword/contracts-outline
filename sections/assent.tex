\section{Assent}

\subsection{Introduction to Interpretation}

\subsubsection{Subjectivity and Objectivity}

\begin{enumerate}
    \item Contracts generally require a ``meeting of the 
    minds.''\footnote{Casebook p. 368.}
    \item What happens in cases of misinterpretation? There's a tension 
    between subjective and objective intent.
\end{enumerate}

\paragraph{Restatement First \S\ 227}

\begin{enumerate}
    \item There are ``six conceivable standards of interpretation'':
    \begin{enumerate}
        \item General usage.
        \item Limited usage---e.g., trade meaning.
        \item Mutual usage---e.g., a meaning common to both parties, even 
        if others don't follow the same meaning.
        \item Individual usage---what the speaker intended, or, what the 
        hearer understood.
        \item Reasonable expectation---the meaning that the speaker reasonably 
        should have expected his words to convey.
        \item Reasonable understanding.
    \end{enumerate}
\end{enumerate}

\paragraph{\emph{Lucy v. Zehmer}}

\begin{enumerate}
    \item Lucy had been trying to buy Zehmer's farm. After a few drinks, 
    Zehmer offered to sell for \$50,000, and Lucy accepted. Zehmer wrote a 
    contract on a napkin, had his wife sign it, and gave it to Lucy.
    \item The court held that it ``must look to the outward expression of a 
    person as manifesting his intention rather than to his secret and 
    unexpressed intention.''\footnote{Casebook p. 373.} Even if Zehmer did not 
    subjectively intend to sell the farm, his conduct gave rise to a 
    reasonable interpretation that he did intend to sell. The court held that 
    the sale contract was binding.
\end{enumerate}

\paragraph{\emph{Keller v. Holderman}}

\begin{enumerate}
    \item No contract is made when ``the whole transaction between the parties 
    was a frolic and a banter.''\footnote{Casebook p. 374.}
\end{enumerate}

\paragraph{\emph{Raffles v. Wichelhaus}}

\begin{enumerate}
    \item Defendants agreed to buy cotton from the Plaintiffs from a ship 
    named Peerless arriving at Liverpool from Bombay.
    \item Another ship named Peerless arrived at Liverpool from Bombay. 
    Defendants refused to pay because it was the wrong ship.
    \item Multiple ships of the same name and origin created a ``latent 
    ambiguity.'' In light of that ambiguity, the court accepted evidence that 
    the defendants meant one Peerless and the defendants another. Held for the 
    defendants.
\end{enumerate}

\paragraph{Simpson, ``Contracts for Cotton to Arrive: The Case of the Two 
Ships \emph{Peerless}}

\begin{enumerate}
    \item At the time of \emph{Raffles}, there were at least eleven ships 
    named Peerless, so some confusion was understandable.
\end{enumerate}

\paragraph{``Chicken'': \emph{Frigaliment Importing Co. v. B.N.S. Intern. 
Sales Co.}}

\begin{enumerate}
    \item ``The issue is, what is chicken?''
    \item The parties contracted for shipment of chicken, but they did not 
    specify what type, and a dispute arose about whether the agreement meant 
    young chicken, suitable for broiling and frying, or stewing chicken. The 
    court looked to several sources to determine the meaning of ``chicken,'' 
    including the language of the contract, trade usage, and Department of 
    Agriculture regulations.
    \item The court ultimately held that the plaintiff failed to meet its 
    burden of showing that it intended the narrower, rather than the broader, 
    definition of chicken.
\end{enumerate}

\paragraph{\emph{Oswald v. Allen}}

\begin{enumerate}
    \item Oswald was interested in Mrs. Allen's coin collection. He agreed to 
    buy her Swiss coins. He thought he had agreed to buy all of her Swiss 
    coins. She thought she had agreed to sell only her Swiss Coin Collection, 
    and not the other Swiss coins in her Rarity Coin Collection.
    \item The court held that this case was ``within the small group of 
    exceptional cases in which there is \enquote{no sensible basis for 
    choosing between conflicting understandings.}''
\end{enumerate}

\paragraph{\emph{Falck v. Williams}}

\begin{enumerate}
    \item The parties used a code for business communications. After an 
    ambiguous communication, the plaintiff sued for breach of contract. The 
    court held for the defendant on the ground that neither party's 
    interpretation was the true one.
\end{enumerate}

\paragraph{\emph{Colfax Envelope Corp. v. Local No. 458-3M}}

\begin{enumerate}
    \item ``If neither party can be assigned the greater blame for the 
    misunderstanding, there is no nonarbitrary basis for deciding which 
    party's understanding to enforce, so the parties are allowed to abandon 
    the contract without liability.''\footnote{Casebook p. 381.}
\end{enumerate}

\paragraph{Intent: \emph{Embry v. Hargadine, McKittrick Dry Goods Co.}}

Contracts generally require a meeting of the minds, but intent is irrelevant 
if the other party could not reasonably know the other's intent.

\begin{enumerate}
    \item Embry supported the company's sales team. After his one-year 
    contract was up, Embry tried to get McKittrick to agree to another 
    contract. After a brief conversation in McKittrick's office where 
    McKittrick said, ``Go ahead, you're all right. Get your men 
    out~.~.~.~''\footnote{Casebook p. 382.} Embry understood his words as 
    assenting to a new one-year contract. McKittrick didn't.
    \item The key issue was whether both parties intended to create a 
    contract.
    \item Contracts generally require a meeting of the minds, but intent is 
    irrelevant if the other party could not reasonably know the other's 
    intent.
    \item Held for Embry.
\end{enumerate}

\paragraph{Restatement Second \S\S\ 20, 201}

\begin{enumerate}
    \item % TODO supp
\end{enumerate}

\paragraph{CISG and Intent: \emph{MCC-Marble Ceramic Center, Inc. v. 
Ceramica Nuova D'Agostino}}

CISG allows the court to discern the parties' subjective intent, even absent 
outward manifestations.

Would this case have come out differently if US law controlled?

\begin{enumerate}
    \item MCC contracted to buy tiles from D'Agostino. They signed a form 
    contract.
    \item MCC brought suit for D'Agostino's failure to fill orders. D'Agostino 
    argued that it was not obligated to fill the orders because MCC had 
    defaulted on earlier payments.
    \item The form contract gave D'Agostino the right to cancel if MCC failed 
    to pay. MCC argued that it failed to pay because some of the tiles were 
    unsatisfactory---but the contract also had a clause requiring written 
    notice of complaints, which MCC did not give.
    \item MCC argued that it had an oral agreement with D'Agostino that the 
    clauses in the form contract would not apply.
    \item CISG allows inquiry into subjective intent, ``even if the parties 
    did not engage in any objectively ascertainable means of registering this 
    intent.''\footnote{Casebook p. 387.}
    \item Three affiants testified to MCC's intent to nullify the clauses. 
    While the affidavits may have been conclusory, they at least presented a 
    triable issue of fact, so summary judgment was inappropriate.
\end{enumerate}

\paragraph{\emph{Mayol v. Weiner Companies, Ltd.}}

\begin{enumerate}
    \item Mayol contracted to buy a piece of property which Weiner sold on 
    behalf of the owner. The contract included a clause granting possession 
    ``on or before November 1, 1979 subject to tenant's rights.'' Mayol paid a 
    \$1,000 deposit.
    \item It turned out that the tenant had a right to purchase. Upon learning 
    this, Mayol breached and sued to recover his deposit.
    \item Mayol had asked about the lease but Weiner had not told him about 
    the tenant's purchase option until after the sale contract was complete. 
    The court held for Mayol, reasoning that he had no reason to believe that 
    he was buying property subject to a purchase option, and the seller had no 
    reason to think that Mayol believed so, either.
\end{enumerate}

\paragraph{Objective and Subjective Elements in Interpretation}

\begin{enumerate}
    \item Classical contract law largely disregarded the parties' intent.
    \item There are four principles of interpretation in modern contract 
    law:\footnote{Casebook p. 394--95.}
    \begin{enumerate}
        \item The more reasonable meaning prevails If both parties attach 
        different subjective meanings to an expression  and they 
        are not equally reasonable.
        \item But if the two meanings are equally reasonable, neither 
        prevails.
        \item If the parties attach the same meaning, that meaning prevails 
        even if it is unreasonable.
        \item If A and B attach different meanings, and A knows B's meaning 
        but B doesn't know A's, B's meaning prevails even if it is less 
        reasonable.
    \end{enumerate}
\end{enumerate}

\paragraph{\emph{Berke Moore Co. v. Phoenix Bridge Co.}}

\begin{enumerate}
    \item Mutual understanding is not private and is therefore valid.
\end{enumerate}

\subsubsection{Problems of Interpreting Purposive Language}

\paragraph{Fish, ``Normal Circumstances, Literal Language, Direct Speech 
Acts, the Ordinary, the Everyday, the Obvious, What Goes Without Saying, and 
Other Special Cases''}

\begin{enumerate}
    \item There is no single, literal meaning of a phrase.\footnote{Casebook 
    pp. 398--99.}
\end{enumerate}

\paragraph{\emph{Haines v. New York}}

\begin{enumerate}
    \item The parties disagreed over whether New York City had an obligation 
    to expand a sewage system arising from a much earlier agreement.
    \item ``~.~.~.~where the parties have not clearly expressed the duration 
    of a contract, the courts will imply that they intended performance to 
    continue for a reasonable time.''\footnote{Casebook p. 401.}
\end{enumerate}

\paragraph{Restatement Second \S\ 204}

\begin{enumerate}
    \item % TODO supp
\end{enumerate}

\paragraph{\emph{Spaulding v. Morse}}

\begin{enumerate}
    \item A divorce agreement included a promise by Morse to pay his son 
    \$1,200 per year until he entered college. His son was drafted into the 
    army and tried to enforce the agreement.
    \item The court held that Morse did not have to pay. The court held that 
    it should take ``material circumstances'' into account to honor the 
    parties' intent.
\end{enumerate}

\paragraph{\emph{Lawson v. Martin Timber Co.}}

\begin{enumerate}
    \item Lawson and Martin agreed that Lawson could cut timber from Martin's 
    land for two years, but in the event of high water, Lawson could get a 
    one-year extension. Lawson continued to cut after the two year period. 
    Martin sued.
    \item There had been high water during the two year period. The court held 
    for Martin.
    \item The dissent argued that the parties intended the extension to take 
    effect only if high water \emph{prevented} Lawson from harvesting the 
    timber. But in fact it only prevented him from harvesting for half the 
    time. During the other half, he could have easily harvested the lumber.
\end{enumerate}

\paragraph{Lieber, ``Legal and Political Hermeneutics''}

\begin{enumerate}
    \item Some amount of interpretation is always necessary---e.g., ``fetch 
    some soupmeat.''\footnote{Casebook p. 405.}
\end{enumerate}

\subsubsection{Usage, Course of Dealing, and Course of Performance}

\paragraph{Trade Usage: \emph{Foxco Industries, Ltd. v. Fabric World, Inc.}}

\begin{enumerate}
    \item The parties disagreed over the meaning of the term ``first quality 
    goods.'' The court, following UCC 2-202(a), held that the trade usage 
    applied.\footnote{Casebook pp. 407--08.}
\end{enumerate}

\paragraph{Restatement Second \S\ 221, Illustration 2}

\begin{enumerate}
    \item % TODO 409
\end{enumerate}

\paragraph{Restatement Second \S\ 222}

\begin{enumerate}
    \item % TODO supp
\end{enumerate}

\paragraph{Restatement Second \S\ 222, Illustrations 1--3}

\begin{enumerate}
    \item % TODO 409
\end{enumerate}

\paragraph{Trade Quantities: \emph{Hurst v. W.J. Lake \& Co.}}

\begin{enumerate}
    \item ``Minimum 50\% protein'' included levels of 49.5\% protein, just as 
    ``a thousand bricks'' only means a wall of a certain size, and 4,000 
    shingles can really mean \$2,500.
\end{enumerate}

\paragraph{Restatement Second \S\ 220, Illustration 9}

\begin{enumerate}
    \item % TODO 410
\end{enumerate}

\paragraph{\emph{Flower City Painting Contractors, Inc. v. Gumina}}

\begin{enumerate}
    \item % FIXME 411
\end{enumerate}

\paragraph{UCC \S\S\ 1-201(3), (11), 1-205, 2-208}

\begin{enumerate}
    \item % TODO supp
\end{enumerate}

\paragraph{Restatement Second Contracts \S\S\ 219--223}

\begin{enumerate}
    \item % TODO supp
\end{enumerate}

\paragraph{J. White \& R. Summers, UCC \S\ 1-2}

\begin{enumerate}
    \item % FIXME 412
\end{enumerate}

\subsection{Offer and Revocation}

\subsubsection{What Constitutes an Offer}

% FIXME 413-424

\subsubsection{Lapse, Rejection, and Counter-Offer}

\paragraph{\emph{Akers v. J.B. Sedberry, Inc.}}

\begin{enumerate}
    \item % FIXME 
\end{enumerate}

\paragraph{\emph{Effect of the Rejection of an Offer}}

\begin{enumerate}
    \item When does an offeree terminate his power of acceptance?
    \item Generally, it's when the offeror would reasonably understand that 
    the offeree has taken the offer off the table.
\end{enumerate}

\paragraph{Qualified vs. Absolute Acceptance: \emph{Ardente v. Horan}}
~\\\\
A qualified acceptance, subject to a condition, does not create a contractual 
obligation if the other party does not satisfy the condition.

\begin{enumerate}
    \item Ardente tried to buy property from Horan. He sent a \$20,000 deposit 
    with a letter asking the sellers to leave behind certain furnishings 
    (patio furniture, etc.). Horan declined and returned the deposit. Ardente 
    sued for specific performance.
    \item The issue was whether Ardente's deposit and letter constituted a 
    contract, hinging on the condition in the letter. Was it a qualified 
    acceptance, subject to a condition, or an ``absolute acceptance together 
    with a mere inquiry concerning a collateral matter''?\footnote{Casebook 
    p. 432.}
    \item The court held that Ardente's letter was a qualified acceptance, 
    binding only if Horan met its conditions. Thus, ``it operated as a 
    rejection of the defendants' offer and no contractual obligation was 
    created.''\footnote{Casebook p. 432.}
\end{enumerate}

\paragraph{\emph{Rhode Island Dep't of Transp. v. Providence \& Worcester 
R.R.}}

\begin{enumerate}
    \item % FIXME 433
\end{enumerate}

\paragraph{\emph{Price v. Oklahoma College of Osteopathic Medicine and Surgey}}

\begin{enumerate}
    \item % FIXME 434
\end{enumerate}

\paragraph{Mirror-Image Rule}

\begin{enumerate}
    \item % FIXME 435
\end{enumerate}

\paragraph{\emph{Livingstone v. Evans}}

\begin{enumerate}
    \item % FIXME 436
\end{enumerate}

\paragraph{\emph{Culton v. Gilchrist}}

\begin{enumerate}
    \item % FIXME 436
\end{enumerate}

\paragraph{Effect of the Offeror's Death or Incapacity Before Acceptance}

\begin{enumerate}
    \item % FIXME 436
\end{enumerate}

\subsubsection{Revocation}

\paragraph{\emph{Dickinson v. Dodds}}

\begin{enumerate}
    \item % FIXME 438
\end{enumerate}

\paragraph{Contract Practice}

\begin{enumerate}
    \item % FIXME 441
\end{enumerate}

\paragraph{What Constitutes Receipt of Written Acceptance}

\begin{enumerate}
    \item % FIXME 441
\end{enumerate}

\paragraph{\emph{Ragosta v. Wilder}}

\begin{enumerate}
    \item % FIXME 442
\end{enumerate}

\paragraph{Offers for Unilateral Contracts}

\begin{enumerate}
    \item % FIXME 446
\end{enumerate}

\paragraph{Reliance and Revocability: \emph{Drennan v. Star Paving Co.}}
~\\\\
Promissory estoppel prevents a subcontractor from revoking its offer once the 
contractor has acted upon the subcontractor's promise.

\begin{enumerate}
    \item Facts:
    \begin{enumerate}
        \item Drennan was preparing a bid for a school construction job. His 
        final bid was \$317,385. He had to provide a 10\% bond as a guarantee 
        that he would enter the contract.
        \item On July 28, 1955 Drennan received between 50 and 75 
        subcontractor bids. Star Paving Co. put in a bid for \$7,131.60 for 
        paving work.
        \item Drennan's was the winning bid. The next day, he stopped by Star 
        Paving's offices, where one of its engineers told him they had 
        mistakenly underbid. The had intended to bid \$15,000 and refused to 
        do the job for anything less.
        \item Drennan found another company, L \& H, to do the paving for 
        \$10,948.60. Drennan then sued Star for the difference, or \$3,817.00.
    \end{enumerate}
    \item Star argued that ``there was no enforceable contract between the 
    parties on the ground that it made a revocable offer and revoked it before 
    plaintiff communicated his acceptance to defendant.''\footnote{Casebook p. 
    449.}
    \item Drennan argued that ``he relied to his detriment on defendant's 
    offer and that defendant must therefore answer in damages for its refusal 
    to perform.''
    \item The trial court held that Star made a definite offer and that 
    Drennan relied on Star's bid in computing his own bid. It awarded 
    \$3,817.00 in reliance damages to Drennan.
    \item Earlier cases held that the subcontractor is not bound because there 
    was no binding promise of an irrevocable offer and no consideration.
    \item On appeal, the question was whether Drennan's reliance made Star's 
    offer irrevocable.
    \item Judge Traynor:
    \begin{enumerate}
        \item Star's offer was a promise. Since Star was silent on revocation, 
        the court had to decide whether the promise was revocable.
        \item Restatement \S\ 90 hold that justifiable reliance can 
        make a promise binding, even absent bargained-for 
        consideration.\footnote{Casebook p. 450. This case was decided in 
        1958. The ALI published the Restatement in 1932 and Restatement Second 
        in 1979.}
        \item Star should have known that Drennan would have relied on its 
        promise in computing its bid---indeed, it wanted him 
        to.\footnote{Casebook p. 450--51.}
        \item Star also argued that its bid was the result of a mistake, so it 
        was entitled to revoke it. The court held that it would be revocable 
        only if Drennan had not relied on it. But here, Star's mistake misled 
        Drennan, causing further detriment and creating an additional reason 
        to enforce the promise.
        \item Star's last argument was that Drennan failed to mitigate 
        damages. The court dismissed this argument, finding that he had.
    \end{enumerate}
\end{enumerate}

\paragraph{Critiques of \emph{Drennan}: \emph{Pavel Enterprises, Inc. v. A.S. 
Johnson Co. 452}}

\begin{enumerate}
    \item Subcontractors are bound to the general, but the general is not 
    bound to the subcontractor, creating incentives for the general contractor 
    to act unethically:
    \begin{enumerate}
        \item \emph{Bid shopping}: using the lowest bid to negotiate lower 
        bids from others.
        \item \emph{Bid chopping}: pressuring the subcontractor to make a 
        lower bid.
        \item \emph{Bid peddling}: a subcontractor waits until other bids are 
        in and then undercutting them, avoiding the cost of estimating his own 
        bid.
    \end{enumerate}
    \item Most courts have followed \emph{Drennan}, but at least one as 
    deviated.\footnote{Casebook p. 453.}
\end{enumerate}

\paragraph{Dodge, ``Teaching the CISG in Contracts''}

\begin{enumerate}
    \item Common law: ``an offer is freely revocable, even if the offeror has 
    promised to hold it open, unless that promise is supported by 
    consideration or reliance.''\footnote{Casebook p. 453.}
    \item UCC: merchants can make a ``firm offer'' (i.e., an irrevocable 
    offer) without the need for consideration. The offeror must be a merchant, 
    etc.\footnote{Casebook p. 453.}
    \item CISG Art. 16 allows an offeror to make an irrevocable offer without 
    these restrictions.
\end{enumerate}

\paragraph{Limiting \emph{Drennan}: \emph{Preload Technology, Inc. v. A.B. \& 
J. Construction Co., Inc.}}

\begin{enumerate}
    \item When the general contractor engages in the practices warned against 
    in \emph{Pavel Enterprises} (bid shopping, etc.), \S\ 90 reliance may not 
    be available.
\end{enumerate}

\paragraph{Restatement Second \S\ 87(2): Option Contract}

\begin{enumerate}
    \item ``An offer which the offeror should reasonably expect to induce 
    action or forbearance of a substantial character on the part of the 
    offence before acceptance and which does induce such action or forbearance 
    is binding as an option contract to the extent necessary to avoid 
    injustice.''
    \item An option contract is an offer in which the offeror promises to keep 
    the offer open for a certain period of time. For instance, a seller grants 
    a buyer the option to buy his house for \$1,000 anytime during the next 
    month.
    \item The distinction between \S\S\ 45 and 87 is that an offeree who has 
    begun performance can recover expectation damages, while an offeree who 
    has not begun performance can only recover reliance 
    damages.\footnote{Casebook p. 455.}
\end{enumerate}

\subsection{Modes of Acceptance}

\subsubsection{Acceptance by Act}

\paragraph{Promise to Bequeath: \emph{Klockner v. Green}}
~\\\\
A promise becomes binding when the offeree acts on the offeror's request.

\begin{enumerate}
    \item Richard and Francis Klockner were stepson and stepgranddaughter of 
    the decedent, Edyth Klockner. Edith Klockner left a will devising her 
    property to her husband, who predeceased her, so upon her death her 
    property would have passed to her heirs by the rules of intestate 
    succession.
    \item However, Edyth Klockner promised to leave her real property to 
    Richard and her personal property to Francis. She had her attorney draw up 
    a will to this effect, but she never signed it---``stymied by her own 
    superstition.''\footnote{Casebook p. 464.}
    \item Richard and Francis testified that they would have taken care of 
    Edyth even if she hadn't promised to leave them her property. The trial 
    court held that there was no contract because there was no offer and 
    acceptance nor consideration. The appellate court held that the statute 
    of frauds barred enforcement because there had been no reliance.
    \begin{enumerate}
        \item Statute of frauds: at common law, it held that contracts are not 
        enforceable if they are not in writing.
    \end{enumerate}
    \item The question on appeal was whether Edyth Klockner entered into a 
    valid, binding contract with Richard and Francis to bequeath her property 
    to them.
    \item Held: Edyth's promise became binding when Richard and Francis acted 
    upon it. It need not have been their sole motivation. Edyth also received 
    the benefit of her bargain.
\end{enumerate}

\paragraph{\emph{De Cicco v.  Schweizer}}

\begin{enumerate}
    \item % FIXME 466
\end{enumerate}

\paragraph{\emph{Simmons v. United States}}

\begin{enumerate}
    \item % FIXME 465
\end{enumerate}

\paragraph{\emph{Stephens v. Memphis}}

\begin{enumerate}
    \item % FIXME 466
\end{enumerate}

\paragraph{Performance of a Condition as Acceptance: \emph{Carlill v. Carbolic 
Smoke Ball, Inc.}}
~\\\\
Performing the conditions counts as acceptance of the offer. For 
instance, if the Carbolic Smoke Ball manufacturers promise a reward for anyone 
who uses the product and gets sick, anyone who performs those conditions has 
accepted the offer and can recover the reward.

\begin{enumerate}
    \item The manufacturers of ``The Carbolic Smoke Ball'' placed a newspaper 
    ad promising a cash reward for anyone who became sick after using their 
    product three times daily for two weeks. The plaintiff did and got sick. 
    She sued for the reward.
    \item The trial court held that she could recover the value of the reward.
    \item Was this a promise or ``mere puff''?\footnote{Casebook p. 467.} It 
    was a promise.
    \item Was it binding? The defendants argued that the promise was made to 
    nobody in particular. The court held that the ad was an offer to pay the 
    reward to ``anyone who will perform these conditions, and the performance 
    of the conditions is the acceptance of the offer.''\footnote{Casebook p. 
    467.}
    \item The defendants also argued that there was no notice requirement. The 
    court held that notice was irrelevant.
    \item The defendants argued, finally, that there was no consideration. The 
    court disagreed, finding that any use of their product conferred a 
    benefit. Moreover, consumers are inconvenienced when they use the Carbolic 
    Smoke Ball.
    \item Affirmed.
\end{enumerate}

\subsubsection{Subjective Acceptance}

\paragraph{\emph{International Filter Co. v. Conroe Gin, Ince, \& Light Co.}}

\begin{enumerate}
    \item International Filter manufactured water purifying machinery. Conroe 
    made ice, etc. International Filter sent Conroe a proposal for the sale of 
    two water tanks. The proposal dictated that it ``becomes a contract when 
    accepted by the purchaser and approved by an executive officer of 
    International Filter Company~.~.~.''\footnote{Casebook p. 471.}
    \item Conroe wrote back with ``Accepted.'' The president of International 
    Filter wrote ``O.K.'' International then sent Conroe a confirmation.
    \item Conroe later tried to back out. International Filter sued for 
    performance.
    \item Conroe argued (1) that International's ``O.K.'' did not amount to an 
    approval by an executive and (2) that International did not notify Conroe 
    of its acceptance of the contract.\footnote{Casebook p. 472.}
    \item The trial court found for Conroe. Affirmed on appeal.
    \item The court here held that International was not required to 
    communicate its acceptance. As long as an executive at International 
    approved the order, notice of the approval to Conroe was not necessary.
    \item Moreover, even if notice had been required (which it wasn't), 
    International's ``O.K.'' would have been good enough.
    \item Reversed.
\end{enumerate}

\subsubsection{Acceptance by Conduct}

\paragraph{Silence as Assent: \emph{Polaroid Corp. v. Rollins Environmental 
Services (NJ), Inc.}}

\begin{enumerate}
    \item Rollins operated a hazardous waste disposal facility. Rollins 
    disposed of hazardous waste from Polaroid and Occidental.
    \item Hazardous waste disposal agreements between Polaroid/Occidental and 
    Rollins included an indemnification for liability for spills. Occidental 
    included its indemnity clause in its purchase orders. Rollins accepted 
    these purchase orders several times before it objected.
    \item Polaroid and Occidental sued for a determination that Rollins was 
    obligated to indemnify them against liability for hazardous waste spills. 
    \item The trial court held that Polaroid and Occidental had valid 
    contracts with Rollins indemnifying them from liability.
    \item The New Jersey Department of Environmental Protection requested 
    compensation of \$9,224,189 for cleanup.
    \item Rollins argued that it was not obligated to indemnify Occidental 
    because it refused the purchase orders containing the indemnity clause. 
    The court here disagreed, holding that Rollins had silently assented to 
    the terms of the contract. ``~.~.~.~when an offeree accepts the offeror's 
    services without expressing any objections to the offer's essential terms, 
    the offeree has manifested assent to those terms.''\footnote{Casebook p. 
    478.}
    \item Affirmed.
\end{enumerate}

\subsubsection{The Effect of Using a Subcontractor's Bid}

\paragraph{\emph{Holman Erection Co. v. Orville E. Madsen \& Sons, Inc.}}

\begin{enumerate}
    \item Madsen was a general contractor. Holman submitted a bid as a 
    subcontractor, which Madsen included in its general bid. Madsen won the 
    bid and awarded the subcontract to a different company. Holman sued for 
    lost profits, arguing that inclusion of Holman's bid in Madsen's general 
    bid created a binding contract.
    \item The trial court held that there had been no contract. It granted 
    Madsen's motion for summary judgment.
    \item ``Does the act of listing Holman in the general bid constitute an 
    acceptance of Holman's offer to do the work when no other communication 
    occurred after the offer and prior to the substitution of a different 
    subcontractor? We think not.''\footnote{Casebook p. 480.}
    \item Holman argued that listing it as a subcontract counted as acceptance 
    because (1) there is no other reasonable explanation, (2) it is unfair to 
    bind Holman without binding Madsen, and (3) Madsen knew that its general 
    bid was public record, so it should have expected that Holman would have 
    seen it.
    \item The court considered several reasons for binding general contractors 
    to subcontractors upon listing their bids,\footnote{Casebook p. 482.} but 
    ultimately held that the justifications for ``unequal treatment of 
    generals and subcontractors'' were more persuasive.\footnote{Casebook p. 
    483.}
    \item Affirmed.
\end{enumerate}

\paragraph{\emph{Southern California Acoustics, Inc. v. C.V. Holder, Inc.}}

\begin{enumerate}
    \item % FIXME 485
\end{enumerate}

\subsubsection{Silence As Acceptance}

\paragraph{\emph{Vogt v. Madden}}

\begin{enumerate}
    \item Vogt had an oral sharecrop agreement with Madden to farm Madden's 
    land in 1979 and 1980. The issue was whether there was an agreement to 
    farm the land in 1981. The crops had been doing poorly. Vogt proposed 
    growing beans. Madden did not respond. Vogt took Madden's silence as an 
    acceptance. Madden did not, and he leased the land to another tenant for 
    1981.
    \item The trial court held for Vogt.
    \item Can silence constitute acceptance?
    \begin{enumerate}
        \item By default, silence does not constitute acceptance.
        \item Restatement Second \S\ 69 recognizes two exceptions: (1) when 
        the offeree silently takes offered benefits and (2) ``where one party 
        relies on the other party's manifestation of intention that silence 
        may operate as acceptance.''\footnote{Casebook p. 493.}
        \item Neither exception applied here.
    \end{enumerate}
    \item Madden's silence was not assent, so there was no contract. Reversed.
\end{enumerate}

\paragraph{\emph{Laurel Race Courses v. Regal Const. Co.}}

\begin{enumerate}
    \item % FIXME 495
\end{enumerate}

\paragraph{\emph{Cole-McIntyre-Norfleet Co. v. Holloway}}

\begin{enumerate}
    \item % FIXME 495-497
\end{enumerate}

\paragraph{\emph{Kukusa v. Home Mut. Hail-Tornado Ins. Co.}}

\begin{enumerate}
    \item % FIXME 498
\end{enumerate}

\paragraph{\emph{Hobbs v. Massasoit Whip Co.}}

\begin{enumerate}
    \item % FIXME 498-499
\end{enumerate}

\paragraph{\emph{Louisville Tin \& Stove Co. v. Lay}}

\begin{enumerate}
    \item % FIXME 499
\end{enumerate}

\paragraph{\emph{Austin v. Burge}}

\begin{enumerate}
    \item % FIXME 499-500
\end{enumerate}

\paragraph{Negative-Option Plans}

\begin{enumerate}
    \item % FIXME 500
\end{enumerate}

\paragraph{The Significance of Unjust Enrichment and Loss through Reliance in 
Cases Where Silence Is Treated as Acceptance}

\begin{enumerate}
    \item % FIXME 500-501
\end{enumerate}

\subsubsection{Acceptance by Electronic Agent}

% FIXME 501-503

\subsection{Implied-in-Law and Implied-in-Fact Contracts; Unilateral Contracts 
Revisited}

\subsubsection{\emph{Nursing Care Services, Inc. v. Dobos}}

\begin{enumerate}
    \item % FIXME 504-506
\end{enumerate}

\subsubsection{\emph{Sceva v. True}}

\begin{enumerate}
    \item % FIXME 506-07
\end{enumerate}

\subsubsection{Terminology}

\begin{enumerate}
    \item % FIXME 507
\end{enumerate}

\subsubsection{Implied-in-Fact and Implied-in-Law Contracts}

\begin{enumerate}
    \item % FIXME 508-10
\end{enumerate}

\subsubsection{Morrison, ``I Imply What You Infer Unless You Are a Court''}

\begin{enumerate}
    \item % FIXME 510-11
\end{enumerate}

\subsubsection{\emph{Day v. Caton}}

\begin{enumerate}
    \item % FIXME 511-13
    \item % FIXME note 513-14
\end{enumerate}

\subsubsection{\emph{Bastian v. Gafford}}

\begin{enumerate}
    \item % FIXME 514-15
\end{enumerate}

\subsubsection{\emph{Hill v. Waxberg}}

\begin{enumerate}
    \item % FIXME 515
\end{enumerate}

\subsubsection{\emph{Ramsey v. Ellis}}

\begin{enumerate}
    \item % FIXME 515-16
\end{enumerate}

\subsubsection{\emph{Paffhausen v. Balano}}

\begin{enumerate}
    \item % FIXME 516
\end{enumerate}

\subsubsection{\emph{Pine River State Bank v. Mettille}}

\begin{enumerate}
    \item % FIXME 531
\end{enumerate}

\subsubsection{Modification of Employee Handbooks}

\begin{enumerate}
    \item % FIXME 531-33
\end{enumerate}

\subsubsection{The Effect of Disclaimers in Employee Handbooks}

\begin{enumerate}
    \item % FIXME 533-34
\end{enumerate}

\subsubsection{Pettit, ``Modern Unilateral Contracts''}

\begin{enumerate}
    \item % FIXME 534-35
\end{enumerate}

\subsection{Preliminary Negotiations, Indefiniteness, and the Duty to Bargain 
in Good Faith}

\begin{enumerate}
    \item Classical contract law drew a strict binary distinction between 
    offer and acceptance, which created a binding agreement, and preliminary 
    negotiations, which did not.\footnote{Casebook p. 536.}
    \item Agreements can be too indefinite to allow courts to fashion 
    remedies for breach, and are therefore unenforceable.
    \item What about agreements to create future agreements? Again classical 
    contract law followed a strict binary: on the one hand, if the future
    agreement was meant only to confirm the original agreement, the original 
    agreement was enforceable; on the other hand, if the parties intended not 
    to be bound unless they executed the future agreement, the original 
    agreement was not enforceable.
    \item Classical contract law recognized no duty to negotiate in good 
    faith.\footnote{Casebook p. 537.}
\end{enumerate}

\subsubsection{Uncertain and Indefinite: \emph{Academy Chicago Publishers v. 
Cheever}}

\begin{enumerate}
    \item The Cheevers entered into an agreement with Academy to publish the 
    John Cheever's uncollected stories. The agreement was vague as to the 
    collection's specifications and editorial control.
    \item Mrs. Cheever breached the agreement, apparently after realizing the 
    book's potential commercial value.
    \item Academy sued to (1) win the exclusive right to publish the 
    collection, (2) designate Franklin Dennis (who was with Mrs. Cheever) as 
    editor, and (3) obligate Mrs. Cheever to deliver the manuscript.
    \item The trial court held that the agreement was enforceable, but it 
    granted significant editorial control to Mrs. Cheever. The appellate court 
    affirmed in part and reversed in part.
    \item The sole issue before the Supreme Court of Illinois was whether the 
    original agreement was valid and enforceable. It held that ``[a]lthough 
    the parties may have had and manifested the intent to make a contract, if 
    the content of their agreement is unduly uncertain and indefinite no 
    contract is formed.''\footnote{Casebook p. 539.}
    \item The parties lacked a mutual understanding of the essential parts of 
    the agreement. The court cannot substitute the missing terms if they are 
    central to the contract.
    \item Reversed.
\end{enumerate}

\subsubsection{\emph{Ridgway v. Wharton}}

\begin{enumerate}
    \item ``An agreement to enter into an agreement upon terms to be 
    afterwards settled between the parties is a contradiction in 
    terms.''\footnote{Casebook p. 541.}
\end{enumerate}

\subsubsection{\emph{Berg Agency v. Sleepworld-Willingboro, Inc.}}

\begin{enumerate}
    \item If the essential parts of an agreement are included and agreed upon, 
    the contract is enforceable even if other clauses are 
    omitted.\footnote{Casebook p. 541.}
\end{enumerate}

\subsubsection{\emph{Rego v. Decker}}

\begin{enumerate}
    \item Courts should fill gaps in contracts where the reasonable 
    expectations of the parties are clear.
    \item But, ``courts should not impose on a party any performance to which 
    he did not and probably would not have agreed.''\footnote{Casebook p. 542.}
\end{enumerate}

\subsubsection{\emph{AROK Construction Co. v. Indian Construction Services}}

\begin{enumerate}
    \item Parties may want to create incomplete contracts to allow future 
    flexibility.
\end{enumerate}

\subsubsection{\emph{Saliba-Kringlen Corp. v. Allen Engineering Co.}}

\begin{enumerate}
    \item Allen submitted a subcontractor bid to the general contractor, 
    Saliba. Saliba submitted its bid in reliance on Allen's bid.
    \item Allen breached, arguing that its bid was too indefinite (and 
    therefore unenforceable) because contained only price and no other details 
    of the contract's performance.
    \item The court held for Saliba, finding that there is a general 
    understanding in the trade that subcontractors will enter into an 
    agreement with the general contractor after submitting a winning bid. If 
    this were not the case, general contractors would be unable to recover \S\ 
    90 reliance damages.
\end{enumerate}

\subsubsection{Gap-Fillers}

\begin{enumerate}
    \item The UCC ``gap-filler'' provisions fill the gaps that parties may 
    leave in contracts for sale of goods.
    \item \textbf{Default rules}: background rules that the law reads into a 
    contract only if the parties left them out.
    \item \textbf{Mandatory rules}: not waivable by the parties, e.g., the 
    requirement of consideration.
    \item There is controversy over whether the appropriate default rule 
    should be (1) what reasonable people in the parties' positions would have 
    agreed to or (2) what the parties themselves probably would have agreed to 
    given the circumstances. Fuller and Eisenberg favor the first because it 
    avoids the complexity of determining the parties' exact position (e.g., 
    bargaining power, degree of risk-averseness).\footnote{Casebook p. 545.}
\end{enumerate}

\subsubsection{Hawkland, ``Sales Contracts Terms under the UCC''}

\begin{enumerate}
    \item Gap-filler terms represent ``ordinary understanding'' used in 
    ``common, unexceptional deals.''\footnote{Casebook p. 545.}
\end{enumerate}

\subsubsection{Agreement to Agree: \emph{Joseph Martin, Jr., Delicatessen, 
Inc. v. Schumacher}}

\begin{enumerate}
    \item Martin agreed to rent from Schumacher for five years of payments 
    ranging from \$500/month to \$650/month with the option to renew for 
    another five years ``at annual rentals to be agreed 
    upon.''\footnote{Casebook p. 546.}
    \item Martin tried to renew after the first five-year term. Schumacher 
    demanded \$900/month in rent, although the fair market value was less than 
    \$550.
    \item The trial court held that the agreement was unenforceable for 
    uncertainty. The appellate court reversed, holding that the trial court 
    could set ``reasonable rent.''
    \item The New York Court of Appeals reversed again, holding that there was 
    no method specified in the original agreement about how to calculate rent. 
    As such, the trial court does not have the authority to determine what is 
    reasonable. There was no hint that either party intended to be bound by 
    fair market value.
\end{enumerate}

\subsubsection{\emph{Moolenaar v. Co-Build Companies, Inc.}}

\begin{enumerate}
    \item Moolenaar agreed to rent land at \$350/month, and upon renewal, rent 
    ``shall be renegotiated.''\footnote{Casebook p. 547.}
    \item Co-Build bought the property subject to Moolenaar's lease. At the 
    renewal period, it demanded \$17,000/month for the new lease.
    \item The court here held that Moolenaar was entitled to ``reasonable'' 
    rent because (1) it would match the parties' original intent and (2) the 
    parties probably paid valuable consideration for the original agreement, 
    e.g., the landlord benefited by inducing the tenant to pay higher rent on 
    the expectation that the lease would continue.
    \item Although this rule is the minority (cf. \emph{Joseph Martin, Jr., 
    Delicatessen} above), the law is moving in this 
    direction.\footnote{Casebook pp. 547--48.}
\end{enumerate}

\subsubsection{Good Faith Negotiations: \emph{Channel Home Centers v. 
Grossman}}

\begin{enumerate}
    \item Channel operated a chain of home improvement stores. Grossman owned 
    a mall. The parties began negotiations for leasing to Channel as an anchor 
    tenant in the mall.
    \item Channel sent a letter of intent indicating among other things that 
    Grossman would ``withdraw the Store from the rental market'' during 
    negotiations. Grossman agreed.
    \item Soon after, Grossman got a better offer from Mr. Good Buy, one of 
    Channel's competitors. He leased to Mr. Good Buy. Channel sued, claiming 
    that Grossman had violated the letter of intent.
    \item The trial court rejected Channel's argument that the letter of 
    intent constituted a binding contract because it did not create any 
    obligation or consideration.
    \item On appeal, Channel argued that the letter and the surrounding 
    circumstances created ``a binding agreement to negotiate in good 
    faith.''\footnote{Casebook p. 553.}
    \item The court here held that an agreement to negotiate in good faith is 
    binding if (1) both parties manifested an intention to be bound by the 
    agreement, (2) whether the agreement's terms were definite and therefore 
    enforceable, and (3) whether there was consideration. It held that the 
    agreement between Channel and Grossman met all of these requirements.
    \item Held, Grossman violated its obligation to negotiate in good faith.
\end{enumerate}

\subsection{The Parol Evidence Rule and the Interpretation of Written 
Contracts}

\subsubsection{The Parol Evidence Rule}

\paragraph{Thayer, ``A Preliminary Treatise on Evidence''}

\begin{enumerate}
    \item Some want a ``lawyer's Paradise'' where written instruments have a 
    precise, fixed meaning. But in reality, context is a ``fatal 
    necessity.''\footnote{Casebook p. 590.}
\end{enumerate}

\paragraph{Calamari \& Perillo, ``A Plea for a Uniform Parl Evidence Rule and 
Principles of Interpretation''}

\begin{enumerate}
    \item There is basic disagreement about the meaning of the parol evidence 
    rule and of the goals of contractual interpretation.
    \item There is substantial agreement that later final expressions replace 
    earlier tentative expressions.
    \item Disagreement arises when the last expression is not in writing. 
    Imagine A sells property to B on the condition, agreed orally, that A will 
    remove an unsightly shack from neighboring property. Can the oral promise 
    be enforced?
    \item The answer depends on whether the written agreement was a 
    \textbf{total integration}.\footnote{Casebook p. 592.}
    \begin{enumerate}
        \item \emph{Williston}: the existence of the oral agreement proves 
        that the written agreement was only a partial intergration. The only 
        question would be whether the parties actually made the alleged prior 
        oral agreement. A written expression is presumptively a complete 
        integration unless it would be natural for similarly situated parties 
        to add additional terms in a separate (oral) 
        agreement.\footnote{Casebook pp. 591--92.}
        \item \emph{Corbin}: Williston's approach---excluding relevant 
        evidence of intent except the writing itself---is absurd.
    \end{enumerate}
    \item The core of the debate is whether it is better to allow oral 
    evidence or to require the parties to put their entire agreement in 
    formal, written terms.\footnote{Casebook p. 592.}
\end{enumerate}

\paragraph{Restatement First \S\S\ 228, 237, 239, 240}

\begin{enumerate}
    \item \S\ 228: an integrated agreement is final and complete.
    \item \S\ 237: an integrated agreement nullifies all previous agreements 
    and all contemporaneous oral agreements.
    \item \S\ 239: effect of partial integration. % TODO --?
    \item \S\ 240: Integration does not supersede earlier agreements if it is 
    consistent and (1) made for separate consideration or (2) would naturally 
    be separate.
\end{enumerate}

\paragraph{Restatement Second \S\S\ 209, 210, 213--16}

\begin{enumerate}
    \item % TODO supp
\end{enumerate}

\paragraph{Braucher, ``Interpretation and Legal Effect in the Second 
Restatement of Contracts''}

\begin{enumerate}
    \item The difficulties in parol evidence lie in:\footnote{Casebook p. 
    593.}
    \begin{enumerate}
        \item Determining whether there is an integrated agreement.
        \item Whether the agreement is completely or partially integrated.
        \item Whether the prior agreement is consistent with and within the 
        scope of the integrated agreement.
    \end{enumerate}
\end{enumerate}

\paragraph{\emph{Hatley v. Stafford}}

\begin{enumerate}
    \item In determining whether oral terms would have naturally been included 
    in the written agreement, courts should look to contextual factors beyond 
    the face of the document, including the parties' business experience, 
    whether they are represented by counsel, their relative bargaining power, 
    and the completeness and detail of the writing itself.\footnote{Casebook 
    p. 593.}
    \item Courts should presume complete integration, ``and should admit 
    evidence of consistent additional terms only if there is substantial 
    evidence that the parties did not intend the writing to embody the entire 
    agreement.''\footnote{Casebook p. 594.}
\end{enumerate}

\paragraph{\emph{Interform Co. v. Mitchell Constr. Co.}}

\begin{enumerate}
    \item The debate between Corbin and Williston centers on ``the attitude 
    with which judges should approach written contracts.''\footnote{Casebook 
    p. 594.}
    \item Williston: the judge should ascertain the legal relations between 
    the parties through form--i.e., the parties' use of normal integration 
    practices and language. The contract means what a reasonably intelligent 
    person would understand them to mean.
    \item Corbin: the judge should ascertain the legal relations between the 
    parties on the basis of the parties' intent, regardless of their use of 
    forms. The contract means what the parties intended it to mean.
    \item Corbin's influence is stronger.\footnote{Casebook p. 594.}
\end{enumerate}

\paragraph{Murray, ``The Parol Evidence Process and Standardized Agreements 
under the Restatement, Second, Contracts''}

\begin{enumerate}
    \item The Restatement Second's rule in \S\ 209 determines  whether an 
    agreeement is integrated as ``a question of fact to be determined in 
    accordance with all relevant evidence.'' The Restatement follows Corbin, 
    who argued that on this issue, ``no relevant testimony should be 
    excluded.''\footnote{Casebook p. 595.}
\end{enumerate}

\paragraph{CISG Art. 8(3)}

\begin{enumerate}
    \item % TODO supp
\end{enumerate}

\paragraph{Dodge, ``Teaching the CISG in Contracts''}

\begin{enumerate}
    \item ``~.~.~.~the CISG lacks a parol evidence rule and allows a court 
    interpreting a written contract to consider not just trade usage, course 
    of dealing, and course of performance, but even parties' prior 
    negotiations.''\footnote{Casebook p. 595.}
\end{enumerate}

\paragraph{Unidroit Principles of International Law Art. 4.3}

\begin{enumerate}
    \item % TODO supp
\end{enumerate}

\paragraph{\emph{Masterson v. Sine}}

\begin{enumerate}
    \item The Mastersons conveyed a ranch to the Sines. Medora Sine was Dallas 
    Masterson's sister. The conveyance gave the Mastersons a purchase option 
    for the following ten years for the ``same consideration as being paid 
    heretofore plus their depreciation value of any improvements Grantees may 
    add to the property from and after two and a half years from this 
    date.''\footnote{Casebook p. 595.}
    \item Dallas Masterson went bankrupt. His trustee and his wife, Rebecca, 
    brought suit to establish their right to enforce the option.
    \item The trial court held that the Mastersons could exercise their 
    option.
    \item On appeal, the dispute centered on whether the trial court properly 
    excluded evidence that ``that the option was personal to the grantors and 
    therefore nonassignable.''\footnote{Casebook p. 596.}
    \item When an agreement is only partially integrated, parol evidence can 
    ``prove elements of the agreement not reduced to 
    writing.''\footnote{Casebook p. 596.}
    \item The option clause in the deed did not provide that it contained the 
    complete agreement.
    \item In this case, the alleged additional terms (that the option was 
    personal and nonassignable) might reasonably have left it out of the 
    written deed.
    \item Therefore, the trial court erred in excluding parol evidence of the 
    parties' agreement that the option was not assignable.
\end{enumerate}

\paragraph{UCC \S\ 2-202}

\begin{enumerate}
    \item % TODO supp
\end{enumerate}

\paragraph{\emph{Interform v. Mitchell Constr. Co.}}

\begin{enumerate}
    \item UCC \S\ 2-202 reflects Corbin's influence in its focus on the intent 
    of the parties, rather than the practices of reasonable people.
\end{enumerate}

\paragraph{\emph{Hunt Foods and Industries, Inc. v. Doliner}}

\begin{enumerate}
    \item Hunt began negotiations to buy Eastern Can Company. Doliner owned 
    the majority of Eastern's stock. It became necessary to pause the 
    negotiations. Doliner granted Hunt an option to purchase Doliner's stock 
    at \$5.50/share. Doliner alleged that there was an additional oral 
    agreement that the option ``was only to be used in the event that he 
    solicited an outside offer.''\footnote{Casebook p. 600.}
    \item Negotiations fell through. Hunt exercised its option. Doliner 
    refused to deliver the stock, so Hunt sued, requesting summary judgment.
    \item Hunt argued that the alleged condition could not be proved under the 
    parol evidence rule.
    \item UCC \S\ 2-202 provides that parol evidence may explain or supplement 
    a written agreement agreement, but not contradict it. According to the 
    court, a term is inconsistent if it contradicts or negates a term of the 
    writing. If it does not, it is admissible.
    \item In this case, the alleged modification---that Hunt could exercise 
    the option only if Doliner got another offer---was consistent with the 
    written agreement and therefore admissible.
    \item
\end{enumerate}

\paragraph{\emph{Alaska Northern Development, Inc. v. Alyeska Pipeline Service 
Co.}}

\begin{enumerate}
    \item The \emph{Hunt Foods} view of consistency is too narrow. A better 
    definition is ``the absence of reasonable harmony in terms of the language 
    and respective obligations of the parties.''\footnote{Casebook p. 601.}
\end{enumerate}

\paragraph{Merger Clauses}

\begin{enumerate}
    \item Merger (or integration) clauses provide that the written contract is 
    the entire agreement between the parties.\footnote{Casebook p. 602.}
    \item Often, a merger clause is not enough to prove complete integration. 
    Courts often hold that the parties must have actually assented to 
    integration.
\end{enumerate}

\paragraph{\emph{ARB (American Research Bureau), Inc. v. E-Systems, Inc.}}

\begin{enumerate}
    \item Courts must consider surroudning circumstances to determine whether 
    the parties actually assented to the merger clause.
\end{enumerate}

\paragraph{\emph{Siebel v. Layne \& Bowler, Inc.}}

\begin{enumerate}
    \item UCC \S\ 2-202 requires intent to integrate in addition to a merger 
    clause. In a written contract, the merger clause must be conspicuous.
\end{enumerate}

\paragraph{The Fraud Exception to the Parol Evidence Rule}

\begin{enumerate}
    \item Otherwise inadmissible parol evidence is admissible if it shows an 
    ``invalidating cause'' of the written agreement---e.g., lack of 
    consideration, duress, mistake, illegality, or fraud.\footnote{Casebook p. 
    604.}
    \item % TODO expand pp. 604-06
\end{enumerate}

\paragraph{The Condition-to-Legal Effectiveness Exception to the Parol 
Evidence Rule}

\begin{enumerate}
    \item The parol evidence rule does not apply when the occurrence or 
    nonoccurence of an event, by spoken agreement, is a condition to making 
    the written agreement binding or effective.\footnote{Casebook p. 606.}
    \item For instance, A and B make a written business agreement, and agree 
    orally that the agreement will be null if the parties fail to raise 
    \$600,000 within 20 days. Evidence of the oral agreement is admissible.
    \item Two criticisms:
    \begin{enumerate}
        \item If the parol agreement is a condition to \emph{performance} of a 
        written agreement, the parol agreement is inadmissible because the 
        written agreement is a binding contract. But if the parol agreement is 
        a condition to the \emph{legal effectiveness} of the written 
        agreement, the parol agreement is admissible.
        \item This exception is hard to reconcile with the parol evidence 
        rule. The rational for the condition-to-legal effectiveness rule is 
        that if a parol agreement indicated that a written agreement would not 
        be effective until an additional condition was met, then the written 
        instrument is not a contract, so the parol evidence rule does not 
        apply. The difficulty is that in most cases where the exception 
        applies, the written agreement \emph{is} a contract, and the parol 
        agreement was a condition to the obligation of performance, rather 
        than a condition to the contract's legal effectiveness. In the above 
        example, for instance, A would likely be in breach of contract for 
        failing to raise the \$600,000.
    \end{enumerate}
\end{enumerate}

\paragraph{UCC \S\ 2-209(2), (4), (5)}

\begin{enumerate}
    \item % TODO supp
\end{enumerate}

\paragraph{Principles of European Contract Law Art. 2.106}

\begin{enumerate}
    \item % TODO supp
\end{enumerate}

\paragraph{No Oral Modification Clauses}

\begin{enumerate}
    \item The parol evidence rule applies only to oral agreements made before 
    or contemporaneously with a written, integrated contract. The rule doesn't 
    apply to a later agreement that modifies the 
    integration.\footnote{Casebook p. 208.}
    \item However, the later oral agreement may be invalid under the legal 
    duty rule or the Statute of Frauds.
    \item Written contracts often provide that they can be modified only in 
    writing---a private Statute of Frauds. These are \textbf{no oral 
    modification (n.o.m.)} clauses.
    \item Common law: oral modifications are enforceable notwithstanding 
    n.o.m. clauses, because the later oral agreement by implication modifies 
    the earlier written agreement containing the n.o.m. clause.
    \item Example: ``I'll pay ypu \$100 to lock me in this room. Don't let me 
    out for 10 hours, no matter what I say.'' You sign the agreement. At 
    common law, cries of ``I changed my mind, I'll pay you \$500 to let me 
    out!'' would modify the original agreement.
    \item UCC \S\ 2-209 makes two key changes to the common law:
    \begin{enumerate}
        \item \S\ 2-209(1): a modification needs no consideration to be 
        enforceable.
        \item \S\ 2-209(2): if a contract for the sale of goods contains a 
        n.o.m. clause, the modification must be in writing.
    \end{enumerate}
    \item But, \S\ 2-209(4) allows that an attempt at modification under \S\ 
    2-209(2) can operate as a waiver. But then \S\ 2-209(5) provides that a 
    party who has created a waiver for an executory (unperformed) part of the 
    contract can retract it if the other party has relied on the waiver.
\end{enumerate}

\subsubsection{The Interpretation of Written Contracts}

\paragraph{Plain Meaning: \emph{Steuart v. McChesney}}

\begin{enumerate}
    \item The Steuarts granted a right of first refusal on a piece of farmland 
    to the McChesneys. It provided that if the Steuarts got a bone fide offer 
    for the land, the McChesneys could buy it for the market value as assessed 
    for the county and state real estate tax levy.
    \item Later, the Steuarts got offers for \$30,000 and \$35,000. A 
    commercial appraiser valued it at \$50,000. But the government appraisal 
    valued it at \$7,820.
    \item The McChesneys sued for specific performance.
    \item ``~.~.~.~where language is clear and unambiguous, the focus of 
    interpretation is upon the terms of the agreement as \emph{manifestly} 
    expressed, rather than as, perhaps, silently intended.''\footnote{Casebook 
    p. 608.}
    \item Despite controversy over the plain meaning rule, the court here 
    decided to follow it.
    \item The trial court held that the government value was a ``protective 
    minimum'' rather than a ``controlling price.''\footnote{Casebook p. 609.}
    \item The appellate court reversed, following the agreement's plain 
    meaning. The court here affirmed.
    \item The government's assessment was outdated. It's not fair to make the 
    parties adhere to it.
\end{enumerate}

\paragraph{\emph{Mellon Bank, N.A. v. Aetna Business Credit, Inc.}}

\begin{enumerate}
    \item The judge's linguistic reference point is necessarily different 
    than the parties'. Parties should be allowed to introduce reasonable 
    alternative interpretations of the written agreement.
\end{enumerate}

\paragraph{\emph{Amoco Production Co. v. Western Slope Gas Co.}}

\begin{enumerate}
    \item UCC \S\ 2-202 assumes that the written contract is not completely 
    integrated. Under the UCC, the court should admit parole evidence by 
    default unless the judge thinks the written agreement is unambiguous.
\end{enumerate}

\paragraph{\emph{Pacific Gas \& Electric Co. v. G.W. Thomas Drayage \& Rigging 
Co.}}

\begin{enumerate}
    \item PG\&E contracted with Thomas Drayage to repair a steam turbine. The 
    contract included an indemnity clause providing that Drayage would 
    indemnify PG\&E ``against all loss, damage, expense and liability 
    resulting from~.~.~.~injury to property~.~.~.~''\footnote{Casebook p. 
    615.}
    \item The turbine was damaged during the work. PG\&E sued to recover the 
    \$25,144.51 it spent on repairs. The trial court held for PG\&E ``on the 
    theory that the indemnity provision covered injury to all property 
    regardless of ownership.''\footnote{Casebook p. 615.}
    \item Drayage wanted to introduce parol evidence showing that the parties 
    meant for the indemnity clause to apply only to damage to property of 
    third parties, not to PG\&E's property. But the trial court held that the 
    contract had a plain meaning, so it excluded the parol evidence.
    \item But language lacks fixed meaning. ``Accordingly, rational 
    interpretation requires at least a preliminary consideration of all 
    credible evidence offered to prove the intention of the 
    parties.''\footnote{Casebook p. 617.}
    % TODO: does this case abolish the parol evidence rule?
    % -- does the UCC abolish the parol evidence rule?
\end{enumerate}

\paragraph{\emph{Garden State Plaza Corp. v. S.S. Kresge Co.}}

\begin{enumerate}
    \item The parol evidence rule does not come into play until the court 
    determines the meaning of the written agreement. In interpreting and 
    constructing the written agreement, courts must allow all relevant 
    evidence indicating its meaning, including evidence from the circumstances 
    of the creation of the agreement.
    \item But, external evidence is invalid if it gives the agreement ``a 
    meaning completely alien to anything its words can possibly 
    express.''\footnote{Casebook p. 618.}
    % TODO: is this a hedging in of the PG&E rule? or merely a restatement of 
    % it?
\end{enumerate}

\paragraph{\emph{Trident Center v. Connecticut General Life Ins. Co.}}

\begin{enumerate}
    \item Trident obtained financing for a construction project from 
    Connecticut General. The contract provided for a \$56.5 million loan at 
    12.25\% interest for 15 years, without the ability to prepay the principal 
    for the first 12 years. If it defaulted before year 12, Trident could 
    accelerate payment and add a 10\% prepayment fee.
    \item Five years later, interest rates dropped. Trident sued for a 
    declaration saying that it could prepay the loan, subject only to the 10\% 
    prepayment fee. Connecticut General argued that the contract unambiguously 
    precluded prepayment before 12 years. The district court granted Trident's 
    motion to dismiss.
    \item The Ninth Circuit reversed. Judge Kozinksi:
    \begin{enumerate}
        \item The language is unambiguous, but nonetheless Trident interpreted 
        it as granting the right to prepay.
        \item Trident was seeking ``judicial sterilization of its intended 
        default.''\footnote{Casebook p. 620.}
        \item In the alternative, Trident sought to introduce external 
        evidence showing that the actual agreement was quite different.
        \item The court grudgingly agreed that the external evidence must be 
        allowed under the California Supreme Court's holding in \emph{Pacific 
        Gas}---which ``we have no difficulty understanding~.~.~.~, even 
        without extrinsic evidence to guide us.''\footnote{Casebook p. 621.}
    \end{enumerate}
    % TODO: what's the problem with following the pg&e rule? if the language 
    % is in fact unambiguous, the case will easily fall at trial. the only 
    % cost is judicial inefficiency, which is costly in terms of time and 
    % money, but not analytically.
\end{enumerate}
