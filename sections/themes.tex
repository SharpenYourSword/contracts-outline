\section{Themes}

\begin{enumerate}
    \item What things make a contract enforceable? How do these things change 
    over time? How do legal definitions encourage or discourage these 
    developments?
    \item How do we determine who is competent to enter into a contract?
    \item When do moral transgressions justify government intervention? Where 
    do moral obligations fail to align with contract law?
    \item What makes a contract binding?
    \begin{enumerate}
        \item Consideration.
        \item Reliance.
    \end{enumerate}
    \item Spectrum of indicators:
    \begin{enumerate}
        \item \textbf{Objective}: seals, forms, etc.
        \item \textbf{Subjective}: intent.
    \end{enumerate}
    \item Should a good judge play with the rules or follow them to the 
    letter? See, e.g., \emph{Ricketts}.
    \item Berring's basic rules:
    \begin{enumerate}
        \item There are no villains in contract law. (Holmes.)
        \item Look at the cold facts. What specifically happened? (Holmes 
        again.)
        \item That which is not clear cannot be made clear.
    \end{enumerate}
    \item To what extent should courts evaluate the adequacy of consideration?
    \item When are contracts enforceable? What goes into a contract to make it 
    enforceable?
    \item When do courts evaluate the adequacy of consideration? See 
    \emph{Schnell v. Nell}, and cf. \emph{Batsakis}.
\end{enumerate}
