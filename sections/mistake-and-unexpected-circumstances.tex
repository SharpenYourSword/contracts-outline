\section{Mistake and Unexpected Circumstances}

\subsection{Mistake}

\subsubsection{Unilateral Mistakes (Mechanical Errors)}

\paragraph{The Nolan Ryan Baseball Card Case}

\begin{enumerate}
    \item Bryan Wrzesinski bought a baseball card worth \$1,200 for \$12. The 
    cashier apparently didn't know its worth. The owner of the store sued to 
    recover either the card or damages.\footnote{Casebook p. 727.}
    \item The parties settled, auctioning of the card and giving the proceeds 
    to charity.
    \item The facts are ambiguous, but under any scenario, Bryan either misled 
    the cashier or knew that the price tag was the result of a mechanical 
    error.
\end{enumerate}

\paragraph{Restatement Second \S\S\ 154, 154} % TODO 153?

\begin{enumerate}
    \item % TODO supp
\end{enumerate}

\paragraph{Unilateral Mistake}

\begin{enumerate}
    \item ``~.~.~.~relief will not be granted unless the other party has 
    either not relied \emph{or} cannot be restored to his precontractual 
    position by the award of reliance damages.''\footnote{Casebook p. 728.}
\end{enumerate}

\subsubsection{Mistakes in Transcription; Reformation}

\paragraph{\emph{Travelers Ins. Co. v. Bailey}}

\begin{enumerate}
    \item Bailey bought life insurance. The plan he applied for was for 
    \$5,000 plus a retirement annuity for \$500 per \emph{year}.
    \item His policy documents said that the annuity was \$500 per 
    \emph{month}.
    \item Bailey brought the issue to Travelers' attention, and it issued a 
    new policy.
    \item ``Where, as here, an antecedent contract has been established by the 
    requisite measure of proof, equity will act to bring the erroneous writing 
    into conformity with the true agreement.''\footnote{Casebook pp. 729--30.} 
    Unilateral mistakes call for reformation.
    \item ``~.~.~.~we hold that where there has been established beyond a 
    reasonable doubt a specific contractual agreement between parties, and a 
    aubsequent erroneous rendition of the terms of the agreement in a material 
    particular, the party penalized by the error is entitled to reformation, 
    if there has been no prejudicial change of position by the other party 
    while ignorant of the mistake.''\footnote{Casebook pp. 730--31.}
\end{enumerate}

\paragraph{Restatement Second \S\ 155}

\begin{enumerate}
    \item % TODO supp
\end{enumerate}

\paragraph{\emph{Chimart Associates v. Paul}}

\begin{enumerate}
    \item ``~.~.~.~reformation based upon mistake is not available where the 
    parties purposely contract based upon uncertain or contingent 
    events.''\footnote{Casebook p. 731.}
    \item The party claiming mistake or fraud must prove it with certainty, 
    and must also prove what the parties actually agreed to.
\end{enumerate}

\subsubsection{Mutual Mistakes (Shared Mistaken Assumptions)}

\begin{enumerate}
    \item \textbf{Shared mistaken assumptions}: tacit or explicit assumptions 
    in the agreement turn out to be incorrect.
    \item Tacit assumptions are always present (e.g., the sun will rise 
    tomorrow). No agreement could possibly render all of them explicit.
\end{enumerate}

\paragraph{Barren Cow: \emph{Sherwood v. Walker}}

\begin{enumerate}
    \item ``Replevin for a cow.''\footnote{Casebook p. 733.}
    \item Sherwood contracted to buy a cow from Walker. Everyone thought the 
    cow was probably barren, so he bought it for only \$80.
    \item After agreeing to sell it, the defendants realized the cow was 
    pregnant, so they refused to give it to Sherwood. Sherwood sued for 
    performance.
    \item The trial court held for Sherwood. The appellate court (``circuit 
    court of Wayne county'') affirmed.\footnote{Casebook p. 733.}
    \item The circuit judge instructed the jury that if it found that the 
    defendants intended to transfer title to the cow when they confirmed 
    Sherwood's order, the agreement was valid. It didn't matter when or where 
    the cow was held.\footnote{Casebook p. 735.}
    \item The defendants argued that title did not pass until the cow was 
    weighed and the price determined (because the price was based on weight), 
    and that ``the barrenness of the cow was a condition precedent to pasing 
    title~.~.~.''\footnote{Casebook p. 736.}
    \item Here, the court held that a party can refuse to perform a contract 
    ``if the assent was founded, or the contract made, upon the mistake of a 
    material fact~.~.~.''\footnote{Casebook p. 736.}
    \item ``The difficulty in every case is to determine whether the mistake 
    or misapprehension is as to the \textbf{substance} of the whole contract, 
    going, as it were, to the root of the matter, or only to \textbf{some 
    point}, even though a material point, an error as to which does not affect 
    the substance of the whole consideration.''\footnote{Casebook p. 736.}
    \item The mistake as to the cow's barrenness ``went to the \textbf{whole 
    substance} of the agreement.''\footnote{Casebook p. 736.} ``She was sold 
    as a beef creature would be sold; she is in fact a breeding cow, and a 
    valuable one.''\footnote{Casebook p. 737.}
    \item Held: if the jury determines that the cow was fertile, the court 
    should allow the defendants to rescind the contract.
    \item Judge Sherwood, dissenting:
    \begin{enumerate}
        \item The plaintiff thought the cow would breed. He entered the 
        agreement on the basis of his judgment. 
        \item ``~.~.~.~it is held that because it turned out that the 
        plaintiff was more correct in his judgment as to one quality of the 
        cow than the defendants, and a quality, too, which could not by any 
        possibility be positively known at the time by either party to exist, 
        the contract may be annulled by the defendants at their 
        pleasure.''\footnote{Casebook p. 738.}
        \item ``When a mistaken fact is relied upon as ground for rescinding, 
        such fact must not only exist at the time the contract is made, but 
        must have been known to one or both of the 
        parties.''\footnote{Casebook p. 738.}
        \item Here, neither party knew whether the cow was barren. The 
        defendants shouldn't be able to rescind because they misjudged the 
        cow's fertility.
    \end{enumerate}
\end{enumerate}

\paragraph{\emph{Nester v. Michigan Land \& Iron Co.}}

\begin{enumerate}
    \item Michigan Land sold ``all the merchantable pine'' on a lot to Nester. 
    The agreement required Nester to estimate the amound of timber on the 
    land.
    \item After purchasing the rights, Nester discovered that the quality and 
    amount of timber were far below what he expected. He sued to reduce the 
    purchase price by 50\%.
    \item The court held for Michigan Land. Judge Sherwood, who dissented a 
    year earlier in \emph{Sherwood}, wrote the opinion, holding that the 
    \emph{Sherwood} rule applies only when ``all the facts and circumstances 
    are precisely the same as in that.''\footnote{Casebook p. 739.}
\end{enumerate}

\paragraph{\emph{Griffith v. Brymer}}

\begin{enumerate}
    \item Griffith paid Brymer \textsterling 100 to rent a room to watch the 
    Coronation Procession.
    \item That morning, the King decided to undergo surgery and cancelled the 
    procession, unbeknownst to Griffith and Brymer.
    \item Griffith sued to recover the money.
    \item The court held that the agreement was based on a supposition that 
    made performance impossible, ``which went to the whole root of the 
    matter.''\footnote{Casebook p. 740.} Griffith could recover his 
    \textsterling 100.
\end{enumerate}

\paragraph{\emph{Wood v. Boynton}}

\begin{enumerate}
    \item Wood brought a small stone to Boynton, a jeweler. He offered her \$1 
    for it. She declined. Later, she needed money, so she returned to Boynton 
    and accepted his offer. It turned out to be an uncut diamond, though 
    neither of them knew it. Wood sued for rescission.
    \item Held for Boynton: ``However unfortunate the plaintiff may have been 
    in selling this valuable stone for a mere nominal sum, she has failed 
    entirely to make out a case either of fraud or mistake in the sale such as 
    will entitle her to a rescission~.~.~.''\footnote{Casebook p. 741.}
    \item Was there a mistake in this case? The court says no.
\end{enumerate}

\paragraph{\emph{Firestone \& Parson, Inc. v. Union League of Philadelphia}}

\begin{enumerate}
    \item Union League owned a painting called ``The Bombardment of Fort 
    Sumter,'' generally regarded as a major work of Albert Bierstadt.
    \item Firestone bought the painting from Union League in 1981 for 
    \$500,000.
    \item By 1986, art critics agreed that the painting was the work of 
    another artist, John Ross Key.
    \item In 1988 Firestone sued for rescission, arguing that as a Bierstadt 
    the painting was worth more than \$500,000, but as a Key it was worth only 
    \$50,000.
    \item The court held that there had not been a mistake. ``Post-sale 
    fluctuations in generally accepted attributions do not necessarily 
    establish that there was a mutural mistake of fact at the time of the 
    sale.''\footnote{Casebook p. 743.} Since art critics generally agreed in 
    1981 that the painting was a Bierstadt, then there was no mutual mistake 
    of fact.
\end{enumerate}

\paragraph{\emph{Everett v. Estate of Sumstad}}

\begin{enumerate}
    \item The Mitchells bought a \$50 safe at an auction. The inner 
    compartment was locked and the key had been lost, so they hired a 
    locksmith to open it. Inside was \$32,207.
    \item ``~.~.~.~we hold reasonable persons would conclude that the 
    auctioneer manifested an objective intent to sell the safe and its 
    contents and that the parties mutually assented to enter into that sale of 
    the safe and the contents of the locked compartment.''\footnote{Casebook 
    p. 743.}
\end{enumerate}

\paragraph{Restatement Second \S\S\ 151, 152, 154}

\begin{enumerate}
    \item % TODO supp
\end{enumerate}

\paragraph{``Basic Assumption'': \emph{Lenawee County Board of Health v. 
Messerly}}

\begin{enumerate}
    \item In 1971, Bloom conveyed one acre plus six hundred square feet to the 
    Messerlys. There was a three-unit apartment building on the 600 square 
    foot portion. Bloom had improperly installed a septic tank.
    \item In 1973, the Messerlys sold it on a land contract to Barnes. The 
    Barnses defaulted, so in 1977, the Messerlys executed a land contract with 
    the Pickleses. It included as-is and integration 
    clauses.\footnote{Casebook p. 744.}
    \item Fix or six days later, the Pickleses noticed raw sewage seeping out 
    of the ground. The Lewanee County Board of Health condemned the property 
    and brought suit against the Messerlys and Pickleses to win a permanent 
    injunction prohibiting human habitation until the property conformed with 
    the sanitation code.
    \item The Pickleses made no payment on the contract. The Messerlys 
    cross-claimed against the Pickleses, seeking foreclosure, sale, and a 
    deficiency judgment. The Pickleses counterclaimed for rescission and filed 
    a third-party complaint against the Barneses. The Pickleses alleged 
    failure of consideration and misrepresentation.
    \item The trial court held that the Pickleses had no cause of action 
    against either the Barnses or the Messerlys because there was no fraud or 
    misrepresentation. Nobody knew of Bloom's ``transgression'' until the 
    Pickleses discovered the sewage leak. The as-is clause protected the 
    seller. The court granted foreclosure and damages in the amount of the 
    land contract (\$25,943.09).
    \item The appellate court affirmed as to the Barnses but reversed as to 
    the Messerlys, holding that ``the mutual mistake~.~.~.~went to a basic, as 
    opposed to a collateral, element of the contract, and that the parties 
    intended to transfer income-producing rental property but, in actuality, 
    the vendees paid \$25,500 for an asset without value.''\footnote{Casebook 
    p. 745.}
    \item A contractual mistake ``must relate to a fact in existence at the 
    time the contract is executed.''\footnote{Casebook p. 745.}
    \item The septic system was defective before the Pickleses executed their 
    agreement. Therefore, there was a mutual mistake.
    \item The Barnses and Messerlys argued that the mistake was collateral to 
    the agreement. The Pickleses argued that it was ``pervasive and 
    essential,'' citing \emph{Sherwood}.\footnote{Casebook p. 746.}
    \item The court here rejected the \emph{Sherwood} rule, limiting it to the 
    facts in that case. It replaced it with the \textbf{basic assumption 
    rule}: ``~.~.~.~rescission is indicated when the mistaken belief relates 
    to a basic assumption of the parties upon which the contract is made, and 
    which materially affects the agreed performance of the 
    party.''\footnote{Casebook p. 747.}
    \item However, even in such cases, courts need not grant rescission.  
    ``Equity suggests that, in this case, the risk should be allocated to the 
    purchases.''\footnote{Casebook p. 748.} The as-is clause allocated the 
    risk to the Pickleses. The Pickleses were not entitled to rescission.  
    Reversed.
\end{enumerate}

\paragraph{\emph{Beachcomber Coins, Inc. v. Boskett}}

\begin{enumerate}
    \item Beachcomber bought a coin for \$500 from Boskett. It was purportedly 
    a dime minted in Denver in 1916. Beachcomber later learned that it was a 
    counterfeit. He sued for rescission based on mutual mistake.
    \item The trial judge held for Boskett on the ground that a dealer has an 
    obligation to determine the genuineness of the coin, and he assumes the 
    risk when he makes the purchase.
    \item The appellate court here reversed. ``[N]egligent failure of a party 
    to know or to discover the facts as to which both parties are under a 
    mistake does not preclude rescission or reformation on account 
    thereof.''\footnote{Casebook p. 749.}
    \item The buyer's assumption of risk is relevant only when the parties are 
    aware of the possibility that they were wrong. Here, both parties were 
    certain that the coin was genuine. It would've been different if the buyer 
    was uncertain about the coin's genuineness and accepted the buyer's expert 
    judgment.
\end{enumerate}

\subsection{The Effect of Unexpected Circumstances}

\begin{enumerate}
    \item Impossibility, impracticability, frustration.
\end{enumerate}

\subsubsection{Implied Condition: \emph{Taylor v. Caldwell}}

\begin{enumerate}
   \item The plaintiffs contracted with the defendants to rent The Surrey 
   Gardens and Music Hall for four nights. The hall burned down before the 
   specified nights. The plaintiff sued to recover payment and other losses.
    \item The existence of the hall was an \textbf{implied 
    condition}.\footnote{Casebook p. 766.}
    \item ``~.~.~.~in contracts in which the performance depends on the 
    continued existence of a given person or thing, a condition is implied 
    that the possibility of performance arising from the perishing or the 
    person or thing shall excuse the performance.''\footnote{Casebook p.  
    768.}
\end{enumerate}

\subsubsection{Tacit Assumptions---Continued}

\begin{enumerate}
    \item Contracts involve many tacit assumptions.
    \item \enquote{We `just know' that the burning of a music hall violates a 
    tacit assumption of the parties who executed a contract for hiring it for 
    a few days; we `just know' that a two per cent increase in the price of 
    beans does not violate a tacit assumption underlying a contract to deliver 
    a ton of beans for a fixed price.}\footnote{Casebook p. 769.}
\end{enumerate}

\subsubsection{\emph{Ocean Tramp Tankers Corp. v. V/O Sovfracht}}

\begin{enumerate}
    \item The theory of an implied term suggests that if an implied term is 
    violated, the deal is off. \emph{Taylor v. Caldwell}. But the parties 
    likely wouldn't want to just end the deal. They would probably want to 
    modify it instead.
\end{enumerate}

\subsubsection{Impracticality: \emph{Mineral Park Land Co. v. Howard}}

\begin{enumerate}
    \item The defendants were building a bridge across a ravine, which the 
    plaintiffs owned. They agreed that the defendants would take all necessary 
    gravel and earth for the project from the plaintiffs' land.
    \item The defendants took 50,869 cubic yards from the plaintiffs' land, 
    but used 101,000 total in the project.
    \item Plaintiff sued for breach. The court held for the defendants 
    because, although the plaintiffs' land contained more than enough earth 
    and gravel, only 50,131 cubic yards were above water and accessible. The 
    rest of the gravel and earth on the plaintiffs' land was impractical (and 
    thus ``impossible in legal contemplation'') to retrieve.\footnote{Casebook 
    p. 770.}
\end{enumerate}

\subsubsection{Bad Gamble: \emph{United States v. Wegematic Corp.}}

\begin{enumerate}
    \item Wegematic secured a contract to sell a computer to the Federal 
    Reserve Board. It had difficulty finishing the order on time, so it 
    ultimately requested cancellation.
    \item The Board bought and IBM machine instead and sued Wegematic for 
    damages.
    \item Wegematic argued that delivery was impossible because of ``basic 
    engineering difficulties,'' which it argued constituted a ``practical 
    impossibility.''\footnote{Casebook p. 772.}
    \item UCC \S\ 2-615 governs practical impossibility.
    \item The court held that Wegematic gambled that it could develop the 
    technology and ship the order within the agreed time. It lost. If 
    Wegematic were allowed to raise impossibility as a defense, technology 
    manufacturers ``would thus enjoy a wide degree of latitude with respect to 
    performance while holding an option to compel the buyer to pay if the 
    gamble should pan out.''\footnote{Casebook p. 772.}
\end{enumerate}

\subsubsection{\emph{Missouri Public Service Co. v. Peabody Coal Co.}}

\begin{enumerate}
    \item Peabody agreed to supply coal to Public Service at \$5.40 per ton. 
    The agreement included an escalation clause that would increase the price 
    per ton according to the Industrial Commodities Index.
    \item The price of coal rose. Peabody tried to negotiate a new price, but 
    Public Service would only add \$1.00 per ton. Peabody notified Public 
    Service of its intent to breach, so Public Service sued.
    \item Among other defenses, Peabody argued that UCC \S\ 2-615 offered a 
    defense of commercial impracticability on the basis of excuse by failure 
    of presupposed conditions. Specifically, Peabody argued (1) that the 
    Industrial Commodities Index was no longer an accurate measure of 
    inflation, and (2) the oil embargo had driven up prices.
    \item The court here held that Peabody was aware of the behavior of the 
    Industrial Commodities Index, so it foresaw or should have foreseen the 
    Index's divergence from other measures. The court also held that Peabody 
    failed to show that the embargo caused financial hardship.
    \item Peabody made a bad bargain, but \enquote{this fact alone does not 
    deal with either the `basic assumption' on which the contract was 
    negotiated or alter the `essential nature of the performance' thereunder 
    so as to constitute `commercial impracticability.'}\footnote{Casebook p. 
    792.}
\end{enumerate}

\subsection{Problems of Performance}

\subsubsection{The Obligation to Perform in Good Faith}

\paragraph{Duty to Avoid Interfering with Performance: \emph{Patterson v. 
Meyerhofer}}

\begin{enumerate}
    \item Meyerhofer agreed to buy four properties from Patterson, which he 
    planned to buy at an auction. Meyerhofer ended up bidding for the same 
    properties in the same auction, winning them for \$620 less than she would 
    have paid Patterson.
    \item The court held for Patterson. ``In the case of every contract there 
    is an implied undertaking on the part of each party that he will not 
    intentionally and purposely do anything to prevent the other party from 
    carrying out the agreement on his part.''\footnote{Casebook p. 887.}
\end{enumerate}

\paragraph{Indirect Interference: \emph{Iron Trade Products Co. v. Wilkoff 
Co.}}

\begin{enumerate}
    \item The plaintiff contracted with the defendant to buy 2,600 tons of 
    relaying rails. Afterward, the plaintiff bought 887 tons of similar rails 
    on the open market. Rails were scarce, so the plaintiff's second purchase 
    caused the defendant to be unable to find rails to fill its obligation 
    except at exorbitant prices, rendering performance impossible.
    \item The court held for the plaintiff, finding that ``[m]ere difficulty 
    of performance will not excuse a breach of contract.''\footnote{Casebook 
    p. 889.}
\end{enumerate}

\paragraph{\emph{Kirke La Shelle Co. v. Paul Armstrong Co.}}

\begin{enumerate}
    \item ``~.~.~.~in every contract there exists an implied covenant of good 
    faith and fair dealing.''\footnote{Casebook p. 890.}
\end{enumerate}

\paragraph{The UCC Definitions of Good Faith}

\begin{enumerate}
    \item Good faith in the UCC means ``honesty in fact and the observance of 
    reasonable commercial standards of fair dealing.''\footnote{Casebook p. 
    891.}
\end{enumerate}

\paragraph{Duty to Perform in Good Faith under the UCC and Restatement 
Second}

\begin{enumerate}
    \item Both the UCC and the Restatement (Second) require good faith 
    performance.
\end{enumerate}

\paragraph{Farnsworth, ``Good Faith in Contract Performance''}

\begin{enumerate}
    \item What does ``good faith'' mean?\footnote{Casebook pp. 892--93.}
    \begin{enumerate}
        \item Farnsworth: it's a fundamental idea of contract law. It implies 
        terms in the contract.
        \item Summers: the ``excluder'' analysis asks, what does a judge want 
        to rule out by use of the phrase ``good faith''? It has no meaning on 
        its own, but it serves to exclude many heterogeneous forms of bad 
        faith.
        \item Burton: the ``forgone opportunity analysis'' argues that ``good 
        faith'' limits what a party can do in performance, so bad faith is to 
        ``recapture opportunities forgone upon 
        contracting~.~.~.~''
        \item Courts apply all three standards.
    \end{enumerate}
\end{enumerate}
