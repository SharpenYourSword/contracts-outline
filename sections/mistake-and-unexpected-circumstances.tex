\section{Mistake and Unexpected Circumstances}

\subsection{Mistake}

\subsubsection{Unilateral Mistakes (Mechanical Errors)}

\paragraph{The Nolan Ryan Baseball Card Case}

\begin{enumerate}
    \item Bryan Wrzesinski bought a baseball card worth \$1,200 for \$12. The 
    cashier apparently didn't know its worth. The owner of the store sued to 
    recover either the card or damages.\footnote{Casebook p. 727.}
    \item The parties settled, auctioning of the card and giving the proceeds 
    to charity.
    \item The facts are ambiguous, but under any scenario, Bryan either misled 
    the cashier or knew that the price tag was the result of a mechanical 
    error.
\end{enumerate}

\paragraph{Restatement Second \S\S\ 154, 154}

\begin{enumerate}
    \item % TODO supp
\end{enumerate}

\paragraph{Unilateral Mistake}

\begin{enumerate}
    \item ``~.~.~.~relief will not be granted unless the other party has 
    either not relied \emph{or} cannot be restored to his precontractual 
    position by the award of reliance damages.''\footnote{Casebook p. 728.}
\end{enumerate}

\subsubsection{Mistakes in Transcription; Reformation}

\paragraph{\emph{Travelers Ins. Co. v. Bailey}}

\begin{enumerate}
    \item Bailey bought life insurance. The plan he applied for was for 
    \$5,000 plus a retirement annuity for \$500 per \emph{year}.
    \item His policy documents said that the annuity was \$500 per 
    \emph{month}.
    \item Bailey brought the issue to Travelers' attention, and it issued a 
    new policy.
    \item ``Where, as here, an antecedent contract has been established by the 
    requisite measure of proof, equity will act to bring the erroneous writing 
    into conformity with the true agreement.''\footnote{Casebook pp. 729--30.} 
    Unilateral mistakes support reformation.
    \item ``~.~.~.~we hold that where there has been established beyond a 
    reasonable doubt a specific contractual agreement between parties, and a 
    aubsequent erroneous rendition of the terms of the agreement in a material 
    particular, the party penalized by the error is entitled to reformation, 
    if there has been no prejudicial change of position by the other party 
    while ignorant of the mistake.''\footnote{Casebook pp. 730--31.}
\end{enumerate}

\paragraph{Restatement Second \S\ 155}

\begin{enumerate}
    \item % TODO supp
\end{enumerate}

\paragraph{\emph{Chimart Associates v. Paul}}

\begin{enumerate}
    \item ``~.~.~.~reformation based upon mistake is not available where the 
    parties purposely contract based upon uncertain or contingent 
    events.''\footnote{Casebook p. 731.}
    \item The party claiming mistake or fraud must prove it with certainty, 
    and must also prove what the parties actually agreed to.
\end{enumerate}

\subsubsection{Mutual Mistakes (Shared Mistaken Assumptions)}

\begin{enumerate}
    \item \textbf{Shared mistaken assumptions}: tacit or explicit assumptions 
    in the agreement turn out to be incorrect.
    \item Tacit assumptions are always present (e.g., the sun will rise 
    tomorrow). No agreement could possibly render all of them explicit.
\end{enumerate}

\paragraph{Barren Cow: \emph{{Sherwood v. Walker}}

\begin{enumerate}
    \item ``Replevin for a cow.''\footnote{Casebook p. 733.}
    \item Sherwood contracted to buy a cow from Walker. Everyone thought the 
    cow was probably barren, so he bought it for only \$80.
    \item After agreeing to sell it, the defendants realized the cow was 
    pregnant, so they refused to give it to Sherwood. Sherwood sued for 
    performance.
    \item The trial court held for Sherwood. The appellate court (``circuit 
    court of Wayne county'') affirmed.\footnote{Casebook p. 733.}
    \item The circuit judge instructed the jury that if it found that the 
    defendants intended to transfer title to the cow when they confirmed 
    Sherwood's order, the agreement was valid. It didn't matter when or where 
    the cow was held.\footnote{Casebook p. 735.}
    \item The defendants argued that title did not pass until the cow was 
    weighed and the price determined (because the price was based on weight), 
    and that ``the barrenness of the cow was a condition precedent to pasing 
    title~.~.~.''\footnote{Casebook p. 736.}
    \item Here, the court held that a party can refuse to perform a contract 
    ``if the assent was founded, or the contract made, upon the mistake of a 
    material fact~.~.~.''\footnote{Casebook p. 736.}
    \item ``The difficulty in every case is to determine whether the mistake 
    or misapprehension is as to the \textbf{substance} of the whole contract, 
    going, as it were, to the root of the matter, or only to \textbf{some 
    point}, even though a material point, an error as to which does not affect 
    the substance of the whole consideration.''\footnote{Casebook p. 736.}
    \item The mistake as to the cow's barrenness ``went to the \textbf{whole 
    substance} of the agreement.''\footnote{Casebook p. 736.} ``She was sold 
    as a beef creature would be sold; she is in fact a breeding cow, and a 
    valuable one.''\footnote{Casebook p. 737.}
    \item Held: if the jury determines that the cow was fertile, the court 
    should allow the defendants to rescind the contract.
    \item Judge Sherwood, dissenting:
    \begin{enumerate}
        \item The plaintiff thought the cow would breed. He entered the 
        agreement on the basis of his judgment. 
        \item ``~.~.~.~it is held that because it turned out that the 
        plaintiff was more correct in his judgment as to one quality of the 
        cow than the defendants, and a quality, too, which could not by any 
        possibility be positively known at the time by either party to exist, 
        the contract may be annulled by the defendants at their 
        pleasure.''\footnote{Casebook p. 738.}
        \item ``When a mistaken fact is relied upon as ground for rescinding, 
        such fact must not only exist at the time the contract is made, but 
        must have been known to one or both of the 
        parties.''\footnote{Casebook p. 738.}
        \item Here, neither party knew whether the cow was barren. The 
        defendants shouldn't be able to rescind because they misjudged the 
        cow's fertility.
    \end{enumerate}
\end{enumerate}

\paragraph{\emph{Nester v. Michigan Land \& Iron Co.}}

\begin{enumerate}
    \item Michigan Land sold ``all the merchantable pine'' on a lot to Nester. 
    The agreement required Nester to estimate the amound of timber on the 
    land.
    \item After purchasing the rights, Nester discovered that the quality and 
    amount of timber were far below what he expected. He sued to reduce the 
    purchase price by 50\%.
    \item The court held for Michigan Land. Judge Sherwood, who dissented a 
    year earlier in \emph{Sherwood}, wrote the opinion, holding that the 
    \emph{Sherwood} rule applies only when ``all the facts and circumstances 
    are precisely the same as in that.''\footnote{Casebook p. 739.}
\end{enumerate}

\paragraph{\emph{Griffith v. Brymer}}

\begin{enumerate}
    \item Griffith paid Brymer \textsterling 100 to rent a room to watch the 
    Coronation Procession.
    \item That morning, the King decided to undergo surgery and cancelled the 
    procession, unbeknownst to Griffith and Brymer.
    \item Griffith sued to recover the money.
    \item The court held that the agreement was based on a supposition that 
    made performance impossible, ``which went to the whole root of the 
    matter.''\footnote{Casebook p. 740.} Griffith could recover his 
    \textsterling 100.
\end{enumerate}

\paragraph{\emph{Wood v. Boynton}}

\begin{enumerate}
    \item Wood brought a small stone to Boynton, a jeweler. He offered her \$1 
    for it. She declined. Later, she needed money, so she returned to Boynton 
    and accepted his offer. It turned out to be an uncut diamond, though 
    neither of them knew it. Wood sued for rescission.
    \item Held for Boynton: ``However unfortunate the plaintiff may have been 
    in selling this valuable stone for a mere nominal sum, she has failed 
    entirely to make out a case either of fraud or mistake in the sale such as 
    will entitle her to a rescission~.~.~.''\footnote{Casebook p. 741.}
    \item Was there a mistake in this case? The court says no.
\end{enumerate}

\paragraph{\emph{Firestone \& Parson, Inc. v. Union League of Philadelphia}}

\begin{enumerate}
    \item % TODO 742
\end{enumerate}

\paragraph{\emph{Everett v. Estate of Sumstad}}

\begin{enumerate}
    \item % TODO 742
\end{enumerate}

\paragraph{Restatement Second \S\S\ 151, 152, 154}

\begin{enumerate}
    \item % TODO supp
\end{enumerate}

\paragraph{UNIDROIT Arts. 3.4, 3.5, 3.18}

\begin{enumerate}
    \item % TODO supp
\end{enumerate}

\paragraph{Principles of European Contract Law Art. 4.103}

\begin{enumerate}
    \item % TODO supp
\end{enumerate}

\paragraph{\emph{Lenawee County Board of Health v. Messerly}}

\begin{enumerate}
    \item % TODO 743
\end{enumerate}

\paragraph{\emph{Beachcomber Coins, Inc. v. Boskett}}

\begin{enumerate}
    \item % TODO 749
\end{enumerate}

\subsection{The Effect of Unexpected Circumstances}

% TODO 765 ff.
