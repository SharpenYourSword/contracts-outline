\section{Introduction}

\begin{enumerate}
    \item Contracts: voluntary obligations---imposed on the basis of a 
    commitment, e.g., a promise. Emphasizes personal autonomy.
    \item Torts: involuntary obligations---imposed regardless of prior 
    commitments.
    \item ``Morality justifies promise-keeping more than law ever 
    will.''---Berring.
    \item Relational contract theory: contracts involve trust and long-term 
    relationships, rather than just discrete transactions.
    \item Spectrum of agreement: Promise $\rightarrow$ Gift $\rightarrow$ 
    Contractual Relationship.
    \item Contract disputes involve someone trying to get out of an 
    obligation---usually an obligation arising from an agreement that person 
    created. Who wants in and who wants out?
    \item Fungible goods are interchangeable with property of the same kind 
    (wheat, corn, money). Nonfungible goods are not (land, your wedding ring).
\end{enumerate}

\subsection{Rawls, ``A Theory of Justice''}

\begin{enumerate}
    \item A person must follow the rules when ``he has voluntarily accepted 
    the benefits of the scheme or has taken advantage of the opportunities of 
    offers to advance his interests~.~.~.~''\footnote{Fuller and Eisenberg, 
    \emph{Basic Contract Law}, 8th Ed., p. 1.}
    \item Making a promise means accepting the social benefits it 
    brings---e.g., the guarantee that the other person will hold up his end of 
    the bargain. This obligation is self-imposed.
    \item ``~.~.~.~promising is an act done with the public intention of 
    deliberately incurring an obligation the existence of which in the 
    circumstances will further one's ends.''\footnote{Casebook p. 2.}
\end{enumerate}

\subsection{Scanlon, ``Promises and Practices''}

\begin{enumerate}
    \item The ``Guilty Secret'': Harold asks you not to tell a particular 
    embarrassing story about him to his new colleagues. You agree.
    \item You now have two moral reasons not to tell the story:
    \begin{enumerate}
        \item It would gratuitously hurt Harold.
        \item It would violate your obligation.
    \end{enumerate}
    \item Where does your obligation come from?
    \begin{enumerate}
        \item Harold might have relied in small part on your promise. He 
        didn't get down and beg, for instance. But for the most part, your 
        promise did not affect his conduct.
        \item The promise itself has inherent value. First, it gives Harold 
        peace of mind. More importantly, Harold has a reason for wanting the 
        assurance, and you have reasons for giving it.
        % QUESTION: What if your reasons for making the promise no longer 
        % exist? Do you still have an obligation to keep Harold's secret?
    \end{enumerate}
    \item \textbf{``Principle F''} describes the mutual obligations between 
    the promisor and promisee.\footnote{Casebook p. 3.}
    \item In ``Principle F'' cases, (1) it would be wrong for the promisor not 
    to perform, and (2) the promisee has a right to rely on performance and a 
    right to enforce if the other party breaks the promise.
\end{enumerate}

\subsection{Gardner, ``Observations in the Course of Contracts''}

\begin{enumerate}
    \item Four principles underly ethical problems in 
    contracts:\footnote{Casebook p. 4.}
    \begin{enumerate}
        \item \emph{The Tort Idea}: promisor must pay for losses from 
        promisee's reliance.
        \item \emph{The Bargain Idea}: buyer must pay the agreed price to the 
        seller.
        \item \emph{The Promissory Idea}: promises are inherently binding.
        \item \emph{The Quasi-Contractual Idea}: anyone who receives anything 
        of value should pay for it unless it's a gift.
    \end{enumerate}
    \item The first and last principles suggest justice happens \emph{after} 
    the transaction, and courts exist to correct deals gone wrong.
    \item In the second and third principles, parties settle justice 
    \emph{before} their voluntary transaction.
\end{enumerate}

\subsection{History of Contract Law}

\begin{enumerate}
    \item Blackstone/Kent: synthesis.
    \item David Field/Joseph Story: treatises.
    \item Langdell: scientific certainty.
    \item Williston: objectivism.
    \item Holmes: sui generis.
    \item Corbin: context.
    \item Grant Gilmore: social settings.
    \item Cardozo.
    \item Posner.
    \item Frank Easterbrook.
\end{enumerate}
