\section{Remedies}

\subsection{Introduction to Contract Damages}

\subsubsection{Restatement Second \S\ 344: Purposes of Remedies}

\begin{enumerate}
    \item \textbf{Expectation}: puts promisee in the position he would have been in if 
    the contract had been \emph{performed}.
    \item \textbf{Reliance}: puts promisee in the position he would have been in 
    if the contract had \emph{not been made}.
    \item \textbf{Restitution}: restores to the promisee any benefit he 
    conferred to the promisor.
\end{enumerate}

\subsubsection{\emph{Hawkins v. McGee}}

The appropriate measure of damages is the difference between (1) performance 
of the contract as promised and (2) actual performance. In this case, damages 
are equal to the difference between the promised ``perfect hand'' and the 
actual state of the hand after the operation.

\begin{enumerate}
    \item The plaintiff, Hawkins, had scar tissue on his hand that resulted 
    from a severe burn from contact with an electrical wire. Dr. McGee removed 
    the scar tissue and grafted some skin from Hawkins's chest onto his hand. 
    Hawkins ended up in the hospital for three months and lost the use of his 
    hand. Also, the new tissue apparently ``filled his hand with a `matted' 
    and `unsightly growth'---presumably hair.''\footnote{Casebook p. 194.}
    \item Hawkins sued for breach of an alleged warranty of the success of the 
    operation. The trial court found for Hawkins. Hawkins also brought a 
    negligence claim, which the trial court dismissed on nonsuit.
    \item After trial, the defendant moved to set aside the verdict as 
    excessive. The trial court ordered the verdict set aside unless Hawkins 
    would remit all damages above \$500. He refused. Exceptions on both sides 
    were transferred to the Supreme Court.
    \item Part 1: was there a contract, and if so, what were its terms?
    \begin{enumerate}
        \item The parties agreed that McGee had said that after the operation, 
        Hawkins would be in the hospital for three or four days, and after 
        going home, he would have a ``perfect hand'' for three or four 
        days.\footnote{Casebook p. 190.} The court held that this alone would 
        not have established a contract guaranteeing that Hawkins would 
        \emph{definitely} be home after three or four days or could go to work 
        a few days after being home. It was only an indication of the probable 
        outcome of the surgery. But such a contract would have been 
        established if Dr. McGee has promised a ``\emph{hundred per cent} 
        perfect hand.''
        \item But the question of the wording of the promise was irrelevant 
        because Dr. McGee had ``repeatedly solicited'' the opportunity to 
        perform the operation. He \enquote{sought an opportunity to 
        `experiment on skin grafting'}. Because these words constituted an 
        inducement of consent, it was reasonable to conclude that the parties 
        had entered into a contract.
    \end{enumerate}
    \item Part 2: what damages were appropriate?
    \begin{enumerate}
        \item The jury instructions requested damages for (1) pain and 
        suffering and (2) ``ill effects of the operation on the plaintiff's 
        hand.''\footnote{Casebook p. 191.}
        \item The appropriate damages would have been the difference between 
        (1) the promised ``perfect hand'' and (2) the state of the hand after 
        the operation. Pain and suffering were part of the bargain, and 
        physical harm to the plaintiffs hand would be subsumed into the 
        rule of damages the court adopted here.
    \end{enumerate}
    \item Part 3: were McGee's requested jury instructions appropriate?
    \begin{enumerate}
        \item No. They were ``loosely drawn'' and ``extremely 
        misleading.''\footnote{Casebook p. 193.}
    \end{enumerate}
\end{enumerate}

\subsubsection{Cooter \& Eisenberg, ``Damages for Breach of Contract''}

``The purpose of the social institution of bargain is to create joint value 
through exchange.''\footnote{Casebook p. 203.} Protecting the expectation 
interest maximizes the value of contracts.

\begin{enumerate}
    \item Should we prefer reliance or expectation damages?
    \item Sometimes, the difference doesn't matter. For instance, in a 
    perfectly competitive market, reliance damages will equal expectation 
    damages because the contract price will equal the forgone price the 
    promisee could have contracted with a third party. But it matters if 
    market conditions are far from perfectly competitive.
    \item Traditionally, courts were united in the view that expectation 
    damages were appropriate. ``In fact, prior to the 1930's the reliance 
    principle operated only in a covert manner.''\footnote{Casebook p. 200.}
    \item The reliance principle expanded after \S\ 90 of the Restatement 
    Second (1932) and Fuller and Perdue, \emph{The Reliance Interest in 
    Contract Damages} (1934).
    \item Should the reliance principle expand contract liability or reduce it 
    by substituting reliance liability for expectation liability?
    \begin{enumerate}
        \item Reliance liability introduced a tort conception of liability.
        \item There is not a clear rationale for expectation damages.
    \end{enumerate}
    \item So: what should happen when reliance and expectation damages 
    diverge?
    \begin{enumerate}
        \item \emph{Efficiency}: the value created by a contract.
        \item \emph{Distribution}: how a contract's value is divided among the 
        parties.
        \item \emph{Nonprice} terms of a contract make it possible to control 
        a contract's efficiency.\footnote{Casebook p. 201.}
        \item \emph{Price} terms make it possible to control distribution.
        \item Damages should be fair and efficient. ``~.~.~.~we take as a 
        theorem that a damage rule is both fair and efficient if it 
        corresponds to the terms that rational parties situated like the 
        contracting parties would have reached when bargaining under ideal 
        conditions.''\footnote{Casebook p. 202.}
        \item Would rational parties prefer reliance or expectation damages?
        \begin{enumerate}
            \item \emph{Administrative effects}: reliance is difficult to 
            measure and prove, while expectation damages are based on the 
            known contract price, rather than the speculative forgone price.
            \item \emph{Incentive effects}: if performance or alternative 
            performance become more profitable than performance of the 
            original contract, the promisor has to decide whether or not to 
            breach. If there are only reliance damages, he will rationally 
            breach. But if there are expectation damages, he will have to 
            account for the promisee's loss of his share of the value of the 
            contract. In other words, he will not internalize the value of the 
            performance to the promisee.\footnote{Casebook p. 202--03.}
            \item ``Thus expectation damages create efficient incentives for 
            the promisor's performance, while reliance damages do not, unless 
            they are identical to expectation damages.''
            \item Expectation damages also give the promisee more confidence 
            in planning.
        \end{enumerate}
    \end{enumerate}
\end{enumerate}

\subsubsection{\emph{U.S. Naval Inst. v. Charter Commc'ns, Inc.}}

\begin{enumerate}
    \item Facts:
    \begin{enumerate}
        \item Plaintiff: US Naval Institute. Defendants: Charter Communications 
        and Berkley Publishing Group.
        \item September 1984: Naval (assignee of the copyright interests in 
        \emph{The Hunt for Red October}) agreed to grant Berkley exclusive 
        paperback rights, not to be published before October 1985.
        \item September 15, 1985: retail sales of the paperback began after 
        Berkley shipped the books early.
    \end{enumerate}
    \item The trial court dismissed, holding that Berkeley was entitled as a 
    matter of industry custom to ship the books early. The Second Circuit 
    reversed, holding that the October 1985 start date must be upheld. After 
    the initial remand, the trial court awarded Naval \$35,380.50 in damages, 
    \$7,760.12 in profits wrongly received by Berkley, \$15,319.27 in 
    prejudgment interest on damages, and costs. Both parties appealed.
    \item Berkley's pre-October sales were \$724,300. Under copyright 
    infringement, Naval sought all of the profits, as well as interest and 
    costs.
    \item Berkley argued that it could not be held liable for copyright 
    infringement because it became the exclusive licensee of the paperback 
    edition copyright on September 14, 1984, so Naval should not have been 
    entitled to any recovery. At most, Naval had a contract claim, but Naval 
    had disavowed that claim.
    \item The trial court held that there was copyright infringement, and that 
    Naval was entitled to ``actual damages'' and \enquote{profits 
    `attributable to the infringement'}.\footnote{Casebook p. 205.} It 
    calculated the damages as its lost hardcover sales attributable to the 
    paperback publication, i.e., the hardcover sales difference between August 
    and September 1985, which amounted to \$35,280.50. It calculate the lost 
    profits as those from sales to customers who would not have bought the 
    hardcover, or \$7,760.12.\footnote{Casebook p. 205.}
    \item On appeal (for the second time):
    \begin{enumerate}
        \item There was no copyright infringement because Berkley became the 
        copyright holder on September 14, 1984. However, Naval was entitled to 
        contract damages:
        \begin{enumerate}
            \item ``~.~.~.~the purpose of damages for breach of contract is to 
            compensate the injured party for the loss caused by the 
            breach.''\footnote{Casebook p. 206.} Damages must therefore match 
            the plaintiff's actual loss.
            \item \emph{Lost profits damages}: affirmed.
            \item \emph{Profits attributable to infringement}: plaintiff would 
            not have received this income if the infringement had not 
            occurred. The only purpose of these damages would be punitive, and 
            contract law does not recognize punitive damages. Reversed.
        \end{enumerate}
    \end{enumerate}
\end{enumerate}

\subsubsection{Profiting from Breach: \emph{Coppola Enters., Inc. v. Alfone}}

``A seller will not be permitted to profit from his breach of contract with a 
buyer, even absent proof of fraud or bad faith, when the breach is followed by 
a sale of the land to a subsequent purchaser.''\footnote{Casebook p. 208.}

\begin{enumerate}
    \item Helen Alfone contracted with Coppola to buy a lot and home for 
    \$105,690. She put down a \$10,568 deposit.
    \item Coppola sent a letter informing Alfone of the tentative closing 
    date. The closing date was to happen within 10 days of written notice.
    \item Alfone tried but failed to find financing. Her lawyer requested an 
    extension to pay the balance due on the property. Coppola refused and 
    resold the property for \$170,000.
    \item The trial court found for Alfone, holding that Coppola had failed to 
    act in good faith by allowing a reasonable time to close and by 
    terminating the contract. She was awarded ``benefit of bargain'' damages 
    for \$64,310---i.e., the difference between Coppola's sale price to Alfone 
    and the second seller---plus interest. Affirmed on appeal. 
    \item \emph{Gassner v. Lockett}: an old and forgetful seller sold property 
    to a buyer and then inadvertently sold it again to a second buyer at a 
    higher price. Even though the seller acted in good faith, the first buyer 
    was entitled to recover the seller's profits from the second sale.
    \item Under the \emph{Gassner} rule, Alfone was entitled to recover 
    Coppola's profits.
\end{enumerate}

\subsubsection{\emph{Laurin v. DeCarolis Constr. Co.}}

\begin{enumerate}
    \item The Laurins bought a home under construction from DeCarolis. While 
    construction continued, DeCarolis bulldozed trees and removed gravel and 
    loam. After paying the purchase price, the Laurins sued to recover the 
    value of the lumber, gravel, and loam.
    \item The trial court held that the Laurins were entitled to recover. The 
    court here reversed, holding that the Laurins had contract rights rather 
    than property rights---however, although the Laurins had not suffered a 
    loss [?---but aren't the Laurins now unable to profit from that same 
    lumber, gravel, and loam?], DeCarolis was nonetheless liable because it 
    ``should not be allowed to retain is gains from a willful breach of 
    contract.''\footnote{Casebook p. 209.}
\end{enumerate}

% \subsubsection{The Theory of Efficient Breach}
% 
% \begin{enumerate}
%     \item % TODO 209-214
% \end{enumerate}
% 
% \subsubsection{\emph{Greer Props., Inc. v. LaSalle Nat'l Bank}}
% 
% \begin{enumerate}
%     \item % TODO 214-215
% \end{enumerate}
% 
% \subsubsection{Punitive Damages}
% 
% \begin{enumerate}
%     \item % TODO 215-216
% \end{enumerate}
% 
% \subsection{The Expectation Measure}
% 
% \subsubsection{Damages for Breach of a Contract to Perform Services}
% 
% % TODO 217-28, 228-41
% 
% \subsubsection{Damages for Breach of a Contract for the Sale of Goods}
% 
% % TODO 241-260
% 
% \subsubsection{Mitigation; Contracts for Employment}
% 
% \subsubsection{Foreseeability}
% 
% \subsubsection{Certainty}
% 
% \subsubsection{Damages for Mental Distress}
% 
% \subsubsection{Liquidated Damages}
% 
% \subsection{Specific Performance}
% 
% \subsection{The Reliance and Restitution Measures}
% 
% \subsubsection{Reliance Damages in a Bargain Context}
% 
% \subsubsection{The Restitution Measure}
% 
