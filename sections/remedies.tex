\section{Remedies}

\subsection{Introduction to Contract Damages}

\subsubsection{Restatement Second \S\ 344: Purposes of Remedies}

\begin{enumerate}
    \item \textbf{Expectation}: puts promisee in the position he would have been in if 
    the contract had been \emph{performed}. You get what you bargained for.
    \item \textbf{Reliance}: puts promisee in the position he would have been in 
    if the contract had \emph{not been made}.
    \item \textbf{Restitution}: restores to the promisee any benefit he 
    conferred to the promisor.
\end{enumerate}

\subsubsection{Expectation Damages: \emph{Hawkins v. McGee}}

The appropriate measure of damages is the difference between (1) performance 
of the contract as promised and (2) actual performance. In this case, damages 
are equal to the difference between the promised ``perfect hand'' and the 
actual state of the hand after the operation.

\begin{enumerate}
    \item The plaintiff, Hawkins, had scar tissue on his hand that resulted 
    from a severe burn from contact with an electrical wire. Dr. McGee removed 
    the scar tissue and grafted some skin from Hawkins's chest onto his hand. 
    Hawkins ended up in the hospital for three months and lost the use of his 
    hand. Also, the new tissue apparently ``filled his hand with a `matted' 
    and `unsightly growth'---presumably hair.''\footnote{Casebook p. 194.}
    \item Hawkins sued for breach of an alleged warranty of the success of the 
    operation. The trial court found for Hawkins. Hawkins also brought a 
    negligence claim, which the trial court dismissed on nonsuit.
    \item After trial, the defendant moved to set aside the verdict as 
    excessive. The trial court ordered the verdict set aside unless Hawkins 
    would remit all damages above \$500. He refused. Exceptions on both sides 
    were transferred to the Supreme Court.
    \item Part 1: was there a contract, and if so, what were its terms?
    \begin{enumerate}
        \item The parties agreed that McGee had said that after the operation, 
        Hawkins would be in the hospital for three or four days, and after 
        going home, he would have a ``perfect hand'' for three or four 
        days.\footnote{Casebook p. 190.} The court held that this alone would 
        not have established a contract guaranteeing that Hawkins would 
        \emph{definitely} be home after three or four days or could go to work 
        a few days after being home. It was only an indication of the probable 
        outcome of the surgery. But such a contract would have been 
        established if Dr. McGee has promised a ``\emph{hundred per cent} 
        perfect hand.''
        \item But the question of the wording of the promise was irrelevant 
        because Dr. McGee had ``repeatedly solicited'' the opportunity to 
        perform the operation. He \enquote{sought an opportunity to 
        `experiment on skin grafting'}. Because these words constituted an 
        inducement of consent, it was reasonable to conclude that the parties 
        had entered into a contract.
    \end{enumerate}
    \item Part 2: what damages were appropriate?
    \begin{enumerate}
        \item The jury instructions requested damages for (1) pain and 
        suffering and (2) ``ill effects of the operation on the plaintiff's 
        hand.''\footnote{Casebook p. 191.}
        \item The appropriate damages would have been the difference between 
        (1) the promised ``perfect hand'' and (2) the state of the hand after 
        the operation. Pain and suffering were part of the bargain, and 
        physical harm to the plaintiffs hand would be subsumed into the 
        rule of damages the court adopted here.
    \end{enumerate}
    \item Part 3: were McGee's requested jury instructions appropriate?
    \begin{enumerate}
        \item No. They were ``loosely drawn'' and ``extremely 
        misleading.''\footnote{Casebook p. 193.}
    \end{enumerate}
\end{enumerate}

\subsubsection{Cooter \& Eisenberg, ``Damages for Breach of Contract''}

``The purpose of the social institution of bargain is to create joint value 
through exchange.''\footnote{Casebook p. 203.} Protecting the expectation 
interest maximizes the value of contracts.

\begin{enumerate}
    \item (A response to Posner's theory of efficient breach.)
    \item Should we prefer reliance or expectation damages?
    \item Sometimes, the difference doesn't matter. For instance, in a 
    perfectly competitive market, reliance damages will equal expectation 
    damages because the contract price will equal the forgone price the 
    promisee could have contracted with a third party. But it matters if 
    market conditions are far from perfectly competitive.
    \item Traditionally, courts were united in the view that expectation 
    damages were appropriate. ``In fact, prior to the 1930's the reliance 
    principle operated only in a covert manner.''\footnote{Casebook p. 200.}
    \item The reliance principle expanded after \S\ 90 of the Restatement 
    Second (1932) and Fuller and Perdue, \emph{The Reliance Interest in 
    Contract Damages} (1934).
    \item Should the reliance principle expand contract liability or reduce it 
    by substituting reliance liability for expectation liability?
    \begin{enumerate}
        \item Reliance liability introduced a tort conception of liability.
        \item There is not a clear rationale for expectation damages.
    \end{enumerate}
    \item So: what should happen when reliance and expectation damages 
    diverge?
    \begin{enumerate}
        \item \emph{Efficiency}: the value created by a contract.
        \item \emph{Distribution}: how a contract's value is divided among the 
        parties.
        \item \emph{Nonprice} terms of a contract make it possible to control 
        a contract's efficiency.\footnote{Casebook p. 201.}
        \item \emph{Price} terms make it possible to control distribution.
        \item Damages should be fair and efficient. ``~.~.~.~we take as a 
        theorem that a damage rule is both fair and efficient if it 
        corresponds to the terms that rational parties situated like the 
        contracting parties would have reached when bargaining under ideal 
        conditions.''\footnote{Casebook p. 202.}
        \item Would rational parties prefer reliance or expectation damages?
        \begin{enumerate}
            \item \emph{Administrative effects}: reliance is difficult to 
            measure and prove, while expectation damages are based on the 
            known contract price, rather than the speculative forgone price.
            \item \emph{Incentive effects}: if performance or alternative 
            performance become more profitable than performance of the 
            original contract, the promisor has to decide whether or not to 
            breach. If there are only reliance damages, he will rationally 
            breach. But if there are expectation damages, he will have to 
            account for the promisee's loss of his share of the value of the 
            contract. In other words, he will not internalize the value of the 
            performance to the promisee.\footnote{Casebook p. 202--03.}
            \item ``Thus expectation damages create efficient incentives for 
            the promisor's performance, while reliance damages do not, unless 
            they are identical to expectation damages.''
            \item Expectation damages also give the promisee more confidence 
            in planning.
        \end{enumerate}
    \end{enumerate}
\end{enumerate}

\subsubsection{\emph{U.S. Naval Inst. v. Charter Commc'ns, Inc.}}

\begin{enumerate}
    \item Facts:
    \begin{enumerate}
        \item Plaintiff: US Naval Institute. Defendants: Charter Communications 
        and Berkley Publishing Group.
        \item September 1984: Naval (assignee of the copyright interests in 
        \emph{The Hunt for Red October}) agreed to grant Berkley exclusive 
        paperback rights, not to be published before October 1985.
        \item September 15, 1985: retail sales of the paperback began after 
        Berkley shipped the books early.
    \end{enumerate}
    \item The trial court dismissed, holding that Berkeley was entitled as a 
    matter of industry custom to ship the books early. The Second Circuit 
    reversed, holding that the October 1985 start date must be upheld. After 
    the initial remand, the trial court awarded Naval \$35,380.50 in damages, 
    \$7,760.12 in profits wrongly received by Berkley, \$15,319.27 in 
    prejudgment interest on damages, and costs. Both parties appealed.
    \item Berkley's pre-October sales were \$724,300. Under copyright 
    infringement, Naval sought all of the profits, as well as interest and 
    costs.
    \item Berkley argued that it could not be held liable for copyright 
    infringement because it became the exclusive licensee of the paperback 
    edition copyright on September 14, 1984, so Naval should not have been 
    entitled to any recovery. At most, Naval had a contract claim, but Naval 
    had disavowed that claim.
    \item The trial court held that there was copyright infringement, and that 
    Naval was entitled to ``actual damages'' and \enquote{profits 
    `attributable to the infringement'}.\footnote{Casebook p. 205.} It 
    calculated the damages as its lost hardcover sales attributable to the 
    paperback publication, i.e., the hardcover sales difference between August 
    and September 1985, which amounted to \$35,280.50. It calculate the lost 
    profits as those from sales to customers who would not have bought the 
    hardcover, or \$7,760.12.\footnote{Casebook p. 205.}
    \item On appeal (for the second time):
    \begin{enumerate}
        \item There was no copyright infringement because Berkley became the 
        copyright holder on September 14, 1984. However, Naval was entitled to 
        contract damages:
        \begin{enumerate}
            \item ``~.~.~.~the purpose of damages for breach of contract is to 
            compensate the injured party for the loss caused by the 
            breach.''\footnote{Casebook p. 206.} Damages must therefore match 
            the plaintiff's actual loss.
            \item \emph{Lost profits damages}: affirmed.
            \item \emph{Profits attributable to infringement}: plaintiff would 
            not have received this income if the infringement had not 
            occurred. The only purpose of these damages would be punitive, and 
            contract law does not recognize punitive damages. Reversed.
        \end{enumerate}
    \end{enumerate}
\end{enumerate}

\subsubsection{Profiting from Breach: \emph{Coppola Enters., Inc. v. Alfone}}

``A seller will not be permitted to profit from his breach of contract with a 
buyer, even absent proof of fraud or bad faith, when the breach is followed by 
a sale of the land to a subsequent purchaser.''\footnote{Casebook p. 208.}

\begin{enumerate}
    \item Helen Alfone contracted with Coppola to buy a lot and home for 
    \$105,690. She put down a \$10,568 deposit.
    \item Coppola sent a letter informing Alfone of the tentative closing 
    date. The closing date was to happen within 10 days of written notice.
    \item Alfone tried but failed to find financing. Her lawyer requested an 
    extension to pay the balance due on the property. Coppola refused and 
    resold the property for \$170,000.
    \item The trial court found for Alfone, holding that Coppola had failed to 
    act in good faith by allowing a reasonable time to close and by 
    terminating the contract. She was awarded ``benefit of bargain'' damages 
    for \$64,310---i.e., the difference between Coppola's sale price to Alfone 
    and the second seller---plus interest. Affirmed on appeal. 
    \item \emph{Gassner v. Lockett}: an old and forgetful seller sold property 
    to a buyer and then inadvertently sold it again to a second buyer at a 
    higher price. Even though the seller acted in good faith, the first buyer 
    was entitled to recover the seller's profits from the second sale.
    \item Under the \emph{Gassner} rule, Alfone was entitled to recover 
    Coppola's profits.
\end{enumerate}

\subsubsection{Diminution in Value: \emph{Laurin v. DeCarolis Constr. Co.}}

\begin{enumerate}
    \item The Laurins bought a home under construction from DeCarolis. While 
    construction continued, DeCarolis bulldozed trees and removed gravel and 
    loam. After paying the purchase price, the Laurins sued to recover the 
    value of the lumber, gravel, and loam.
    \item The trial court held that the Laurins were entitled to recover. The 
    court here reversed, holding that the Laurins had contract rights rather 
    than property rights---however, although the Laurins had not suffered a 
    loss, DeCarolis was nonetheless liable because it ``should not be allowed 
    to retain is gains from a willful breach of contract.''\footnote{Casebook 
    p. 209.}
\end{enumerate}

\subsubsection{The Theory of Efficient Breach}

\begin{enumerate}
    \item Posner, theory of efficient breach: ``breach of contract is efficient, and 
    therefore desirable, if the promisor's gain from breach, after payment of 
    expectation damages, will exceed promisee's loss from 
    breach.''\footnote{Casebook p. 209.}
    \item Fuller and Eisenberg challenge the theory in the context of the 
    Overbidder Paradigm, in which a seller breaches a contract with a buyer in 
    favor of a sale with another buyer willing to pay more. They contend that 
    efficient breach is inefficient in this case for three 
    reasons:\footnote{Casebook pp. 210--14.}
    \begin{enumerate}
        \item It remakes the parties' contract into a contract that allows and 
        encourages sellers to seek overbidders.
        \item It would reduce rewards for planning and investment.
        \item It undermines the morality of promise keeping.
    \end{enumerate}
\end{enumerate}

\subsubsection{Rejecting Efficient Breach: \emph{Greer Props., Inc. v. LaSalle 
Nat'l Bank}}

Efficient breach is a breach of the covenant of good faith and fair dealing.

\begin{enumerate}
    \item LaSalle agreed to sell land to Searle for \$1.1 million, with Searle 
    having the right to terminate the agreement if the soil was contaminated. 
    The deal fell through because Searle discovered the soil was contaminated 
    and would cost \$500,000 to clean up.
    \item LaSalle then contracted with Greer to sell the property for \$1.25 
    million, with the option to terminate if the cost of cleanup became 
    impracticable. It estimated the cost to be \$100,000-\$200,000.
    \item LaSalle then negotiated a \$1,455,000 contract with Searle and 
    terminated its contract with Greer under the cleanup provision.
    \item Greer brought suit for specific performance and damages. The trial 
    court held for LaSalle. On appeal, the Seventh Circuit reversed, holding 
    that by contracting with Greer, LaSalle had given up the right to seek a 
    better price. Breaching the contract violated the covenant of good faith 
    and fair dealing.
\end{enumerate}

\subsection{The Expectation Measure}

\subsubsection{Damages for Breach of a Contract to Perform Services}

\paragraph{\emph{Louise Caroline Nursing Home, Inc. v. Dix Construction Co.}}

\begin{enumerate}
    \item Dix breached a contract to build a nursing home.
    \item An auditor found that Dix had breached the contract, but that Louise 
    Caroline had suffered no compensable damages because ``the cost to 
    complete the nursing home was within the contract price less what had been 
    paid to Dix''\footnote{Casebook p. 218.}---i.e., if the cost to completion 
    was within the contract price, the nursing home could just pay another 
    contractor to finish the job.
    \item The nursing home argued that it was entitled to the difference 
    between what Dix did and what it promised---i.e., expectation damages. It 
    argued that it was entitled to the ``benefits of its 
    bargain.''\footnote{Casebook p. 218.} 
    \item The court held that the principle underlying damages is that 
    compensation ``is the value of the performance of the contract.'' Applying 
    this principle to the case of uncompleted construction contracts, the 
    court held that damages ``can only be in the amount of the reasonable cost 
    of completing the contract and repairing the defendant's defective 
    performance less such part of the contract price as has not been 
    paid.''\footnote{Casebook p. 219.} It affirmed the auditor's application 
    of the ``cost of completion'' standard.
\end{enumerate}

\paragraph{\emph{Peevyhouse v. Garland Coal \& Mining Co.}}

\begin{enumerate}
    \item Garland contracted to strip mine on the Peevyhouse's farm. Garland 
    agreed to perform remedial work on the land when it was done.
    \item It turned out that the remedial work would cost \$25,000 or more, 
    but would result in only \$300 in improvements to the property. Garland 
    refused to perform.
    \item The trial court directed a verdict in favor of the plaintiffs. It 
    left the amount of damages to the jury, which returned an award of 
    \$5,000---less than the cost of performance but more than the value of the 
    improvement to the land, and indeed more than the land itself.
    \item The court held that in coal mining cases where remedial provisions 
    are incidental, damages for breach of the remedial provision are limited 
    to the diminution to the value of the property resulting from 
    non-performance---in this case, \$300.
    \item Dissent: the parties were well aware of all the conditions when they 
    signed the contract. The majority's holding ``completely rescinds and 
    holds for naught the solemnity of the contract before us and makes an 
    entirely new contract for the parties.''\footnote{Casebook p. 225.}
\end{enumerate}

<<<<<<< HEAD
\paragraph{Affirming the \emph{Peevyhouse} Rule: \emph{Schneberger v. Apache 
Corp.}}
=======
% TODO 228-41
>>>>>>> 40f97d1ea557fc71deaa5bf58cb8442c6482db3b

The court affirmed the \emph{Peevyhouse} ``diminution in value'' rule.

\begin{enumerate}
    \item Facts similar to \emph{Peevyhouse} arose in Oklahoma. Apache Corp. 
    drilled for oil on the Schnebergers' property. Restoration of the land 
    would have cost \$1.3 million, but the increase in property value would 
    have been only \$5,175. 
    \item The court affirmed \emph{Peevyhouse}, despite recent statutory 
    revisions. ``~.~.~.~where the cost is grossly disproportionate to the cost 
    of reclamation as in \emph{Peevyhouse}, a review of recent case law 
    suggests that courts are adhering to the diminution in value 
    award~.~.~.~''\footnote{Casebook pp. 230--31.}
\end{enumerate}

\paragraph{Grossly Disproportionate: \emph{H.P. Droher \& Sons v. Toushin}}

\begin{enumerate}
    \item A homebuilder built a \$44,000 house. A steel post was too low, 
    causing the floor to sag significantly. It would have cost more than 
    \$20,000 to fix, but the benefit to the property value would have been far 
    less than \$20,000. The court held that the diminution in value rule 
    (\emph{Peevyhouse}) applies when ``the cost of remedying the defects is 
    grossly proportionate to the benefits~.~.~.~''\footnote{Casebook p. 231.}
\end{enumerate}

\paragraph{Waste: \emph{Eastern Steamship Lines, Inc. v. United States}}

\begin{enumerate}
    \item The government chartered a private vessel during WWII, promising to 
    return it in the same condition. After the war, the owner sued the 
    government for \$4 million, which was the cost to return it to its prewar 
    condition. But the value of the ship would have been only \$2 million. 
    \item The court held that paying the full cost of restoration would be ``a 
    useless and wasteful expenditure of public 
    funds~.~.~.~''\footnote{Casebook p. 232.}
\end{enumerate}

\paragraph{Aesthetics and Good Faith: \emph{City School Dist. of the City of 
Elmira v.  McLane Const.  Co.}}

When the damages are aesthetically important, and when the builder does not 
act in good faith, the diminution in value rule does not apply. Rather, the 
owner can recover the full cost of repair.

\begin{enumerate}
    \item A school district contracted for the construction of a swimming pool 
    with special wooden beams. Repairs would have cost \$357,000 for an 
    increase in value of only \$3,000. Because ``the building was to be a 
    showplace,'' and because the builder knew its methods would cause damage, 
    the court held for the school district.
\end{enumerate}

\paragraph{Residential vs. Commercial Construction: \emph{Fox v. Webb}}

\begin{enumerate}
    \item ``~.~.~.~a distinction exists between a contract to construct a 
    dwelling for the owner who plans to live therein and a contract to 
    construct a commercial structure where the aesthetic taste of the owner is 
    not so deeply involved.''\footnote{Casebook p. 233.}
\end{enumerate}

\paragraph{Measuring Diminished Value: \emph{Grossman Holdings Ltd. v. 
Hourihan}}

Diminished value damages should be measured at the date of the breach, not the 
original date of the contract.

\begin{enumerate}
    \item The contractor built the house backwards. The trial court held for 
    the owner, but refused to award damages because (1) cost-of-completion 
    damages would be wasteful and (2) diminished-value damages would be 
    improper because the house's value had increased substantially since the 
    date of the contract.
    \item The appellate court held that diminished-value damages should be 
    measured at the date of the breach.
\end{enumerate}

\paragraph{Factfinder's Discretion: \emph{Advanced, Inc. v. Wilks}}

\begin{enumerate}
    \item Cost-of-completion damages will sometimes put the owner in a better 
    economic position than he would have been in if the contract had been 
    fully performed. The factfinder should determine whether the owner is 
    likely to pocket the difference or actually put it towards repairs.
\end{enumerate}

\paragraph{\emph{Ruxley Electronics \& Construction Ltd. v. Forsyth}}

\begin{enumerate}
    \item Forsyth contracted with Ruxley to build a swimming pool with a 
    maximum depth of 7'6.'' It ended up building a pool only 6' deep. The 
    trial court found that 6' was safe for diving, so it did not award the 
    cost of repair, \textsterling 21,560. However, it did award ``pleasure 
    and amenity'' damages of \textsterling 2,500.\footnote{Casebook p. 235.}
    \item Cost-of-completion and diminished-value damages are not the only 
    possibilities. The measure should be ``the loss truly suffered by the 
    promisee.''\footnote{Casebook p. 236.}
\end{enumerate}

\paragraph{Calculating Damages: \emph{Aiello Const., Inc. v. Nationwide 
Tractor Trailer Training and Placement Corp.}}

\begin{enumerate}
    \item Defendant agreed to pay \$33,000 for construction work. It stopped 
    paying after paying \$10,500.
    \item The trial court calculated damages for the plaintiff as as: 
    plaintiff's costs so far (\$21,500) plus expected profit from performance 
    (\$3,000) minus payments already made (\$10,500) = \$14,000, plus interest 
    = \$16,800.
\end{enumerate}

\paragraph{\emph{Formulas for Measuring Damages for Breach by a Person Who has 
Contracted to Have Services Performed}}

\begin{enumerate}
    \item There are two formulae for calculating a contractor's damages from 
    the owner's breach:
    \begin{enumerate}
        \item Contractor's expenses + contractor's lost profits - amount paid 
        by owner prior to breach. \emph{Aiello.}
        \item Contract price - cost of completion that the contractor saved - 
        amount paid by owner prior to breach. Restatement Second \S\ 346.
    \end{enumerate}
    \item The two are algebraically equivalent, but courts usually use the 
    second.
\end{enumerate}

\paragraph{Restatement Second \S\ 347: Measure of Damages in General}

\begin{enumerate}
    \item A contracts to build a house for B for \$100,000. B breaches 
    halfway through. A would have to spend \$60,000 to finish. A's damages are 
    the contract price (\$100,000) - cost of completion that the contractor 
    saved (\$60,000) = \$40,000. A can recover \$40,000 minus any payments 
    already made.\footnote{Casebook p. 239.}
\end{enumerate}

\paragraph{Multiple Contracts: \emph{Wired Music, Inc. v. Clark}}

\begin{enumerate}
    \item Wired provided music through rented telephone lines. Clark stopped 
    paying, but another tenant at the same location offered to take up Clark's 
    contract. Wired refused, and charged the new tenant a higher monthly rate. 
    Wired sued to recover lost profits from Clark, minus the cost of renting 
    the wire. The court held for Wired because it was possible for Wired to 
    fulfill multiple contracts.
\end{enumerate}

\paragraph{Overhead: \emph{Vitex Mfg. Corp. v. Caribtex Corp.}}

\begin{enumerate}
    \item ``~.~.~.~overhead should be considered to be a compensable item of 
    damage.''\footnote{Casebook p. 241.}
\end{enumerate}

\subsubsection{Damages for Breach of a Contract for the Sale of Goods}

\paragraph{Damages for the Seller's Breach of a Contract for the Sale of 
Goods}

\begin{enumerate}
    \item The UCC governs breach of contract for sale of goods. There are two 
    types of buyer's remedies for a seller's breach:
    \begin{enumerate}
        \item Specific relief.
        \item Damages.
        \begin{enumerate}
            \item Remedies when the seller fails to deliver or the buyer 
            rightfully revokes acceptance.
            \item Remedies when the goods are defective (usually, breach of 
            warranty).
        \end{enumerate}
    \end{enumerate}
\end{enumerate}

\paragraph{UCC \S\S\ 2-711(1) \emph{et seq.}}

\begin{enumerate}
    \item % TODO 242, and supp
\end{enumerate}

\paragraph{CISG Arts. 45, 49, 50, 74, 75, 76}

\begin{enumerate}
    \item % TODO 242, and supp
\end{enumerate}

\paragraph{\emph{Continental Sand \& Gravel, Inc. v. K \& K Sand \& Gravel, 
Inc.}}

\begin{enumerate}
    \item % TODO 242
\end{enumerate}

\paragraph{Direct Damages for Breach of Warranty: \emph{Manouchehri v. Heim}}

\begin{enumerate}
    \item UCC \S\ 2-714(2) sets the measure of direct damages for breach of 
    warranty as the difference between the value of the goods as warranted and 
    the value of the goods as accepted. Cost of repair is often a good 
    approximation of this value, so courts commonly award it as direct 
    damages.
\end{enumerate}

\paragraph{\emph{Egerer v. CSR West, LLC}}

\begin{enumerate}
    \item 1995: Egerer needed fill to develop his land. He needed 17,000 more cubic 
    yards than his normal supplier could supply (at \$1.10 per cubic yard). 
    CSR was engaged in an excavation project and agreed to sell all of its 
    excavations to Egerer for \$0.50 per cubic yard.
    \item 1997: After delivering for two nights, changed circumstances made it 
    more 
    profitable for CSR to leave the material at the original site, so it 
    stopped delivering to Egerer.
    \item 1998: Egerer found a new supplier who quoted \$8.25 per cubic yard.
    \item 1999: Egerer found a new supplier and bought material at \$6.39 per 
    cubic yard.
    \item 2000: Egerer sued CSR, alleging breach and demanding damages under 
    the UCC.
    \item There are two UCC remedies available to buyers when sellers fail to 
    deliver goods sold:
    \begin{enumerate}
        \item \emph{Cover}: the buyer purchases a replacement. Damages are the 
        difference between the replacement price and the contract price. \S\ 
        7-212.
        \item \emph{Hypothetical cover}: the buyer recovers the difference 
        between the market price at the time of the breach and the contract 
        price. \S\ 7-213.
    \end{enumerate}
    \item The trial court found that Egerer could recover under 7-213 for 
    \$8.25 per cubic yard, or \$129,812.50 total.
    \item On appeal, CSR agreed that 7-213 applied, but argued that \$8.25 was 
    not ``the price for goods of the same kind,'' because the material was a 
    different type.\footnote{Casebook p. 245.} It argued that the appropriate 
    price was \$1.10---the price from Egerer's original supplier.
    \item The court held that \$8.25 per cubic yard was appropriate because 
    (1) the UCC allows courts to use goods of a different quality and (2) 
    there was no evidence that suitable material was available at the time at 
    a lower price.
\end{enumerate}

\paragraph{\emph{Panhandle Agri-Service, Inc. v. Becker}}

\begin{enumerate}
    \item Under the UCC, buyers can choose between cover and damages for 
    nondelivery.
\end{enumerate}

\paragraph{J. White \& R. Summers, ``Uniform Commercial Code''}

\begin{enumerate}
    \item When a buyer has covered, can he ignore 2-712 and sue for a larger 
    contract-market differential under 2-713?
    \item The goal of contract remedies is to put the buyer in the position he 
    would have been in if there had been no breach. But allowing a buyer to 
    ignore 2-712 lets him wait to cover when market prices drop, and then 
    recover under 2-713 for the higher damages of the earlier contract-market 
    differential.
    \item We should force covering buyers to use 2-712.
\end{enumerate}

\paragraph{The Availability of Market-Price Damages to a Buyer Who Has Covered}

\begin{enumerate}
    \item The NCCSL and ALI updated the UCC to force covering buyers to use 
    2-712---but most states have not yet adopted the 
    amendment.\footnote{Casebook p. 247.}
\end{enumerate}

\paragraph{\emph{Delchi Carrier SpA v. Rotorex Corp.}}

\begin{enumerate}
    \item Facts:
    \begin{enumerate}
        \item January 1988: Rotorex agreed to sell 10,800 compressors to 
        Delchi for its air conditioners to be delivered in three shipments 
        before May 15. Rotorex sent a sample compressor that met Delchi's 
        specifications.
        \item March 26, 1988: Rotorex sent its first shipment, which arrived 
        at Delchi on April 20.
        \item May 9: Rotorex send a second shipment. While it was en route, 
        Delchi discovered that the compressors were defective.
        \item May 13: a Rotorex rep visited the Delchi factory. Delchi 
        informed Rotorex that 93\% of the compressors were unsuitable. Delchi 
        asked Rotorex to supply new compressors. Rotorex refused.
        \item May 23: Delchi canceled the contract. It expedited a previous 
        order of compressors from Sanyo, but was unable to obtain substitutes 
        from other sources, so it lost suffered sales losses during the 1988 
        season. 
    \end{enumerate}
    \item Delchi brought suit under CISG. In 1991, the court granted Delchi's 
    motion for partial summary judgment. After discovery and a bench trial, 
    the court found Rotorex liable for ~\$1.2 million for (i) lost profits, 
    (ii) expenses, (iii) expediting the Sanyo shipment, and (iv) handling and 
    storing the rejected compressors.
    \item The court denied Delchi's other claims---(i) shipping, etc., 
    of the Rotorex compressors, (ii) obsolete parts to be used only for the 
    Rotorex compressors, (iii) obsolete tooling, and (iv) labor costs when 
    Delchi's factory was idle---on the ground that they were accounted for 
    in its lost profits damages.\footnote{Casebook p. 249.}
    \item Two arguments on appeal:
    \begin{enumerate}
        \item Rotorex: it did not breach the agreement and the calculation of 
        damages was inappropriate.
        \item Delchi: it was entitled to the expenses and lost profits that 
        the trial judge denied.
    \end{enumerate}
    \item Held:
    \begin{enumerate}
        \item \emph{Breach}: Rotorex breached its contract.
        \item \emph{Damages award}: CISG awards damages for breach equal to 
        the loss, including lost profits, not to exceed the amount which the 
        party in breach foresaw (or should have foreseen). The trial court's 
        damage award was appropriate. Moreover, Delchi should have won the 
        expenses that the trial court denied. These expenses were legitimate 
        and reasonably foreseeable.
        \item On a few other issues of fact, the appellate court deferred to 
        the trial court.\footnote{Casebook p. 252--53.}
    \end{enumerate}
\end{enumerate}

\paragraph{UCC \S\S\ 2-501(1) \emph{et seq.}}

\begin{enumerate}
    \item % TODO 253, and supp
\end{enumerate}

\paragraph{\emph{CISG Arts. 61, 62, 64, 74, 75, 76}}

\begin{enumerate}
    \item % TODO 253, and supp
\end{enumerate}

\paragraph{\emph{KGM Harvesting Co. v. Fresh Network}}
~\\\\
When a buyer covers in good faith without reasonable delay, he can recover 
from the seller the difference between the cost of cover and the contract 
price. Whether the buyer can pass on the cost of cover to future buyers is 
irrelevant.

\begin{enumerate}
    \item Facts:
    \begin{enumerate}
        \item July 1989: KGM and Fresh entered into a lettuce contract.
        \item May 1991: the terms of the contract were settled. KGM would sell 
        14 loads per week to Fresh at \$0.09/pound, or \$55,440 per week.
        \item May/June 1991: the price of lettuce increased dramatically. KGM 
        refused to sell to Fresh at the contract price. Instead, it sold to 
        others at a profit of \$800,000 and \$1.1 million.
    \end{enumerate}
    \item The issue was whether Fresh was entitled to damages from KGM for 
    the cost of obtaining substitute lettuce. The jury awarded damages of 
    \$655,960.22, or the difference between the contract price and the cost of 
    substitute lettuce during May and June. Subtracting the \$233,000 Fresh 
    owed KGM, Fresh won \$422,960.22 and interest.
    \item Fresh proceeded under UCC \S\ 2-712: after the breach, it covered, 
    and recovered the difference between the cover price and the contract 
    price.
    \item On appeal, KGM argued that Fresh should not be able to recover under 
    \S\ 2-712 because it was able to pass on its costs to subsequent buyers. 
    Fresh suffered no actual loss, so damages would be a windfall.
    \item \emph{Allied Canners}: the buyer, Allied Canners, contracted to buy 
    raisins from Victor, which it planned to resell to two Japanese companies 
    at cost plus 4 percent. That year, the cost of raisins soared. Victor 
    breached. Allied sued under the 7-212 formula, under which he would have 
    won \$150,000. The Japanese companies made no claim (one let it go and the 
    other's claim expired under the statute of limitations). The court held 
    that 7-212 damages should be limited to \textbf{actual loss} when three 
    conditions apply:
    \begin{enumerate}
        \item The seller knew the buyer had a resale contract.
        \item The buyer can't show it would be liable for damages on the 
        forward contract.
        \item The seller did not act in bad faith.
    \end{enumerate}
    \item The court here held that the \emph{Allied Canners} focus on good 
    faith is inappropriate in commercial contexts. ``[C]ourts should not 
    differentiate between good and bad motives for breaching a contract in 
    assessing the measure of the nonbreaching party's 
    damages.''\footnote{Casebook p. 259.}
    \item Held: when a buyer covers in good faith without reasonable delay, 
    he can recover from the seller the difference between the cost of cover 
    and the contract price.
\end{enumerate}

\paragraph{\emph{Neri v. Retail Marine Corp.}}

\begin{enumerate}
    \item % TODO 260
\end{enumerate}

\paragraph{\emph{Teradyne, Inc. v. Teledyne Industries, Inc.}}

\begin{enumerate}
    \item % TODO 264
\end{enumerate}

\paragraph{Childres and Burgess, ``Seller's Remedies: The Primacy of UCC 
2-708(2)''}

\begin{enumerate}
    \item % TODO  264
\end{enumerate}

\paragraph{\emph{R.E. Davis Chemical Corp. v. Diasonics, Inc.}}

\begin{enumerate}
    \item % TODO 264
\end{enumerate}

\paragraph{\emph{Lazenby Garages Ltd. v. Wright}}

\begin{enumerate}
    \item % TODO 265
\end{enumerate}

\subsubsection{Mitigation; Contracts for Employment} 266

\paragraph{\emph{Rockingham County v. Luten Bridge Co.}}

\begin{enumerate}
    \item % TODO 266
\end{enumerate}

\paragraph{\emph{Madsen v. Murray \& Sons, Co.}}

\begin{enumerate}
    \item % TODO 269
\end{enumerate}

\paragraph{\emph{In Re Kellet Aircraft Corp.}}

\begin{enumerate}
    \item % TODO 269
\end{enumerate}

\paragraph{\emph{Bank One, Texas N.A. v. Taylor}}

\begin{enumerate}
    \item % TODO 270
\end{enumerate}

\paragraph{\emph{S.J. Groves \& Sons Co. v. Warner Co.}}

\begin{enumerate}
    \item % TODO 270
\end{enumerate}

\paragraph{Restatement Second \S\ 350}

\begin{enumerate}
    \item % TODO supp
\end{enumerate}

\paragraph{UCC \S\S\ 2-704(2), 2-715(2)}

\begin{enumerate}
    \item % TODO supp
\end{enumerate}

\paragraph{CISG Art. 77}

\begin{enumerate}
    \item % TODO supp
\end{enumerate}

\paragraph{UNIDROIT Arts. 7.4.7, 7.4.8}

\begin{enumerate}
    \item % TODO supp
\end{enumerate}

\paragraph{Principles of European Contract Law Arts. 9.504, 9.505}

\begin{enumerate}
    \item % TODO 272
\end{enumerate}

\paragraph{\emph{Shirley MacLaine Parker v. Twentieth Century-Fox Film Corp.}}

\begin{enumerate}
    \item % TODO 272
\end{enumerate}

\paragraph{\emph{Punkar v. King Plastic Corp.}}

\begin{enumerate}
    \item % TODO 275
\end{enumerate}

\paragraph{\emph{Mr. Eddie, Inc. v. Ginsburg}}

\begin{enumerate}
    \item % TODO 275
\end{enumerate}

\paragraph{\emph{Southern Keswick, Inc. v. Whetherholt}}

\begin{enumerate}
    \item % TODO 276
\end{enumerate}

\paragraph{Damages for Loss of Opportunity to Practice One's Profession}

\begin{enumerate}
    \item % TODO  276
\end{enumerate}

\newpage

\subsubsection{Foreseeability}

\paragraph{Reasonably Foreseeable Damages and Special Circumstances: 
\emph{Hadley v. Baxendale}}

\begin{enumerate}
    \item A shaft broke at Hadley's mill. He contracted with Baxendale to ship 
    it to the engineer to use as a model for a new shaft. Baxendale promised 
    it could be shipped the next day, but in fact the shipment was delayed for 
    several days. In the meantime, the mill was out of operation. Hadley sued 
    for lost profits.
    \item Held: Baxendale was unaware that the Mill couldn't operate without 
    the shaft, and it was not a reasonable inference.
    \item \emph{First rule}: the party in breach is liable for reasonably 
    foreseeable damages.
    \item \emph{Second rule}: if the party in breach is aware of special 
    circumstances, he is liable for extra loss those circumstances create. But 
    he is not liable for the extra loss if he was not aware of the special 
    circumstances.
\end{enumerate}

\paragraph{Commercial Breach: \emph{Victoria Laundry (Windsor) Ltd. v. Newman 
Indus. Ltd.}}

A commercial machine has only commercial uses. It's reasonably foreseeable 
that failing to deliver the machine will hinder commerce, especially when 
``the demand for laundry services at that time was 
insatiable.''\footnote{Casebook p. 284.}

\begin{enumerate}
    \item The plaintiff contracted to buy a bigger boiler for their laundry 
    business. The defendant delayed delivery by more than five months. The 
    plaintiff sued for lost profits.
    \item The court held that the defendant should have known that the boiler 
    would have been put into immediate commercial use. The plaintiff could 
    recover because the lost profits were reasonably foreseeable damages 
    resulting from breach.
    \item Plaintiffs could not recover for ``particularly lucrative 
    contracts'' it expected to procure, but they could still recover for 
    ``loss of business~.~.~.~to be reasonably expected.''\footnote{Casebook p. 
    285.}
\end{enumerate}

\paragraph{Market Fluctuations: \emph{Koufos v. C. Czarnikow, Ltd. [The Heron 
II]}}

\begin{enumerate}
    \item The plaintiff chartered the defendant's ship to deliver a load of 
    sugar. The defendant delayed shipment by nine days, during which the 
    market price of sugar dropped.
    \item Held: fluctuations in the price of sugar were reasonably 
    foreseeable, so the shipper was liable for lost profits.
    \item Lord Reid: a 25 percent change of an event occurring is reasonably 
    foreseeable and thus compensable, but a 2 percent chance is not.
\end{enumerate}

\paragraph{Commercial Machinery: \emph{Hector Martinez \& Co. v. Southern 
Pacific Transp. Co.}}
~\\\\
``We must not lose sight of the basic common law rule, enunciated in 
\emph{Hadley}, of damages for foreseeable loss.''\footnote{Casebook p. 288.}

\begin{enumerate}
    \item The plaintiff contracted with the defendant for the delivery of 
    strip mining machinery. The defendant delayed shipment by a month.
    \item Held: the plaintiff could recover for the rental value of the 
    machinery for the month it was delayed. This case is distinct from 
    \emph{Hadley} in that commercial machinery has an obvious commercial use, 
    while in \emph{Hadley} it was not clear that the shaft was essential to 
    the Mill's operation.
\end{enumerate}

\paragraph{The Scope of Hadley v. Baxendale}

\begin{enumerate}
    \item Courts usually apply \emph{Hadley} to whether a given \emph{type} of 
    loss would be a reasonably foreseeable result of breach.
    \item Courts also apply \emph{Hadley} to determine the \emph{amount} of 
    loss---e.g., the court's holding in \emph{Victoria Laundry} that the 
    plaintiff could not recover for loss of particularly lucrative contracts 
    because those contracts were not reasonably foreseeable.
    \item But in \emph{Wroth v. Tyler}, the defendant breached a contract to 
    sell a house. After the breach, property values rose dramatically and 
    unexpectedly. The court held the defendant was liable for the full 
    difference, despite the unforeseeability of the damages 
    amount.\footnote{Casebook pp. 288--89.}
\end{enumerate}

\subsubsection{Certainty}

\paragraph{\emph{Kenford Co. v. Erie County}}

\begin{enumerate}
    \item Kenford and Dome contracted to build a stadium with Erie County. 
    The parties were unable to negotiate a lease during the agreed one-year 
    window, nor did construction ever begin. Kenford and Dome brought suit.
    \item The trial court awarded the plaintiffs a multimillion-dollar jury 
    verdict. The appellate court reversed lost profits damages and other 
    expenses. The New York Court of Appeals heard only the issue of lost 
    profits.
    \item The New York standard for lost profits damages required (1) 
    certainty that the breach caused the damages and (2) ``the alleged loss 
    must be capable of proof with reasonable certainty.''\footnote{Casebook p. 
    293.} Damages must also have been within the contemplation of the parties 
    when they made the contract.
    \item The plaintiff's damages were not within the parties' contemplation 
    and their projections, while sophisticated, are too remote (they rely only 
    on the Astrodome). Reversed.
    \item The court rejected the rational basis test from \emph{Perma 
    Research}, which allowed lost profits damages for new ventures if there 
    was a ``rational basis'' for their calculation.\footnote{Casebook pp. 
    294--95.}
\end{enumerate}

\paragraph{\emph{Ashland Management Inc. v. Janien}}

\begin{enumerate}
    \item Ashland contracted with Janien to produce Eta, a sophisticated 
    financial modeling tool. Ashland breached.
    \item The court held that Janien could recover lost profits because they 
    were not speculative---rather, the parties both believed the claims 
    represented a fair forecast of future earnings, they were not entering an 
    unfamiliar business, they had a ``ready reservoir'' of customers, and Eta 
    had been extensively tested.
\end{enumerate}

\paragraph{The New-Business Rule}

\begin{enumerate}
    \item Courts are wary of awarding speculative damages to new ventures. 
    ``~.~.~.~the plaintiff must lay a basis for a reasonable estimate of the 
    extent of his harm, measured in money.''\footnote{Casebook pp. 296--97.}
\end{enumerate}

\paragraph{Consistency of Past Performance: \emph{Rombola v. Cosindas}}

\begin{enumerate}
    \item Rombola agreed to maintain and race Cosinda's horse in exchange for 
    25\% of the profits. Before a race, Cosindas took possession of the horse, 
    depriving Rombola of the chance to win money.
    \item The horse and Rombola had proved their ability to consistently win 
    money. Rombola could recover consistent with their earlier performance.
\end{enumerate}

\paragraph{Statistics of Similar Businesses: \emph{Contemporary Mission, Inc. 
v. Famous Music Corp.}}
% TODO: reconcile with kenfield?
\begin{enumerate}
    \item Famous Music contracted with Contemporary Mission to produce records 
    from the master tape of the rock opera \emph{Virgin}. Famous Music failed 
    to promote the records to Contemporary's satisfaction. The court held that 
    statistical evidence of similarly-performing songs and albums in the past 
    was admissible evidence to establish Contemporary's lost profits damages.
\end{enumerate}

\paragraph{UCC \S\ 1-106(1)}

\begin{enumerate}
    \item % TODO supp
\end{enumerate}

\paragraph{Restatement Second \S\ 352}

\begin{enumerate}
    \item % TODO supp
\end{enumerate}

\paragraph{UNIDROIT Art. 7.3}

\begin{enumerate}
    \item % TODO supp
\end{enumerate}

\paragraph{Uncertainty}

\begin{enumerate}
    \item \emph{All or nothing rule}: one premise of \emph{Kenfield} is that 
    there is some level of certainty above which the plaintiff can fully 
    recover, and below which the plaintiff can recover nothing. This is 
    wrong---the plaintiff should be compensated for the value of the 
    \emph{chance} to earn a profit, even if it is not certain that a profit 
    would result.
    \item Fuller and Eisenberg propose a formula for calculating damages based 
    on probability based on the Capital Asset Pricing Model: damages should be 
    awarded in proportion their likelihood.  So if a venture has as 10\% 
    chance of \$20 million and a 90\% chance of \$10 million, the award should 
    be (0.10 x \$20 million) + (0.90 x \$10 million), or \$11 million.
\end{enumerate}

\subsubsection{Liquidated Damages}

\paragraph{\emph{Wasserman's Inc. v. Middletown}}

\begin{enumerate}
    \item % TODO 308
\end{enumerate}

\paragraph{Note on Liquidated Damages}

\begin{enumerate}
    \item % TODO 318
\end{enumerate}

\paragraph{Provisions that Limit Damages}

\begin{enumerate}
    \item % TODO 322
\end{enumerate}

\subsection{Specific Performance} 324

\paragraph{Law \& Equity}

\begin{enumerate}
    \item Common law distinction between law and equity:
    \begin{enumerate}
        \item \emph{Law}: broadly, the principles for administering justice. 
        Narrowly, the principles applied by common law courts.
        \item \emph{Equity}: broadly, fairness. Narrowly, the principles 
        administered by equity courts. They did not have general jurisdiction. 
        Rather, they existed to correct defects in common law court decisions.
    \end{enumerate}
    \item Regarding contract law, the defect of common law courts was that  
    they could not order specific performance. They could only award money 
    damages. But equity courts could order specific performance (though 
    sometimes they declined---e.g., if an opera singer breaches a contract, 
    you don't want to force her to sing).\footnote{Casebook p. 325.}
    \item Modern American courts have erased the law-equity structure, but 
    judges and lawyers often preserve the mental distinction.
\end{enumerate}

\paragraph{\emph{London Bucket Co. v. Stewart}}

\begin{enumerate}
    \item London Bucket Company contracted with a motel to install a heating 
    system. The motel brought suit for breach, requesting specific performance 
    ``before the fall of cold weather.''\footnote{Casebook p. 326.}
    \item The trial court ordered specific performance. The appellate court 
    reversed, holding that money damages would have adequately compensated the 
    plaintiff for its injury.
    \item ``It is the general rule that contracts for building construction 
    will not be specifically enforced because ordinarily damages are an 
    adequate remedy and, in part, because of the incapacity of the court to 
    superintend the performance.''\footnote{Casebook p. 327.}
\end{enumerate}

\paragraph{Restatement Second \S\ 359}

\begin{enumerate}
    \item % TODO supp
\end{enumerate}

\paragraph{Restatement Second \S\ 360}

\begin{enumerate}
    \item % TODO supp
\end{enumerate}

\paragraph{UNIDROIT Arts. 7.2.1, 7.2.2, 7.2.3}

\begin{enumerate}
    \item % TODO supp
\end{enumerate}

\paragraph{Principles of European Contract Law Arts. 9.101, 9.102, 9.103}

\begin{enumerate}
    \item % TODO supp
\end{enumerate}

\paragraph{\emph{Walgreen Co. v. Sara Creek Property Co.}}

\begin{enumerate}
    \item Walgreen had a lease with Sara Creek to operate a pharmacy in a 
    mall, with the promise that Walgreen would be the only pharmacy. Sara 
    Creek wanted to lease a unit in the mall to Phar-Mor, which planned to 
    operate a pharmacy equal in size to Walgreen. Walgreen sought an 
    injunction.
    \item Sara Creek argued that Walgreen's damages could be easily 
    calculated. Walgreen argued that damages would be difficult to compute 
    because they included intangibles like goodwill.
    \item The trial court awarded an injunction.
    \item The choice of whether to award damages or an injunction rests on a 
    cost-benefit analysis (for which the trial judge is responsible):
    \begin{enumerate}
        \item \emph{Benefits of an injunction}:
        \begin{enumerate}
            \item It shifts the burden of determining the costs of the 
            defendant's conduct from the court to the parties.
            \item ``~.~.~.~prices and costs are more accurately determined by 
            the market than by government.''\footnote{Casebook p. 329.}
        \end{enumerate}
        \item \emph{Costs of an injunction}:
        \begin{enumerate}
            \item It requires continuous court supervision.
            \item It creates a ``bilateral monopoly'' in which two parties can 
            only deal with each other. % FIXME: continue here, p. 330 top
        \end{enumerate}
        \item \emph{Benefits of damages}:
        \item \emph{Costs of damages}:
    \end{enumerate}
\end{enumerate}

\paragraph{\emph{Stokes v. Moore}}

\begin{enumerate}
    \item % TODO 333
\end{enumerate}

\paragraph{\emph{Van Wagner Advertising Corp. v. S \& M Enterprises}}

\begin{enumerate}
    \item % TODO 333-34
\end{enumerate}

\paragraph{UCC \S\S\ 2-709, 2-716}

\begin{enumerate}
    \item % TODO supp
\end{enumerate}

\paragraph{CISG Arts. 46, 62}

\begin{enumerate}
    \item % TODO supp
\end{enumerate}

% \subsection{The Reliance and Restitution Measures} 341
% 
% \subsubsection{Reliance Damages in a Bargain Context} 341
% 
% \subsubsection{The Restitution Measure} 348
% 
