\section{Consideration}

\begin{enumerate}
    \item Consideration is the thing that locks up a contractual 
    relationship---``[s]omething (such as an act, a forbearance, or a return 
    promise) bargained for and received by a promisor from a 
    promiee~.~.~.~''\footnote{Black's Law.}
    \item Why do we require consideration?
    \item Can you sell a bottle of Gatorade for \$10,000? Yes---courts are 
    hesitant to interfere with private transactions (with a few exceptions).
    \item Spectrum of ways a contract can be established:
    \begin{enumerate}
        \item \textbf{Objective}: forms, labels, seals, magic words.
        \item \textbf{Subjective}: intent---e.g., the intent to make a 
        promise.
    \end{enumerate}
    \item Consideration traditionally has two elements:
    \begin{enumerate}
        \item Mutual assent.
        \item Recognized consideration.
    \end{enumerate}
\end{enumerate}

\subsection{Donative Promises, Form, and Reliance}

\subsubsection{Simple Donative Promises}

Made for affective reasons, usually informal, and not demonstrably relied 
upon.\footnote{Casebook p. 6.}

\paragraph{\emph{Dougherty v. Salt}}

\begin{enumerate}
    \item A boy's aunt gave him a promissory note for \$3,000, payable at her 
    death or before. The aunt died. The boy's guardian brought an action 
    against the aunt's estate to recover the promised money.
    \item The jury found for the plaintiff. The trial judge set aside the 
    verdict and dismissed the complaint. The appellate court reversed and 
    reinstated because it held that the promissory note was valid 
    consideration.
    \item The New York Court of Appeals (Cardozo) reversed, holding that the 
    note was ``the voluntary and unenforceable promise of an executory 
    gift.''\footnote{Casebook p. 7.} The note was inadequate consideration in 
    this context because paper does not make a donative promise enforceable.
\end{enumerate}

\paragraph{Restatement Second \S\ 1: Definition}

\begin{enumerate}
    \item ``A contract is a promise or a set of promises for the breach of 
    which the law gives a remedy, or the performance of which the law in some 
    way recognizes as a duty.''
\end{enumerate}

\paragraph{Restatement Second \S\ 17: Requirement of a Bargain}

\begin{enumerate}
    \item ``~.~.~.~a contract requires a bargain in which there is a 
    manifestation of mutual assent to the exchange and a consideration.''
    \item But, there are many exceptions.
\end{enumerate}

\paragraph{Restatement Second \S\ 71: Requirement of Exchange; Types of 
Exchange}

\begin{itemize}
    \item (1) Consideration must be bargained for. 
    \item (2) A performance or return promise is ``bargained for'' if sought and given in exchange.
    \item (3) Performance can be an act, forbearance, or creation, 
    modification, or destruction of a legal relation.
    \item (4) Performance may be given to another person.
\end{itemize}

\paragraph{Restatement Second \S\ 79: Adequacy of Consideration; Mutuality 
of Obligation}

\begin{enumerate}
    \item If the consideration requirement is met, there are no requirements 
    of gain/loss, equivalence in value, or ``mutuality of obligation.''
\end{enumerate}

\paragraph{On the Restatement Second}

\begin{enumerate}
    \item Samuel Williston, Reporter for the Restatement First: contract law 
    is ``a set of axioms that were deemed to be self-evident, together with a 
    set of subsidiary rules that were purportedly deduced from the 
    axioms.''\footnote{Casebook p. 8.}
    \item Arthur Corbin, Consultant to the Restatement Second: father of 
    ``modern contract law.''\footnote{Casebook p. 8.}
\end{enumerate}

\paragraph{Consideration}

\begin{enumerate}
    \item Consideration is giving something and getting something in return. 
    Both parties get something they want from the other.
    \item Mere acceptance is not consideration, but at some point acceptance 
    becomes an act that qualifies as consideration.
    \begin{enumerate}
        \item The \textbf{Tramp Example}: I'll call the store and have two 
        suits ready for you. You go to the store and find out I just kidding. 
        There's no act there. But what if I make you drive to a store in San 
        Francisco? Is that an act that constitutes consideration? I got what I 
        wanted (for you to go to the store) and you thought you were getting 
        what you wanted (new clothes).
    \end{enumerate}
    \item Courts generally do not consider the adequacy of consideration.
    \item Two conceptions: broad and narrow.
    \item \emph{Broad}: ``consideration'' refers collectively to the things 
    that make contracts legally enforceable---e.g., bargain or 
    reliance.\footnote{Casebook p. 8.}
    \item \emph{Narrow}: ``consideration'' is the same thing as ``bargain.'' 
    The Restatement Second adopts this approach, known as the \textbf{bargain 
    theory of consideration}.
    \item The bargain theory of consideration creates two kinds of distortion:
    \begin{enumerate}
        \item \emph{Terminological}: many other elements besides bargain can 
        make a contract enforceable. Therefore, ``under the terminology of the 
        Restatement Second, a promise needs consideration to be enforceable 
        unless it does not need consideration to be 
        enforceable.''\footnote{Casebook p. 9.}
        \item \emph{Substantive}: the bargain theory presumes that all 
        nonbargain promises are unenforceable. It does not allow the law to 
        develop other means of making promises enforceable.
    \end{enumerate}
\end{enumerate}

\paragraph{Gifts}

\begin{enumerate}
    \item The common law distinguishes between gifts and promises to make 
    gifts. A promise to make a gift is not enforceable. \emph{Dougherty}.
    \item A \textbf{deed of gift} (or \textbf{inter vivos document of 
    transfer}) transfers ownership via a written instrument.\footnote{Casebook 
    pp. 9--10.}
    \item Another way to make a gift is for the owner to declare herself a 
    \textbf{trustee} of her property for the benefit of another. The trustee 
    retains legal title but the beneficiary receives beneficial 
    ownership.\footnote{Casebook p. 10.}
    \item Once made, a gift cannot be taken back---but before it's made, we're 
    uncertain.
\end{enumerate}

\paragraph{Donative Promises}

\begin{enumerate}
    \item Why make a donative promise, rather than wait and give the gift 
    later?
    \begin{enumerate}
        \item Instincts vary. The promisor might think she has better judgment 
        right now than she will in the future.
        \item The promisor wants to derive satisfaction from the promisee's 
        gratitude.
        \item \textbf{Beneficial reliance}: the promisee can rely on the 
        promise to her benefit---e.g., knowing that her Aunt will pay for 
        college, a niece decides to finish high school, rather than look for a 
        paying job.
    \end{enumerate}
    \item The ``basic fault line'' in classical contract law was between 
    bargain promises (enforceable) and gratuitous promises (unenforceable).
    \item Should contract law be based on the moral belief that breaking 
    promises is unethical, or should it promote utilitarian goals like 
    ``compensating injured promisees and increasing social 
    wealth''?\footnote{Casebook p. 11.}
    \item Lon Fuller introduced \textbf{substantive} and \textbf{process} 
    bases for enforcing promises.
    \begin{enumerate}
        \item The two process bases are evidentiary (making sure a promise 
        has actually been made) and cautionary (preventing inconsiderate 
        action by the promisor).
        \item Simple donative promises are problematic on both bases. They 
        raise problems of proof (evidence may be uncertain) and 
        deliberativeness (the promisor is likely to be emotionally involved 
        with the promisee).
        \item Substantive reasons for enforcing donative promises include 
        compensating the promisee's disappointment (a form of injury), move 
        assets from the wealthy to the less wealthy, and increase the 
        likelihood of beneficial reliance. But these substantive bases are 
        open to question.
    \end{enumerate}
\end{enumerate}

\paragraph{Conditional Donative Promises}

\begin{enumerate}
    \item In a bargain promise, the condition is the \textbf{price} of the 
    promise---e.g., I'll give you \$20 if you mow my lawn.
    \item In a donative promise, the condition is the \textbf{means} to make 
    the gift---e.g., I'll buy you a car if you pick one that costs less than 
    \$15,000. 
    \item We have to rely on a ``reasonable interpretation'' to determine 
    whether a condition is a means or a price.\footnote{Casebook p. 13.}
\end{enumerate}

\subsubsection{The Element of Form}

\begin{enumerate}
    \item We often look for an official indication of a valid 
    agreement, like wax seals, but the requirement has mostly evaporated.
\end{enumerate}

\paragraph{Von Mehren, ``Civil-Law Analogues to Consideration: An Exercise 
in Comparative Analysis''}

\begin{enumerate}
    \item French, German, and common law systems recognize four policies that 
    lead them to treat a transaction as unenforceable:
    \begin{enumerate}
        \item \textbf{Evidentiary}: Evidentiary security---protecting against 
        manufactured evidence and dealing with insufficient proof.
        \item \textbf{Cautionary}: Safeguarding the individual against his own 
        rashness.
        \item \textbf{Channeling}: Signaling to make sure that the promisor 
        knows the promise is enforceable.
        \item \textbf{Deterrent}: Unwillingness to enforce contracts of 
        ``suspect or marginal value.''\footnote{Casebook pp. 13--14.}
    \end{enumerate}
\end{enumerate}

\paragraph{Channeling Function of Contract-Law Rules}

\begin{enumerate}
    \item Von Mehren's channeling policy assumes actors know contract 
    law---but must don't. So, this policy only applies to rules that actors 
    are likely to know. 
\end{enumerate}

\paragraph{\emph{Schnell v. Nell}}

\begin{enumerate}
    \item Zacharias Schnell's wife Theresa agreed to pay \$200 each to Nell and 
    the two Lorenzes on her death. After she died, Schnell agreed with the 
    three claimants to pay out the promised money over three years. In 
    exchange, the three claimants gave him one cent. Now, Schnell wants to 
    back out of the agreement.
    \item Schnell argued that there was no consideration because his wife did 
    not own any property and therefore could not make such a promise. The 
    lower court sustained a demurrer against Schnell.
    \item The question was whether the contract ``express[ed] a consideration 
    sufficient to give it legal obligation, as against Zacharias 
    Schnell.''\footnote{Casebook p. 15.}
    \item The court considered three bases for consideration:
    \begin{enumerate}
        \item The claimants promised to pay Schnell one cent.---Although 
        courts generally do not evaluaate the adequacy of consideration, this 
        exchange---one cent for \$200---is too unconscionable to 
        sustain. (``Even in traditional times, courts were not willing agents 
        of absurdity or oppression.''---Berring.)
        \item Mr. Schnell bore love and affection to his late wife, and ``she 
        had done her part, as his wife, in the acquisition of the 
        property.''\footnote{Casebook p. 15.}---Invalid because these are past 
        considerations and have no bearing on Schnell's promise to the three 
        claimants.
        \item Mrs. Schnell wished to give the money to the three 
        claimants.---Invalid because she had no property of her own.
    \end{enumerate}
\end{enumerate}

\paragraph{Form}

\begin{enumerate}
    \item % TODO 21-23
    ~\\\\\\\\\\\\
\end{enumerate}

\subsubsection{The Element of Reliance}

\begin{enumerate}
    \item A moral obligation to keep a promise is generally insufficient to 
    establish a legal obligation.
    \item Lawmakers should also consider social policy and experiential 
    concerns (e.g., problems of proof).
    \item Injury from reliance is a moral and policy reason for enforcing 
    promises.
\end{enumerate}

\paragraph{\emph{Kirksey v. Kirksey}}

These facts are analogous to the tramp example. Was this a mere gift or was 
there consideration? The defendant got nothing in return for his offer, but 
the plaintiff \emph{did} do what he wanted---so you can start the 
consideration analysis.

\begin{enumerate}
    \item ``Dear Sister Antillico~.~.~.~''
    \item The defendant, plaintiff's brother in law, invited the plaintiff and 
    her family to come live at his house, offering her living space and land 
    to tend, but asking nothing in return. She abandoned her property in the 
    country to come live with him. He let her live comfortably in the house 
    for two years. Then, he put her in a house, ``not comfortable,'' in the 
    woods, and then made her leave.
    \item The lower court found for the plaintiff for \$200.
    \item The judge writing the opinion for the Supreme Court of Alabama 
    believed that the ``loss and inconvenience'' she suffered justified the 
    award, but the majority believed the offer was a ``mere gratuity.'' 
    Reversed.\footnote{Casebook pp. 24--25.}
\end{enumerate}

\paragraph{Restatement Second \S\ 90}

\begin{enumerate}
    \item ``A promise which the promisor should reasonably expect to induce 
    action or forbearance on the part of the promisee or a third person and 
    which does induce such action or forbearance is binding if injustice can 
    be avoided only by enforcement of the promise. The remedy granted for 
    breach may be limited as justice requires.''
\end{enumerate}

\paragraph{Estoppel in Pais and Promissory Estoppel}

\begin{enumerate}
    \item Under classical contract law, only a bargain was consideration. 
    Donative promises were not enforceable, even if the promisee relied on the 
    promise. \emph{Kirksey}.\footnote{Casebook p. 25.}
    \item \emph{Ricketts}: a man promised his granddaughter \$2,000 so she 
    would not have to work anymore. She quit her job. The man died. The 
    granddaughter successfully brought suit. The court avoided the question of 
    whether reliance constituted consideration. Instead, reliance estopped the 
    man's estate from pleading a lack of consideration. This case lies 
    somewhere between estoppel in pais and promissory estoppel.
    \item \textbf{Estoppel in pais}: if A made a statement and B foreseeably 
    relied on that statement, A is estopped from denying that statement's 
    truth. Based on a \emph{statement of fact}.
    \item \textbf{Promissory estoppel}: a promise is binding if the promisee 
    reasonably relied on it. Codified in Restatement Second \S\ 90. Based on a 
    \emph{promise}.
    \item Both are instances of a broader \textbf{``reliance principle''}: 
    ``When one person, A, uses words or actions that he knows or should know 
    would induce another, B, to reasonably believe that A is committed to take 
    a certain course of action, and A knows or should know that B will incur 
    costs if A doesn't take action, A should take steps to ensure that if he 
    doesn't take the action, B will not suffer a loss.''\footnote{Casebook p. 
    27.}
\end{enumerate}

\paragraph{Promissory Estoppel Problem}

\begin{enumerate}
    \item An uncle promises his nephew \$50,000 to buy a business. The nephew 
    buys the business with his own money. The uncle dies before depositing the 
    promised money. The executor of the uncle's estate refuses to recognize 
    the nephew's claim.
    \item Can the nephew base an action on:
    \begin{enumerate}
        \item Deceit?
        \item Estoppel in pais?
        \item A promise made enforceable by unbargained-for reliance?
        \item A promise supported by a bargained-for consideration?
    \end{enumerate}
\end{enumerate}

\paragraph{Promissory Estoppel: \emph{Feinberg v. Pfeiffer Co.}}

\begin{enumerate}
    \item Facts:
    \begin{enumerate}
        \item 1910: Feinberg began working for Pfeiffer.
        \item 1947: Still working for Pfeiffer, now in more senior positions.
        \item December 27, 1947: the company resolved to raise her salary to 
        \$400/month and offer a lifetime retirement pay of \$200 per month 
        (but with the hope that she would continue working).
        \item June 30, 1949: Feinberg retired and began collecting her monthly 
        payments.
        \item Spring 1956: the company's new ownership refused to continue 
        sending the monthly retirement checks.
        \item April 1, 1956: the company sent its final check, for \$100. 
        Feinberg refused it and brought suit to recover her full monthly 
        retirement checks.
    \end{enumerate}
    \item The trial judge held for Feinberg and awarded \$5,100.
    \item Pfeiffer raised two peripheral arguments on appeal:
    \begin{enumerate}
        \item The court erroneously allowed evidence that at the time of 
        trial, Feinberg could not seek employment because she was suffering 
        from cancer.
        \begin{enumerate}
            \item Held: this was not the basis for the trial court's decision.
        \end{enumerate}
        \item There was insufficient evidence to support the court's finding 
        that (1) Feinberg would not have quit unless she relied on the monthly 
        retirement checks and (2) Feinberg did in fact rely on the checks 
        after retiring.
        \begin{enumerate}
            \item Held: the lower court properly credited Feinberg's testimony 
            that she relied on the checks.
        \end{enumerate}
    \end{enumerate}
    \item The key issue was whether Feinberg had a right to recover her 
    retirement checks on the basis of a legally binding contractual 
    obligation.
    \begin{enumerate}
        \item Pfeiffer: the company's decision to give the retirement checks 
        was a mere promise to make a gift. Donative promises are not 
        enforceable unless supported by consideration. The only possible 
        consideration here was a reward for past service, which is not valid. 
        Further, Feinberg's position was at-will.
        \item Feinberg: agreed that a promise based on past services would not 
        constitute valid consideration, but (1) she continued to work for a 
        year and a half after the company's resolution and (2) she based her 
        decision to retire on reliance on the monthly check.
        \item Court: 
        \begin{enumerate}
            \item Feinberg's first contention, that she continued to work 
            after the resolution, does not constitute valid consideration 
            because that was not a condition of the agreement.
            \item But Feinberg's second argument, that she relied on the 
            promise, was valid. The promise met the Restatement Second \S\ 90 
            requirements for promissory estoppel (in fact, one of the 
            Restatement illustrations involves retirement annuities).
            \item Affirmed.
        \end{enumerate}
    \end{enumerate}
\end{enumerate}

\paragraph{Distinguishing \emph{Feinberg}: \emph{Hayes v. Plantations 
Steel Co.}}

\begin{enumerate}
    \item Plantations Steel told Hayes it would ``take care'' of him upon his 
    retirement, but there was no mention of an actual sum. It gave a first 
    payment ``as a token of appreciation'' and implied that the payments would 
    continue annually.
    \item The company changes ownership. Later owners discontinued the 
    payments. Hayes brought suit.
    \item The key distinction from \emph{Feinberg} is that the company's offer 
    of annuities was not an inducement to retire, because Hayes had given 
    notice of retiring several months earlier. He did not rely on the 
    company's promise because nothing about the promise changed his behavior. 
    Therefore, the promise did not meet the requirements for promissory 
    estoppel.
    \item Would it matter if the check had come annually for 10 years? 30 
    years?
    \begin{enumerate}
        \item He chose to retire before the offer.
        \item He expressed the uncertainty by showing up each year and 
        wondering if the check was there for him.
    \end{enumerate}
\end{enumerate}

\paragraph{Remedies and Consideration}

\begin{enumerate}
    \item To what extent should a promise be enforced?
    \item Two types of \textbf{injury}:\footnote{Casebook p. 35.}
    \begin{enumerate}
        \item \textbf{Reliance interest}: the promisee is worse off than he 
        would have been if the promise had not been made. \textbf{Reliance 
        damages} restore a promisee to the position he would have been in had 
        the promise not been made. % TODO example
        \item \textbf{Expectation interest}: the promisee is worse off than he 
        would have been if the promise had been performed. \textbf{Expectation 
        damages} restore the promisee to the position he would have been in 
        had the promise been performed. % TODO example
    \end{enumerate}
    \item Two types of \textbf{reliance damages}:
    \begin{enumerate}
        \item \textbf{Out-of-pocket costs}: costs incurred by promisee prior 
        to the breach. % TODO example
        \item \textbf{Opportunity costs}: the surplus the promisee would have 
        enjoyed if he had taken the opportunity that the promise led him to 
        forgo.
    \end{enumerate}
    \item To what extent should a relied-upon donative promise be enforced? 
    Should it be the extent of the reliance (the \textbf{reliance measure}) or 
    the full expectation of the promise (the \textbf{expectation measure})? See 
    casebook pp. 37--38 % TODO.
    \item The expectation measure can be useful where reliance damages are 
    difficult to establish but obviously significant.\footnote{Casebook p. 39.}
\end{enumerate}

\paragraph{\emph{Goldstick v. ICM Realty}}

\begin{enumerate}
    \item The expectation measure is valuable because it's simple, and it 
    often covers opportunity costs that the reliance measure 
    misses.\footnote{Casebook p. 40.}
\end{enumerate}

\paragraph{Reliance Damages vs. Expectation Damages: \emph{D \& G Stout, 
Inc. v. Bacardi Imports, Inc.}}

Lesson 1: get it in writing. Lesson 2: create consideration in the agreement, 
e.g., exchange money, in order to guarantee enforceability of the promise.

\begin{enumerate}
    \item D \& G, operating as General Liquors, faced the dilemma of whether 
    to sell out or scale down. Bacardi promised it would continue to deal with 
    General as its Northern Indiana distributor. Relying on this promise, 
    General turned down the selling price it had been offered from National 
    Wine \& Spirits. A week later, Bacardi withdrew its account, forcing 
    General to sell at a much lower price--\$550,000 lower than National's 
    original offer.
    \item Can General recover the difference in price from Bacardi? Trial 
    court: no. Bacardi's promise was not one on which it should reasonably 
    expected General to rely.\footnote{Casebook p. 42.}
    \item Indiana followed Restatement Second \S\ 90 on promissory estoppel.
    \item Appellate court:
    \begin{enumerate}
        \item In at-will employment relationships in Indiana, employees cannot 
        sue for lost wages, but they \emph{can} sue for moving expenses 
        incurred on the basis of a promise.\footnote{Casebook p. 42.}
        \item This is the distinction between reliance damages (moving 
        expenses---\emph{Eby}) and expectation damages (lost anticipated 
        wages---\emph{Ewing}).
        \item Was General's loss the result of lost expectations of future 
        profit (expectation damages) or an opportunity foregone in reliance on 
        the promise (reliance damages)?
        \item In its negotiations, General relied on Bacardi's promise. The 
        devaluation of the company as a result of the broken promise was a 
        reliance injury, analogous to moving expenses.
        \item Reversed.
    \end{enumerate}
\end{enumerate}

\paragraph{Expectation Damages: \emph{Walters v. Marathon Oil Co.}}

% TODO takeaway

\begin{enumerate}
    \item Mr. and Mrs. Walters hoped to open a gas station. Relying on 
    promises from Marathon Oil, the Walters bought and renovated and abandoned 
    service station. Before signing the final agreement, Marathon put a 
    moratorium on new applications for dealerships. The Walters brought suit.
    \item The Walters argued that they should be able to recover lost 
    potential profits, totaling \$22,200 (370,000 gallons at six cents per 
    gallon).
    \item Marathon argued that the Walters should only be able to recover for 
    their reliance injury, i.e., the difference between their expenditures on 
    the site (purchase and renovations) and its present value---and actually, 
    real estate values increased, so they really made a small profit.
    \item The trial court sided with the Walters and the appellate court 
    agreed, holding that the Walters had forgone the opportunity to make their 
    investment elsewhere.
    \item ``Since promissory estoppel is an equitable matter, the trial court 
    has broad power in its choice of a remedy~.~.~.~''\footnote{Casebook p. 
    45.}
\end{enumerate}

\subsection{The Bargain Principle and Its Limits}

\begin{enumerate}
    \item The ``two major props of the bargain principle'' are 
    \textbf{fairness} and \textbf{efficiency}.\footnote{Casebook p. 59. See 
    also Eisenberg, ``The Bargain Principle and Its Limits'' below.}
\end{enumerate}

\subsubsection{The Bargain Principle}

\paragraph{\emph{Westlake v. Adams, C.P.}}

\begin{enumerate}
    \item ``It is an elementary principle, that the law will not enter into an 
    inquiry as to the adequacy of consideration.''\footnote{Casebook p. 47.}
\end{enumerate}
 
\paragraph{Consideration and Benefit: \emph{Hamer v. Sidway}}

\enquote{\enquote{Consideration} means not so much that one party is profiting 
as that the other abandons some legal right in the present, or limits his 
legal freedom of action in the future, as an inducement for the promise of the 
first.}\footnote{Casebook p. 49.}

\begin{enumerate}
    \item William E. Story, Sr. promised his nephew, William E. Story, 2d, 
    \$5,000 if he would avoid drinking, gambling, etc. until he was 21. When 
    Story 2d reached 21, he and his uncle agreed that the uncle would keep 
    the money until Story 2d was mature enough to manage it, at which point he 
    would receive the \$5,000 plus interest. Story Sr. died without 
    transferring the money. Story 2d brought suit against the estate's 
    executor to recover.
    \item The executor argued that Story 2d, the promisee, was not harmed, but 
    benefited, and that Story Sr., the promisor, was not benefited. He argued 
    that ``unless the promisor was benefited, the contract was without 
    consideration.''\footnote{Casebook p. 49.}
    \item The court here disagreed, holding that consideration exists when 
    ``something is promised, done, forborne, or suffered by the party to whom 
    the promise is made as consideration for the promise made to 
    him.''\footnote{Casebook p. 49.}
\end{enumerate}

\paragraph{Statute of Frauds}

\begin{enumerate}
    \item What promises must be in writing to be enforceable? See section on 
    statute of frauds below.\footnote{Also see casebook p. 49.}
\end{enumerate}

\paragraph{Defining Detriment: \emph{Davies v. Martel Laboratory Services, 
Inc.}}

``Detriment'' can include giving up freedom of action.

\begin{enumerate}
    \item Davies was an at-will employee of Martel. Martel offered her a 
    contract in which she would become a permanent vice president if she 
    earned an MBA, and Martel offered to pay for half of the degree. Davies 
    enrolled in an MBA program. Martel fired her a year later.
    \item Martel argued that there was no consideration because the pursuit of 
    an MBA was a benefit to Davies, not a disadvantage or detriment.
    \item The trial court granted summary judgment for Martel.
    \item The appellate court reversed, holding that Martel's definition of 
    ``detriment'' was inaccurate. Legal detriment also means giving up the 
    privilege to do something.
\end{enumerate}

\paragraph{Bad Deals: \emph{Hancock Bank \& Trust Co. v. Shell Oil Co.}}

Bad deals are binding.

\begin{enumerate}
    \item Hancock bought land after making a deal with Shell that Shell would 
    lease the land for fifteen years, with the option of terminating the lease 
    with 90 days' notice.
    \item The bank then tried to get out of the agreement, arguing that the 90 
    days' notice provision in a fifteen-year lease is ``so lacking in 
    mutuality as to be void against public policy.''\footnote{Casebook p. 51.} 
    The court disagreed, holding that if there is valid consideration, parties 
    cannot escape a contract simply because they realized they got a bad deal.
\end{enumerate}

\paragraph{Restatement Second \S\ 71}

\begin{enumerate}
    \item % TODO 
\end{enumerate}

\paragraph{Restatement Second \S\ 72}

\begin{enumerate}
    \item % TODO 
\end{enumerate}

\paragraph{Restatement Second \S\ 79}

\begin{enumerate}
    \item % TODO 
\end{enumerate}

\paragraph{Selling Money: \emph{Batsakis v. Demotsis}}

It's possible to sell money. Loans are enforceable, even if the amount to be 
repaid is orders of magnitude larger than the amount loaned.

\begin{enumerate}
    \item During World War II, Batsakis loaned the equivalent \$25 to 
    Demotsis. Demotsis agreed to pay him \$2,000 plus interest as soon as she 
    could access her American accounts. Demotsis signed a letter indicating 
    that she received \$2,000 and would repay it with interest. Batsakis sued 
    to recover the \$2,000 plus interest.
    \item The trial court awarded \$750 plus interest.
    \item Demotsis argued that the contract did not have valid consideration 
    because (she appears to argue---but it's unclear) she only received \$25. 
    The appellate court disagreed, holding that there was valid consideration. 
    Batsakis essentially sold \$25 to Demotsis at the price of \$2,000 plus 
    interest. ``Defendant got exactly what she contracted for according to her 
    own testimony.''\footnote{Casebook p. 54.} The court awarded \$2,000 plus 
    interest.
\end{enumerate}

\paragraph{Sweet-Escott, Greece---A Political and Economic Survey, 
1939--1953}

\begin{enumerate}
    \item Greece was in crisis during World War II. This was the context for 
    \emph{Batsakis}. Batsakis perhaps took advantage of Demotsis under the 
    circumstances, weakening his argument that consideration was valid.
\end{enumerate}

\paragraph{Consideration and Remedies}

\begin{enumerate}
    \item To what extent should a particular type of promise be enforced?
    \item This problem doesn't matter when the three types of 
    damages---expectation, reliance, and restitution---produce identical 
    results. For instance, if a homeowner fails to pay a plumber \$60 for an 
    hour of work, the plumber's expectation damages were \$60, and his 
    reliance damages were also likely \$60 because he could have been working 
    for the same rate on another job.
    \item But sometimes the damages differ. In \emph{Batsakis}, for instance, 
    the defendant argued that the value of what she received was far less than 
    the value she promised. She was willing to repay the \$25 she received 
    (restitution damages) but not the value of what she had promised 
    (expectation damages).
    \item In \emph{Batsakis}, both parties agreed that the contract was 
    enforceable, but they differed on the extent. So when the court said it 
    would not review the adequacy of consideration, it was saying that the 
    normal remedy for breach of contract is expectation damages, and it 
    doesn't matter whether the value of one promised performance exceeded the 
    value of the other.\footnote{Casebook p. 55.}
\end{enumerate}
 
\paragraph{Restatement Second \S\ 175: When Duress by Threat Makes a 
Contract Voidable}

\begin{enumerate}
    \item A contract is void when:
    \begin{enumerate}
        \item When the victim had no choice but to assent because of a threat.
        \item When a third party creates duress, unless the other party to the 
        transaction in good faith did not know of the duress and gives value 
        or relies on the transaction.
    \end{enumerate}
\end{enumerate}

\paragraph{Restatement Second \S\ 176: When a Threat is Improper}

\begin{enumerate}
    \item A threat is improper when it is:
    \begin{enumerate}
        \item A crime or tort.
        \item Criminal prosecution.
        \item Bad faith civil process.
        \item Breach of duty of good faith and fair dealing.
    \end{enumerate}
    \item % TODO illustration 16, p. 55 -- why is this improper? Why can't the 
    % water company decide to withhold water?
\end{enumerate}

% \paragraph{Unidroit Principles of International Commercial Contracts Art. 
% 3.9}
% 
% \begin{enumerate}
%     \item % TODO: note on unidroit -- 55-56
% \end{enumerate}
% 
% \paragraph{Principles of European Contract Law \S\ 4.108}
% 
% \begin{enumerate}
%     \item % TODO note -- 56
% \end{enumerate}
% 
\paragraph{Duress: \emph{Chouinard v. Chouinard}}

Driving a hard bargain is not wrongful.

\begin{enumerate}
    \item The Chouinard family owned Arc Corporation. Fred ran the business. 
    Al, his father, and Ed, Fred's twin brother, both claimed they each owned 
    37.5\% of the company, or about \$500,000 each.
    \item After a bad financial decision, Fred managed to secure a loan 
    commitment from the Heller Company. But Heller refused to make the loan 
    until the ownership dispute was settled, so Fred's attorney drafted an 
    instrument under which all parties would agree that Heller could make the 
    loan but no one would acknowledge anyone else's ownership claim. Al and 
    Ed's attorney rejected the proposal because they wanted to take the 
    opportunity to settle the ownership issue. Fred agreed to pay Ed and Al 
    \$95,000 each, mostly in promissory notes, in exchange for their release 
    of their claims to ownership of the company.
    \item Fred then brought suit to set the promissory notes aside. He argued 
    that the contract was void because it was made under duress resulting from 
    ``impending bankruptcy'' and ``financial peril.'' The court disagreed, 
    holding that Ed and Al had driven a hard bargain but had not acted 
    wrongfully. \enquote{We conclude, therefore, that there is simply no 
    duress shown on this record, for one crucial element is missing: a 
    wrongful act by the defendants to create and take advantage of an 
    untenable situation. Ed and Al had nothing to do with the financial 
    quagmire in which Fred found himself, and we cannot find duress simply 
    because they refused to throw him a rope free of any 
    \enquote{strings.}}\footnote{Casebook p. 57.}
\end{enumerate}

\paragraph{\emph{Post v. Jones}}

% TODO: takeaway

\begin{enumerate}
    \item The Richmond, a whaling ship, ran aground with a full hold of whale 
    oil. Three other whaling ships sailed by. The captain of the Richmond 
    agreed to auction off its oil to the other three ships. Upon return, the 
    Richmond's owners brought suit to recover the value of the oil the other 
    ships took, arguing that the other ships were entitled to ``salvage'' but 
    not to the oil at the auction price.
    \item The court held for the Richmond's owners, reasoning that the auction 
    transactions were invalid because ``one party had absolute power, and the 
    other no choice but submission.'' The rules of salvage allow the rescuers 
    to be compensated for their effort, but the court was not willing to 
    ``permit the performance of a public duty to be turned into a traffic or 
    profit.''\footnote{Casebook p. 58.}
\end{enumerate}

\paragraph{Eisenberg, ``The Bargain Principle and Its Limits''}

\begin{enumerate}
    \item \emph{The Desperate Traveler}: T is stranded in the desert. G passes 
    by and offers a ride for two-thirds of T's wealth or \$100,000, whichever 
    is more.
    \item Under traditional contract doctrine, this agreement would be 
    enforceable. The duress limitation applied only if the promisee was 
    responsible for putting the promisor in a position of distress. But the 
    passerby was not responsible for the desperate traveler's distress.
    \item But such an outcome serves neither fairness nor efficiency.
    \item The maritime salvage doctrine is a reasonable solution. It prevents 
    the rescuer from exploiting the one in distress, but it includes a 
    ``generous bonus to provide a clear incentive for action and compensation 
    for the benefit conferred.''\footnote{Casebook p. 60.}
\end{enumerate}

% \paragraph{New York Gen. Bus. Law \S\ 396-r: Price Gouging}
% 
% \begin{enumerate}
%     \item % TODO 60-62
% \end{enumerate}
% 
% \paragraph{\emph{People v. Two Wheel Corp.}}
% 
% \begin{enumerate}
%     \item % TODO 62-63
% \end{enumerate}
% 

\subsubsection{Unconscionability}

\paragraph{\emph{Williams v. Walker-Thomas Furniture Co.}}

\begin{enumerate}
    \item Over five years, the plaintiffs bought furniture from Walker-Thomas 
    on its installment plan, which provided that if the buyer defaulted on any 
    item, Walker-Thomas could repossess every item purchased at the store. 
    Plaintiffs argued that the contracts were unconscionable and thus 
    unenforceable.
    \item The trial and appellate courts held for Walker-Thomas, finding ``no 
    ground upon which this court can declare the contracts in question 
    contrary to public policy.''\footnote{Casebook p. 65.}
    \item The D.C. Circuit disagreed, holding that unconscionability includes 
    ``an absence of meaningful choice on the part of one of the parties 
    together with contract terms which are unreasonably favorable to the other 
    party.''\footnote{Casebook pp. 65--65.}
    \item Remanded to determine whether the contracts were unconscionable.
    \item Many states later passed statutes to outlaw this type of installment 
    contract.
\end{enumerate}
 
\paragraph{Skilton \& Helstad, ``Protection of the Installment Buyer of 
Goods under the Uniform Commercial Code''}

\begin{enumerate}
    \item % TODO 68-69
\end{enumerate}
 
\paragraph{Pricing by Low-Income Market Retailers}

\begin{enumerate}
    \item % TODO 69-70
\end{enumerate}

\paragraph{UCC \S\ 2-302}

\begin{enumerate}
    \item % TODO supp
\end{enumerate}

\paragraph{Uniform Consumer Credit Code \S\S\ 1.301}

\begin{enumerate}
    \item % TODO  supp
\end{enumerate}

\paragraph{Uniform Consumer Credit Code \S\S\ 5.108}

\begin{enumerate}
    \item % TODO  supp
\end{enumerate}

\paragraph{Federal Trade Commission Regulations---Door-to-Door Sales}

\begin{enumerate}
    \item % TODO  supp
\end{enumerate}

\paragraph{The Uniform Commercial Code}

\begin{enumerate}
    \item % TODO  supp
\end{enumerate}

\paragraph{\emph{Pittsley v. Houser}}

\begin{enumerate}
    \item % TODO 74-75
\end{enumerate}

\paragraph{Comments to the UCC}

\begin{enumerate}
    \item % TODO 75
\end{enumerate}

\paragraph{Unidroit Principles of International Commercial Contracts Art. 
3.10}

\begin{enumerate}
    \item % TODO supp
\end{enumerate}

\paragraph{Principles of European Contract Law \S\S\ 4.109, 4.110}

\begin{enumerate}
    \item % TODO supp
\end{enumerate}

\paragraph{Commission of the European Communities, Council Directive 
93/13/EEC}

\begin{enumerate}
    \item % TODO supp
\end{enumerate}

\paragraph{\emph{Maxwell v. Fidelity Fin. Servs., Inc.}}

\begin{enumerate}
    \item % TODO 
\end{enumerate}

\paragraph{Classical and Modern Contract Law}

\begin{enumerate}
    \item % TODO 83-86
\end{enumerate}

\subsubsection{The Problem of Mutuality}

\paragraph{Corbin, ``The Effect of Options on Consideration}

\begin{enumerate}
    \item % TODO 95
\end{enumerate}

\paragraph{Williston on Contracts \S\ 103B}

\begin{enumerate}
    \item % TODO 95-96
\end{enumerate}

\paragraph{Restatement Second \S\ 77}

\begin{enumerate}
    \item % TODO supp
\end{enumerate}

\paragraph{Illusory Promises}

\begin{enumerate}
    \item % TODO 96-98
\end{enumerate}

\paragraph{\emph{Lindner v. Mid-Continent Petroleum Corp.}}

\begin{enumerate}
    \item % TODO 98--99
\end{enumerate}

\paragraph{\emph{Gurfein v. Werbelovsky}}

\begin{enumerate}
    \item % TODO 99
\end{enumerate}

\paragraph{Mattei v. Hopper}

\begin{enumerate}
    \item % TODO 99-100
\end{enumerate}

\paragraph{Helle v. Landmark, Inc.}

\begin{enumerate}
    \item % TODO 100
\end{enumerate}

\paragraph{Harris v. Time, Inc.}

\begin{enumerate}
    \item % TODO 100--01
\end{enumerate}

\paragraph{Wood v. Lucy, Lady Duff-Gordon}

\begin{enumerate}
    \item % TODO 101-02
\end{enumerate}

\paragraph{UCC \S\ 2-306}

\begin{enumerate}
    \item % TODO supp
\end{enumerate}

\paragraph{Requirements and Output Contracts}

\begin{enumerate}
    \item % TODO 103-04
\end{enumerate}

\subsubsection{Performance of a Legal Duty as Consideration; Modification and 
Waiver of Contractual Duties}

\begin{enumerate}
    \item % TODO 107
\end{enumerate}

\paragraph{\emph{Slattery v. Wells Fargo Armored Serv. Corp.}}

\begin{enumerate}
    \item % TODO 107--09
\end{enumerate}

\paragraph{\emph{Shadwell v. Shadwell}}

\begin{enumerate}
    \item % TODO 109
\end{enumerate}

\paragraph{Restatement Second \S\ 73}

\begin{enumerate}
    \item % TODO supp
\end{enumerate}

\paragraph{N.Y. Penal Law \S\S\ 200.30, 200.35}

\begin{enumerate}
    \item % TODO 109-10
\end{enumerate}

\paragraph{\emph{Denney v. Reppert}}

\begin{enumerate}
    \item % TODO 110-111
\end{enumerate}

\paragraph{\emph{Lingenfelder v. Wainwright Brewery Co.}}

\begin{enumerate}
    \item % TODO 111-114
\end{enumerate}

\paragraph{\emph{Austin Instrument, Inc. v. Loral Corp.}}

\begin{enumerate}
    \item % TODO  117-123
\end{enumerate}

\paragraph{More on the Legal Duty Rule}

\begin{enumerate}
    \item % TODO 123-125
\end{enumerate}

\paragraph{Contract Practice}

\begin{enumerate}
    \item % TODO 125-126
\end{enumerate}

\paragraph{\emph{Schwartzreich v. Bauman-Basch, Inc.}}

\begin{enumerate}
    \item % TODO 126
\end{enumerate}

\paragraph{Restatement First \S\ 406, Illustration 1}

\begin{enumerate}
    \item % TODO 126-127
\end{enumerate}

\paragraph{UCC \S\S\ 3-103(a)(4), 3-104, 3-311}

\begin{enumerate}
    \item % TODO supp
\end{enumerate}

\paragraph{Restatement Second \S\ 279}

\begin{enumerate}
    \item % TODO supp
\end{enumerate}

\paragraph{Restatement Second \S\ 281}

\begin{enumerate}
    \item % TODO supp
\end{enumerate}

\paragraph{The Legal Effect of the ``Executory Accord''}

\begin{enumerate}
    \item % TODO 135-138
\end{enumerate}

\paragraph{Restatement Second \S\ 89}

\begin{enumerate}
    \item % TODO supp
\end{enumerate}

\paragraph{\emph{Angel v. Murray}}

\begin{enumerate}
    \item % TODO 138-141
\end{enumerate}

\paragraph{\emph{Watkins \& Son v. Carrig}}

\begin{enumerate}
    \item % TODO 141-143
\end{enumerate}

\paragraph{UCC \S\ 2-209}

\begin{enumerate}
    \item % TODO supp
\end{enumerate}

\paragraph{CISG Art. 29}

\begin{enumerate}
    \item % TODO 
    \item % TODO note on CISG 143
\end{enumerate}

\paragraph{\emph{Clark v. West}}

\begin{enumerate}
    \item % TODO 144-148
\end{enumerate}

\paragraph{Restatement Second \S\ 84}

\begin{enumerate}
    \item % TODO supp
    \item % TODO illustrations 3, 4, 6, -- p 148
\end{enumerate}

\paragraph{\emph{Nassau Trust Co. v. Montrose Concrete Prods. Corp.}}

\begin{enumerate}
    \item % TODO 149
\end{enumerate}

\paragraph{UCC \S\ 1-107}

\begin{enumerate}
    \item % TODO supp
\end{enumerate}

\paragraph{UCC \S\ 2-209}

\begin{enumerate}
    \item % TODO supp
\end{enumerate}

\paragraph{\emph{BMC Indus., Inc. v. Barth Indus., Inc.}}

\begin{enumerate}
    \item % TODO 149-150
\end{enumerate}

\subsection{Past Consideration}

\subsubsection{Restatement Second \S\ 82}

\begin{enumerate}
    \item % TODO supp
\end{enumerate}

\subsubsection{Restatement Second \S\ 83}

\begin{enumerate}
    \item % TODO supp
\end{enumerate}

\subsubsection{Bankruptcy Code, 11 U.S.C. \S\ 524}

\begin{enumerate}
    \item % TODO supp
\end{enumerate}

\subsubsection{Three Situations in Which a Promise to Discharge an Unenforceable 
Obligation is Binding}

\begin{enumerate}
    \item % TODO 151-152
\end{enumerate}

\subsubsection{\emph{Mills v. Wyman}}

\begin{enumerate}
    \item % TODO background note p. 155
    \item % TODO 152-156
\end{enumerate}

\subsubsection{\emph{Webb v. McGowin}}

\begin{enumerate}
    \item % TODO 156-159
\end{enumerate}

\subsubsection{\emph{Harrington v. Taylor}}

\begin{enumerate}
    \item % TODO 159-160
\end{enumerate}

\subsubsection{Restatement Second \S\ 86}

\begin{enumerate}
    \item % TODO supp
    \item % TODO illustrations p. 160
\end{enumerate}

\subsubsection{ALI 42d Annual Proceedings}

\begin{enumerate}
    \item % TODO 160-161
\end{enumerate}

\subsubsection{Past Consideration}

\begin{enumerate}
    \item % TODO 161-162
\end{enumerate}

\subsection{The Limits of Contract}

\subsubsection{\emph{Balfour v. Balfour}}

\begin{enumerate}
    \item % TODO 163--64
\end{enumerate}

\subsubsection{\emph{In Re the Marriage of Witten}}

\begin{enumerate}
    \item % TODO 164-173 
\end{enumerate}

\subsubsection{Who is a Parent?}

\begin{enumerate}
    \item % TODO 182-83
\end{enumerate}

\subsubsection{\emph{R.R. v. M.H.}}

\begin{enumerate}
    \item % TODO 183-185 
\end{enumerate}

\subsubsection{Surrogate-Parenting Legislation}

\begin{enumerate}
    \item % TODO 185
\end{enumerate}

\subsubsection{42 U.S. Code \S\ 274e: Prohibition of Organ Purchases}

\begin{enumerate}
    \item % TODO 185
\end{enumerate}

\subsubsection{Radin, ``Market-Inalienability''}

\begin{enumerate}
    \item % TODO 186-87
\end{enumerate}
