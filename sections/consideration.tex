\section{Consideration}

\begin{enumerate}
    \item Consideration is giving something and getting something in return. 
    Both parties get something they want from the other.
    \item Consideration is the thing that locks up a contractual 
    relationship---``[s]omething (such as an act, a forbearance, or a return 
    promise) bargained for and received by a promisor from a 
    promisee~.~.~.~''\footnote{Black's Law.}
    \item Why do we require consideration?
    \item Can you sell a bottle of Gatorade for \$10,000? Yes---courts are 
    hesitant to interfere with private transactions (with a few exceptions).
    \item Spectrum of ways a contract can be established:
    \begin{enumerate}
        \item \textbf{Objective}: forms, labels, seals, magic words.
        \item \textbf{Subjective}: intent---e.g., the intent to make a 
        promise.
    \end{enumerate}
    \item Consideration traditionally has two elements:
    \begin{enumerate}
        \item Mutual assent.
        \item Exchange of value.
    \end{enumerate}
    \item ``The rule that a mere verbal promise, without any consideration, 
    cannot be enforced by action, is universal in its action, and cannot be 
    departed from to suit particular cases in which a refusal to perform such 
    a promise may be disgraceful.''\footnote{Casebook p. 153.}
\end{enumerate}

\subsection{Donative Promises, Form, and Reliance}

\subsubsection{Donative Promises}

Made for affective reasons, usually informal, and not demonstrably relied 
upon.\footnote{Casebook p. 6.}

\paragraph{Gifts: \emph{Dougherty v. Salt}}
~\\\\
Donative promises are not enforceable.

\begin{enumerate}
    \item A boy's aunt gave him a promissory note for \$3,000, payable at her 
    death or before. The aunt died. The boy's guardian brought an action 
    against the aunt's estate to recover the promised money.
    \item The jury found for the plaintiff. The trial judge set aside the 
    verdict and dismissed the complaint. The appellate court reversed and 
    reinstated because it held that the promissory note was valid 
    consideration.
    \item The New York Court of Appeals (Cardozo) reversed, holding that the 
    note was ``the voluntary and unenforceable promise of an executory 
    gift.''\footnote{Casebook p. 7.} The note was inadequate consideration in 
    this context because paper does not make a donative promise enforceable.
\end{enumerate}

\paragraph{Restatement Second \S\ 1: Definition}

\begin{enumerate}
    \item ``A contract is a promise or a set of promises for the breach of 
    which the law gives a remedy, or the performance of which the law in some 
    way recognizes as a duty.''
\end{enumerate}

\paragraph{Restatement Second \S\ 17: Requirement of a Bargain}

\begin{enumerate}
    \item ``~.~.~.~a contract requires a bargain in which there is a 
    manifestation of mutual assent to the exchange and a consideration.''
    \item But, there are many exceptions.
\end{enumerate}

\paragraph{Restatement Second \S\ 71: Requirement of Exchange; Types of 
Exchange}

\begin{itemize}
    \item (1) Consideration must be bargained for. 
    \item (2) A performance or return promise is ``bargained for'' if sought and given in exchange.
    \item (3) Performance can be an act, forbearance, or creation, 
    modification, or destruction of a legal relation.
    \item (4) Performance may be given to another person.
\end{itemize}

\paragraph{Restatement Second \S\ 79: Adequacy of Consideration; Mutuality 
of Obligation}

\begin{enumerate}
    \item If the consideration requirement is met, there are no requirements 
    of gain/loss, equivalence in value, or ``mutuality of obligation.''
\end{enumerate}

\paragraph{On the Restatement Second}

\begin{enumerate}
    \item Samuel Williston, Reporter for the Restatement First: contract law 
    is ``a set of axioms that were deemed to be self-evident, together with a 
    set of subsidiary rules that were purportedly deduced from the 
    axioms.''\footnote{Casebook p. 8.}
    \item Arthur Corbin, Consultant to the Restatement Second: father of 
    ``modern contract law.''\footnote{Casebook p. 8.}
\end{enumerate}

\paragraph{Consideration}

\begin{enumerate}
    \item Acceptance alone is not consideration, but at some point acceptance 
    becomes an act that qualifies as consideration.
    \begin{enumerate}
        \item The \textbf{Tramp Example}: I'll call the store and have two 
        suits ready for you. You go to the store and find out I was just 
        kidding.  There's no act there. But what if I make you drive to a 
        store in San Francisco? Is that an act that constitutes consideration? 
        I got what I wanted (for you to go to the store) and you thought you 
        were getting what you wanted (new clothes).
    \end{enumerate}
    \item Courts generally do not consider the adequacy of consideration.
    \item The \textbf{mutuality rule} required both parties to be obligated to 
    each other, but it did not require equal consideration.
    \item Two conceptions: broad and narrow.
    \item \emph{Broad}: ``consideration'' refers collectively to the things 
    that make contracts legally enforceable---e.g., bargain or 
    reliance.\footnote{Casebook p. 8.}
    \item \emph{Narrow}: ``consideration'' is the same thing as ``bargain.'' 
    The Restatement Second adopts this approach, known as the \textbf{bargain 
    theory of consideration}.
    \item The bargain theory of consideration creates two kinds of distortion:
    \begin{enumerate}
        \item \emph{Terminological}: many other elements besides bargain can 
        make a contract enforceable. Therefore, ``under the terminology of the 
        Restatement Second, a promise needs consideration to be enforceable 
        unless it does not need consideration to be 
        enforceable.''\footnote{Casebook p. 9.}
        \item \emph{Substantive}: the bargain theory presumes that all 
        nonbargain promises are unenforceable. It does not allow the law to 
        develop other means of making promises enforceable.
    \end{enumerate}
\end{enumerate}

\paragraph{Gifts}

\begin{enumerate}
    \item The common law distinguishes between gifts and promises to make 
    gifts. A promise to make a gift is not enforceable. \emph{Dougherty}.
    \item A \textbf{deed of gift} (or \textbf{inter vivos document of 
    transfer}) transfers ownership via a written instrument.\footnote{Casebook 
    pp. 9--10.}
    \item Another way to make a gift is for the owner to declare herself a 
    \textbf{trustee} of her property for the benefit of another. The trustee 
    retains legal title but the beneficiary receives beneficial 
    ownership.\footnote{Casebook p. 10.}
    \item Once given, a gift cannot be taken back---but before it's given, 
    we're uncertain.
\end{enumerate}

\paragraph{Donative Promises}

\begin{enumerate}
    \item A donative promise is a promise to make a gift, without any 
    expectation of getting anything in return.
    \item Why make a donative promise, rather than wait and give the gift 
    later?
    \begin{enumerate}
        \item Instincts vary. The promisor might think she has better judgment 
        right now than she will in the future.
        \item The promisor wants to derive satisfaction from the promisee's 
        gratitude.
        \item \textbf{Beneficial reliance}: the promisee can rely on the 
        promise to her benefit---e.g., knowing that her Aunt will pay for 
        college, a niece decides to finish high school, rather than look for a 
        paying job.
    \end{enumerate}
    \item The ``basic fault line'' in classical contract law was between 
    bargain promises (enforceable) and gratuitous promises (unenforceable).
    \item Should contract law be based on the moral belief that breaking 
    promises is unethical, or should it promote utilitarian goals like 
    ``compensating injured promisees and increasing social 
    wealth''?\footnote{Casebook p. 11.}
    \item Lon Fuller introduced \textbf{substantive} and \textbf{process} 
    bases for enforcing promises.
    \begin{enumerate}
        \item The two process bases are evidentiary (making sure a promise 
        has actually been made) and cautionary (preventing inconsiderate 
        action by the promisor).
        \item Simple donative promises are problematic on both bases. They 
        raise problems of proof (evidence may be uncertain) and 
        deliberativeness (the promisor is likely to be emotionally involved 
        with the promisee).
        \item Substantive reasons for enforcing donative promises include 
        compensating the promisee's disappointment (a form of injury), move 
        assets from the wealthy to the less wealthy, and increase the 
        likelihood of beneficial reliance. But these substantive bases are 
        open to question.
    \end{enumerate}
\end{enumerate}

\paragraph{Conditional Donative Promises}

\begin{enumerate}
    \item In a bargain promise, the condition is the \textbf{price} of the 
    promise---e.g., I'll give you \$20 if you mow my lawn.
    \item In a conditional donative promise, the condition is the 
    \textbf{means} to make the gift---e.g., I'll buy you a car if you pick one 
    that costs less than \$15,000.  \item We have to rely on a ``reasonable 
    interpretation'' to determine whether a condition is a means or a 
    price.\footnote{Casebook p. 13.}
\end{enumerate}

\subsubsection{Form}

\begin{enumerate}
    \item We often look for an official indication of a valid 
    agreement, like wax seals, but power of form has mostly evaporated.
\end{enumerate}

\paragraph{Von Mehren, ``Civil-Law Analogues to Consideration: An Exercise 
in Comparative Analysis''}

\begin{enumerate}
    \item French, German, and common law systems recognize four policies that 
    lead them to treat a transaction as unenforceable:
    \begin{enumerate}
        \item \textbf{Evidentiary}: Evidentiary security---protecting against 
        manufactured evidence and dealing with insufficient proof.
        \item \textbf{Cautionary}: Safeguarding the individual against his own 
        rashness.
        \item \textbf{Channeling}: Signaling to make sure that the promisor 
        knows the promise is enforceable.
        \item \textbf{Deterrent}: Unwillingness to enforce contracts of 
        ``suspect or marginal value.''\footnote{Casebook pp. 13--14.}
    \end{enumerate}
\end{enumerate}

\paragraph{Channeling Function of Contract-Law Rules}

\begin{enumerate}
    \item Von Mehren's channeling policy assumes actors know contract 
    law---but must don't. So, this policy only applies to rules that actors 
    are likely to know. 
\end{enumerate}

\paragraph{No Enforcement without Consideration: \emph{Schnell v. Nell}}
~\\\\
Ritual is not enough. To be valid, consideration must impose a legal 
obligation.

\begin{enumerate}
    \item Zacharias Schnell's wife Theresa agreed to pay \$200 each to Nell and 
    the two Lorenzes on her death. After she died, Schnell agreed with the 
    three claimants to pay out the promised money over three years. In 
    exchange, the three claimants gave him one cent. Schnell later decided he 
    wanted to back out of the agreement.
    \item Schnell argued that there was no consideration because his wife did 
    not own any property and therefore could not make such a promise. The 
    lower court sustained a demurrer against Schnell.
    \item The question was whether the contract ``express[ed] a consideration 
    sufficient to give it legal obligation, as against Zacharias 
    Schnell.''\footnote{Casebook p. 15.}
    \item The court considered three bases for consideration:
    \begin{enumerate}
        \item First, the claimants promised to pay Schnell one cent. Although 
        courts generally do not evaluaate the adequacy of consideration, this 
        exchange---one cent for \$200---is too unconscionable to 
        sustain. (``Even in traditional times, courts were not willing agents 
        of absurdity or oppression.''---Berring.)
        \item Second, Mr. Schnell bore love and affection to his late wife, 
        and ``she had done her part, as his wife, in the acquisition of the 
        property.''\footnote{Casebook p. 15.}. This was invalid because these 
        are past considerations and have no bearing on Schnell's promise to 
        the three claimants.
        \item Third, Mrs. Schnell wished to give the money to the three 
        claimants. This was invalid because she had no property of her own.
    \end{enumerate}
\end{enumerate}

\paragraph{Form}

\begin{enumerate}
    \item Formal elements (wax seals, etc.) used to create special legal 
    effect, but their use is waning.
    \item One consequence of the removal of legal effect of formal devices is 
    that there is no longer any device to make a donative promise binding. 
    Some states have adopted statutes that make written instruments 
    enforceable. European countries have implemented systems that weigh a 
    range of factors involved in a donative promise, like the promisor's 
    financial situation and the grantee's acts of 
    ingratitude.\footnote{Casebook pp. 22--23.}
\end{enumerate}

\subsubsection{Reliance and Promissory Estoppel}

\begin{enumerate}
    \item A moral obligation to keep a promise is generally insufficient to 
    establish a legal obligation.
    \item Lawmakers should also consider social policy and experiential 
    concerns (e.g., problems of proof).
    \item Injury from reliance is a moral and policy reason for enforcing 
    promises.
\end{enumerate}

\paragraph{No Reliance in Traditional Common Law: \emph{Kirksey v. Kirksey}}
~\\\\
These facts are analogous to the tramp example. Was this a mere gift or was 
there consideration? The defendant got nothing in return for his offer, but 
the plaintiff \emph{did} do what he wanted---so you can start the 
consideration analysis.

\begin{enumerate}
    \item ``Dear Sister Antillico~.~.~.~''
    \item The defendant, plaintiff's brother in law, invited the plaintiff and 
    her family to come live at his house, offering her living space and land 
    to tend, but asking nothing in return. She abandoned her property in the 
    country to come live with him. He let her live comfortably in the house 
    for two years. Then, he put her in a house, ``not comfortable,'' in the 
    woods, and then made her leave.
    \item The lower court found for the plaintiff for \$200.
    \item The judge writing the opinion for the Supreme Court of Alabama 
    believed that the ``loss and inconvenience'' she suffered justified the 
    award, but the majority believed the offer was a ``mere gratuity.'' 
    Reversed.\footnote{Casebook pp. 24--25.}
\end{enumerate}

\paragraph{Restatement Second \S\ 90: Promise Reasonably Inducing Action or 
Forbearance}

\begin{enumerate}
    \item ``A promise which the promisor should reasonably expect to induce 
    action or forbearance on the part of the promisee or a third person and 
    which does induce such action or forbearance is binding if injustice can 
    be avoided only by enforcement of the promise. The remedy granted for 
    breach may be limited as justice requires.''
\end{enumerate}

\paragraph{Estoppel in Pais and Promissory Estoppel}

\begin{enumerate}
    \item Under classical contract law, only a bargain was consideration. 
    Donative promises were not enforceable, even if the promisee relied on the 
    promise. \emph{Kirksey}.\footnote{Casebook p. 25.}
    \item \emph{Ricketts}: a man promised his granddaughter \$2,000 so she 
    would not have to work anymore. She quit her job. The man died. The 
    granddaughter successfully brought suit. The court avoided the question of 
    whether reliance constituted consideration. Instead, reliance estopped the 
    man's estate from pleading a lack of consideration. This case lies 
    somewhere between estoppel in pais and promissory estoppel.
    \item \textbf{Estoppel in pais} (or equitable estoppel): if A made a 
    statement and B foreseeably relied on that statement, A is estopped from 
    denying that statement's truth. Based on a \emph{statement of fact}. For 
    instance, say a store receives a suspicious check. It calls the owner of 
    the bank account to make sure it's valid. The owner absent-mindedly said 
    yes, but the check was in fact a forgery. The customer is estopped from 
    denying the truth of his statement.\footnote{See Understanding Contracts 
    p. 139.}
    \item \textbf{Promissory estoppel}: a promise is binding if the promisee 
    reasonably relied on it. Codified in Restatement Second \S\ 90. Based on a 
    \emph{promise}.
    \item Both are instances of a broader \textbf{``reliance principle''}: 
    ``When one person, A, uses words or actions that he knows or should know 
    would induce another, B, to reasonably believe that A is committed to take 
    a certain course of action, and A knows or should know that B will incur 
    costs if A doesn't take action, A should take steps to ensure that if he 
    doesn't take the action, B will not suffer a loss.''\footnote{Casebook p. 
    27.}
\end{enumerate}

\paragraph{Promissory Estoppel Problem}

\begin{enumerate}
    \item An uncle promises his nephew \$50,000 to buy a business. The nephew 
    buys the business with his own money. The uncle dies before depositing the 
    promised money. The executor of the uncle's estate refuses to recognize 
    the nephew's claim.
    \item Can the nephew base an action on:
    \begin{enumerate}
        \item Deceit? (No. The uncle did not try to conceal anything.)
        \item Estoppel in pais? (No. The dispute does not turn on the truth of
        of a factual statement.)
        \item A promise made enforceable by unbargained-for reliance? (Yes.)
        \item A promise supported by a bargained-for consideration? (No. There 
        was no bargain.)
    \end{enumerate}
\end{enumerate}

\paragraph{Promissory Estoppel: \emph{Feinberg v. Pfeiffer Co.}}
~\\\\
Feinberg planned her retirement on the basis of the company's promise to pay 
her a pension. Her reliance made the promise binding.

\begin{enumerate}
    \item Facts:
    \begin{enumerate}
        \item 1910: Feinberg began working for Pfeiffer.
        \item 1947: Still working for Pfeiffer, now in more senior positions.
        \item December 27, 1947: the company resolved to raise her salary to 
        \$400/month and offer a lifetime retirement pay of \$200 per month 
        (but with the hope that she would continue working).
        \item June 30, 1949: Feinberg retired and began collecting her monthly 
        payments.
        \item Spring 1956: the company's new ownership refused to continue 
        sending the monthly retirement checks.
        \item April 1, 1956: the company sent its final check, for \$100. 
        Feinberg refused it and brought suit to recover her full monthly 
        retirement checks.
    \end{enumerate}
    \item The trial judge held for Feinberg and awarded \$5,100.
    \item Pfeiffer raised two peripheral arguments on appeal:
    \begin{enumerate}
        \item The court erroneously allowed evidence that at the time of 
        trial, Feinberg could not seek employment because she was suffering 
        from cancer.
        \begin{enumerate}
            \item Held: this was not the basis for the trial court's decision.
        \end{enumerate}
        \item There was insufficient evidence to support the court's finding 
        that (1) Feinberg would not have quit unless she relied on the monthly 
        retirement checks and (2) Feinberg did in fact rely on the checks 
        after retiring.
        \begin{enumerate}
            \item Held: the lower court properly credited Feinberg's testimony 
            that she relied on the checks.
        \end{enumerate}
    \end{enumerate}
    \item The key issue was whether Feinberg had a right to recover her 
    retirement checks on the basis of a contractual 
    obligation.
    \begin{enumerate}
        \item Pfeiffer: the company's decision to give the retirement checks 
        was a mere promise to make a gift. Donative promises are not 
        enforceable unless supported by consideration. The only possible 
        consideration here was a reward for past service, which is not valid. 
        Further, Feinberg's position was at-will.
        \item Feinberg: agreed that a promise based on past services would not 
        constitute valid consideration, but (1) she continued to work for a 
        year and a half after the company's resolution and (2) she based her 
        decision to retire on reliance on the monthly check.
        \item Court: 
        \begin{enumerate}
            \item Feinberg's first contention, that she continued to work 
            after the resolution, does not constitute valid consideration 
            because that was not a condition of the agreement.
            \item But Feinberg's second argument, that she relied on the 
            promise, was valid. The promise met the Restatement Second \S\ 90 
            requirements for promissory estoppel (in fact, one of the 
            Restatement illustrations involves retirement annuities).
            \item Affirmed.
        \end{enumerate}
    \end{enumerate}
\end{enumerate}

\paragraph{Distinguishing \emph{Feinberg}: \emph{Hayes v. Plantations 
Steel Co.}}
~\\\\
An employee cannot claim reliance on a retirement pension if the employer's 
promise did not affect his decisionmaking.

\begin{enumerate}
    \item Plantations Steel told Hayes it would ``take care'' of him upon his 
    retirement, but there was no mention of an actual sum. It gave a first 
    payment ``as a token of appreciation'' and implied that the payments would 
    continue annually.
    \item The company changes ownership. Later owners discontinued the 
    payments. Hayes brought suit.
    \item The key distinction from \emph{Feinberg} is that the company's offer 
    of annuities was not an inducement to retire, because Hayes had given 
    notice of retiring several months earlier. He did not rely on the 
    company's promise because nothing about the promise changed his behavior. 
    Therefore, the promise did not meet the requirements for promissory 
    estoppel.
    \item Would it matter if the check had come annually for 10 years? 30 
    years?
    \begin{enumerate}
        \item He chose to retire before the offer.
        \item He expressed the uncertainty by showing up each year and 
        wondering if the check was there for him.
    \end{enumerate}
\end{enumerate}

\paragraph{Remedies and Consideration}

\begin{enumerate}
    \item To what extent should a promise be enforced?
    \item For more, see the Remedies section below.
    \item Two types of \textbf{injury}:\footnote{Casebook p. 35.}
    \begin{enumerate}
        \item \textbf{Reliance interest}: the promisee is worse off than he 
        would have been if the promise had not been made. \textbf{Reliance 
        damages} restore a promisee to the position he would have been in had 
        the promise not been made.
        \item \textbf{Expectation interest}: the promisee is worse off than he 
        would have been if the promise had been performed. \textbf{Expectation 
        damages} restore the promisee to the position he would have been in 
        had the promise been performed.
    \end{enumerate}
    \item Two types of \textbf{reliance damages}:
    \begin{enumerate}
        \item \textbf{Out-of-pocket costs}: costs incurred by promisee prior 
        to the breach.
        \item \textbf{Opportunity costs}: the surplus the promisee would have 
        enjoyed if he had taken the opportunity that the promise led him to 
        forgo.
    \end{enumerate}
    \item To what extent should a relied-upon donative promise be enforced? 
    Should it be the extent of the reliance (the \textbf{reliance measure}) or 
    the full expectation of the promise (the \textbf{expectation measure})? See 
    casebook pp. 37--38.
    \item The expectation measure can be useful where reliance damages are 
    difficult to establish but obviously significant.\footnote{Casebook p. 39.}
\end{enumerate}

\paragraph{\emph{Goldstick v. ICM Realty}}

\begin{enumerate}
    \item The expectation measure is valuable because it's simple, and it 
    often covers opportunity costs that the reliance measure 
    misses.\footnote{Casebook p. 40.}
\end{enumerate}

\paragraph{Reliance Damages vs. Expectation Damages: \emph{D \& G Stout, 
Inc. v. Bacardi Imports, Inc.}}
~\\\\
D \& G suffered significant damages from its reliance on Bacardi's promise. 
Bacardi was liable for reliance damages.

Lesson 1: get it in writing. Lesson 2: create consideration in the agreement, 
e.g., exchange money, in order to guarantee enforceability of the promise.

\begin{enumerate}
    \item D \& G, operating as General Liquors, faced the dilemma of whether 
    to sell out or scale down. Bacardi promised it would continue to deal with 
    General as its Northern Indiana distributor. Relying on this promise, 
    General turned down the selling price it had been offered from National 
    Wine \& Spirits. A week later, Bacardi withdrew its account, forcing 
    General to sell at a much lower price--\$550,000 lower than National's 
    original offer.
    \item Can General recover the difference in price from Bacardi? Trial 
    court: no. Bacardi's promise was not one on which it should reasonably 
    expected General to rely.\footnote{Casebook p. 42.}
    \item Indiana followed Restatement Second \S\ 90 on promissory estoppel.
    \item Appellate court:
    \begin{enumerate}
        \item In at-will employment relationships in Indiana, employees cannot 
        sue for lost wages, but they \emph{can} sue for moving expenses 
        incurred on the basis of a promise.\footnote{Casebook p. 42.}
        \item This is the distinction between reliance damages (moving 
        expenses---\emph{Eby}) and expectation damages (lost anticipated 
        wages---\emph{Ewing}).
        \item Was General's loss the result of lost expectations of future 
        profit (expectation damages) or an opportunity foregone in reliance on 
        the promise (reliance damages)?
        \item In its negotiations, General relied on Bacardi's promise. The 
        devaluation of the company as a result of the broken promise was a 
        reliance injury, analogous to moving expenses.
        \item Reversed.
    \end{enumerate}
\end{enumerate}

\paragraph{Promissory Estoppel and Expectation Damages: \emph{Walters v. 
Marathon Oil Co.}}
~\\\\
The Walters reasonably relied on Marathon's promise to let them open a 
dealership. The court awarded remedies to put the Walters in the position they 
would have been in if the contract had been performed (expectation damages).

\begin{enumerate}
    \item Mr. and Mrs. Walters hoped to open a gas station. Relying on 
    promises from Marathon Oil, the Walters bought and renovated and abandoned 
    service station. Before signing the final agreement, Marathon put a 
    moratorium on new applications for dealerships. The Walters brought suit.
    \item The Walters argued that they should be able to recover lost 
    potential profits, totaling \$22,200 (370,000 gallons at six cents per 
    gallon).
    \item Marathon argued that the Walters should only be able to recover for 
    their reliance injury, i.e., the difference between their expenditures on 
    the site (purchase and renovations) and its present value---and actually, 
    real estate values increased, so they really made a small profit.
    \item The trial court sided with the Walters and the appellate court 
    agreed, holding that the Walters had forgone the opportunity to make their 
    investment elsewhere.
    \item ``Since promissory estoppel is an equitable matter, the trial court 
    has broad power in its choice of a remedy~.~.~.~''\footnote{Casebook p. 
    45.}
\end{enumerate}

\subsection{The Bargain Principle} 

\begin{enumerate}
    \item Consideration exists when the promise is part of an exchange. Both 
    parties receive something in exchange for their promise---a 
    ``bargained-for exchange.''
    \item The ``two major props of the bargain principle'' are 
    \textbf{fairness} and \textbf{efficiency}.\footnote{Casebook p. 59. See 
    also Eisenberg, ``The Bargain Principle and Its Limits'' below.}
    \item Moral obligations alone are generally insufficient to establish 
    consideration.
    \item Under classical contract law, one-sided contractual modifications, 
    where only one party changes its obligations, are not enforceable because 
    there has been no consideration.
\end{enumerate}

\subsubsection{The Bargain Principle}

\begin{enumerate}
    \item ``It is an elementary principle, that the law will not enter into an 
    inquiry as to the adequacy of consideration.''\footnote{Casebook p. 47.} 
    \emph{Westlake v. Adams, C.P.}
\end{enumerate}
 
\paragraph{Benefit and Detriment: \emph{Hamer v. Sidway}}

\enquote{\enquote{Consideration} means not so much that one party is profiting 
as that the other abandons some legal right in the present, or limits his 
legal freedom of action in the future, as an inducement for the promise of the 
first.}\footnote{Casebook p. 49.}

\begin{enumerate}
    \item William E. Story, Sr. promised his nephew, William E. Story, 2d, 
    \$5,000 if he would avoid drinking, gambling, etc. until he was 21. When 
    Story 2d reached 21, he and his uncle agreed that the uncle would keep 
    the money until Story 2d was mature enough to manage it, at which point he 
    would receive the \$5,000 plus interest. Story Sr. died without 
    transferring the money. Story 2d brought suit against the estate's 
    executor to recover.
    \item The executor argued that Story 2d, the promisee, was not harmed, but 
    benefited, and that Story Sr., the promisor, was not benefited. He argued 
    that ``unless the promisor was benefited, the contract was without 
    consideration.''\footnote{Casebook p. 49.}
    \item The court here disagreed, holding that consideration exists when 
    ``something is promised, done, forborne, or suffered by the party to whom 
    the promise is made as consideration for the promise made to 
    him.''\footnote{Casebook p. 49.}
\end{enumerate}

\paragraph{Statute of Frauds}

\begin{enumerate}
    \item What promises must be in writing to be enforceable?
    \item Lesson: get it in writing.
\end{enumerate}

\paragraph{Defining Detriment: \emph{Davies v. Martel Laboratory Services, 
Inc.}}
~\\\\
``Detriment'' can include giving up freedom of action, even if it generally 
``benefits'' a person. In this case, there was consideration because the 
company got what it wanted.

\begin{enumerate}
    \item Davies was an at-will employee of Martel. Martel offered her a 
    deal in which she would become a permanent vice president if she 
    earned an MBA, and Martel offered to pay for half of the degree. Davies 
    enrolled in an MBA program. Martel fired her a year later.
    \item Martel argued that there was no consideration because the pursuit of 
    an MBA was a benefit to Davies, not a disadvantage or detriment.
    \item The trial court granted summary judgment for Martel.
    \item The appellate court reversed, holding that Martel's definition of 
    ``detriment'' was inaccurate. Legal detriment also means giving up the 
    privilege to do something.
\end{enumerate}

\paragraph{Bad (or Great) Deals: \emph{Hancock Bank \& Trust Co. v. Shell Oil 
Co.}}
~\\\\
Bad (or great) deals are binding.

\begin{enumerate}
    \item Hancock bought land after making a deal with Shell that Shell would 
    lease the land for fifteen years at very low rent, with the option to 
    renew for another fifteen years or to terminate the lease with 90 days' 
    notice.
    \item The bank then tried to get out of the agreement, arguing that the 90 
    days' notice provision in a fifteen-year lease is ``so lacking in 
    mutuality as to be void against public policy.''\footnote{Casebook p. 51.} 
    The court disagreed, holding that if there is valid consideration, parties 
    cannot escape a contract simply because they realized they got a bad deal.
\end{enumerate}

\paragraph{Restatement Second \S\ 71: Requirements of Exchange; Types of 
Exchange}

\begin{enumerate}
    \item Consideration exists only if the promise is bargained for.
    \item Performance is bargained for if is sought and given.
    \item Performance can consist of:
    \begin{enumerate}
        \item Act.
        \item Forbearance.
        \item Creation, modification, or destruction of a legal relation.
    \end{enumerate}
    \item Performance may be given to the promisor or some other person.
\end{enumerate}

\paragraph{Restatement Second \S\ 72: Exchange of Promise for Performance}

\begin{enumerate}
    \item Any performance that is bargained for is consideration (except for 
    cases in \S\ 73, legal duty, and \S\ 74, settlement of claims).
\end{enumerate}

\paragraph{Restatement Second \S\ 79: Adequacy of Consideration; Mutuality of 
Obligation}

\begin{enumerate}
    \item If there is consideration, there is no requirement of 
    gain/detriment, equivalence in values exchanged, or mutuality of 
    obligation.
\end{enumerate}

\paragraph{Selling Money: \emph{Batsakis v. Demotsis}}
~\\\\
It's possible to sell money. Loan agreements are enforceable even if the 
amount to be repaid is orders of magnitude larger than the amount loaned.

This case is distinct from \emph{Schnell v. Nell} in two ways. First, the \$25 
in this case was not meant as a gift. It was a real exchange, which may have 
meant the difference between life and death under the 
circumstances.\footnote{See \emph{Understanding Contracts} pp. 88--89.} 
Second, this case involved exchanging foreign currency. The act of exchanging 
currency in a market created consideration.

\begin{enumerate}
    \item During World War II, Batsakis loaned the equivalent of \$25 to 
    Demotsis. Demotsis agreed to pay him \$2,000 plus interest as soon as she 
    could access her American accounts. Demotsis signed a letter indicating 
    that she received \$2,000 and would repay it with interest. Batsakis sued 
    to recover the \$2,000 plus interest.
    \item The trial court awarded \$750 plus interest, on the half-baked 
    theory that half-restitution would be fair (---half-baked because 
    contracts are binary: you either have a contract or not, and you can't 
    just cut it in half).
    \item Demotsis argued that the contract did not have valid consideration 
    because she only received \$25. 
    The appellate court disagreed.  Batsakis essentially sold \$25 to Demotsis 
    at the price of \$2,000 plus interest. ``Defendant got exactly what she 
    contracted for according to her own testimony.''\footnote{Casebook p. 54.} 
    The court awarded \$2,000 plus interest.
\end{enumerate}

\paragraph{Sweet-Escott, Greece---A Political and Economic Survey, 
1939--1953}

\begin{enumerate}
    \item Greece was in crisis during World War II. This was the context for 
    \emph{Batsakis}. Batsakis perhaps took advantage of Demotsis under the 
    circumstances, weakening his argument that consideration was valid.
\end{enumerate}

\paragraph{Consideration and Remedies}

\begin{enumerate}
    \item To what extent should a particular type of promise be enforced?
    \item This problem doesn't matter when the three types of 
    damages---expectation, reliance, and restitution---produce identical 
    results. For instance, if a homeowner fails to pay a plumber \$60 for an 
    hour of work, the plumber's expectation damages were \$60, and his 
    reliance damages were also likely \$60 because he could have been working 
    for the same rate on another job.
    \item But sometimes the damages differ. In \emph{Batsakis}, for instance, 
    the defendant argued that the value of what she received was far less than 
    the value she promised. She was willing to repay the \$25 she received 
    (restitution damages) but not the value of what she had promised 
    (expectation damages).
    \item In \emph{Batsakis}, both parties agreed that the contract was 
    enforceable, but they differed on the extent. So when the court said it 
    would not review the adequacy of consideration, it was saying that the 
    normal remedy for breach of contract is expectation damages, and it 
    doesn't matter whether the value of one promised performance exceeded the 
    value of the other.\footnote{Casebook p. 55.}
\end{enumerate}
 
\paragraph{Restatement Second \S\ 175: When Duress by Threat Makes a 
Contract Voidable}

\begin{enumerate}
    \item A contract is void when:
    \begin{enumerate}
        \item When the victim had no choice but to assent because of a threat.
        \item When a third party creates duress, unless the other party to the 
        transaction in good faith did not know of the duress and gives value 
        or relies on the transaction.
    \end{enumerate}
\end{enumerate}

\paragraph{Restatement Second \S\ 176: When a Threat is Improper}

\begin{enumerate}
    \item A threat is improper when it is:
    \begin{enumerate}
        \item A crime or tort.
        \item Criminal prosecution.
        \item Bad faith civil process.
        \item Breach of duty of good faith and fair dealing.
    \end{enumerate}
\end{enumerate}

\paragraph{UNIDROIT Principles of International Commercial Contracts Art.  
3.9: Threat}

\begin{enumerate}
    \item A threat invalidates a contract if the threat is ``so imminent and 
    serious as to leave the first party no reasonable alternative'' but to 
    agree
    \item (The UNIDROIT Principles was an attempt to develop a framework for 
    international contract law.)
\end{enumerate}

\paragraph{Principles of European Contract Law \S\ 4.108: Threats}

\begin{enumerate}
    \item A threat invalidates a contract if (1) the threat was wrongful in 
    itself or (2) wrongful to use as a means to obtain the conclusion of the 
    contract, unless the threatened party had a reasonable alternative.
\end{enumerate}

\paragraph{Duress: \emph{Chouinard v. Chouinard}}
~\\\\
Driving a hard bargain is not wrongful.

\begin{enumerate}
    \item The Chouinard family owned Arc Corporation. Fred ran the business. 
    Al, his father, and Ed, Fred's twin brother, both claimed they each owned 
    37.5\% of the company, or about \$500,000 each.
    \item After a bad financial decision, Fred managed to secure a loan 
    commitment from the Heller Company. But Heller refused to make the loan 
    until the ownership dispute was settled, so Fred's attorney drafted an 
    instrument under which all parties would agree that Heller could make the 
    loan but no one would acknowledge anyone else's ownership claim. Al and 
    Ed's attorney rejected the proposal because they wanted to take the 
    opportunity to settle the ownership fight. Fred agreed to pay Ed and Al 
    \$95,000 each, mostly in promissory notes, in exchange for their release 
    of their claims to ownership of the company.
    \item Fred then brought suit to set the promissory notes aside. He argued 
    that the contract was void because it was made under duress resulting from 
    ``impending bankruptcy'' and ``financial peril.'' The court disagreed, 
    holding that Ed and Al had driven a hard bargain but had not acted 
    wrongfully. \enquote{We conclude, therefore, that there is simply no 
    duress shown on this record, for one crucial element is missing: a 
    wrongful act by the defendants to create and take advantage of an 
    untenable situation. Ed and Al had nothing to do with the financial 
    quagmire in which Fred found himself, and we cannot find duress simply 
    because they refused to throw him a rope free of any 
    \enquote{strings.}}\footnote{Casebook p. 57.}
\end{enumerate}

\paragraph{Rescue and Salvage: \emph{Post v. Jones}}
~\\\\
The rules against duress prevent a rescuer from unfairly profiting from the 
rescue. However, the rules of salvage create an incentive to rescue, striking 
a balance between the rescuer's effort and the victim's need.

\begin{enumerate}
    \item The Richmond, a whaling ship, ran aground with a full hold of whale 
    oil. Three other whaling ships sailed by. The captain of the Richmond 
    agreed to auction off its oil to the other three ships. Upon return, the 
    Richmond's owners brought suit to recover the value of the oil the other 
    ships took, arguing that the other ships were entitled to ``salvage'' but 
    not to the oil at the auction price.
    \item The court held for the Richmond's owners, reasoning that the auction 
    transactions were invalid because ``one party had absolute power, and the 
    other no choice but submission.'' The rules of salvage allow the rescuers 
    to be compensated for their effort, but the court was not willing to 
    ``permit the performance of a public duty to be turned into a traffic or 
    profit.''\footnote{Casebook p. 58.}
\end{enumerate}

\paragraph{Eisenberg, ``The Bargain Principle and Its Limits''}

\begin{enumerate}
    \item \emph{The Desperate Traveler}: T is stranded in the desert. G passes 
    by and offers a ride for two-thirds of T's wealth or \$100,000, whichever 
    is more.
    \item Under traditional contract doctrine, this agreement would be 
    enforceable. The duress limitation applied only if the promisee was 
    responsible for putting the promisor in a position of distress. But the 
    passerby didn't cause the desperate traveler's distress.
    \item But that outcome serves neither fairness nor efficiency.
    \item The maritime salvage doctrine is a reasonable solution. It prevents 
    the rescuer from exploiting the one in distress, but it includes a 
    ``generous bonus to provide a clear incentive for action and compensation 
    for the benefit conferred.''\footnote{Casebook p. 60.}
\end{enumerate}

\paragraph{New York Gen. Bus. Law \S\ 396-r: Price Gouging}

\begin{enumerate}
    \item Sellers cannot take ``unfair advantage of consumers during abnormal 
    disruptions of the market.''\footnote{Casebook p. 60.}
\end{enumerate}

\paragraph{Price Gouging: \emph{People v. Two Wheel Corp.}}

\begin{enumerate}
    \item A power generator seller ran afoul of the New York price gouging 
    rule.
\end{enumerate}

\subsubsection{Unconscionability}

\paragraph{Lack of Meaningful Choice: \emph{Williams v. Walker-Thomas 
Furniture Co.}}
~\\\\
``Unconscionability has generally been recognized to include an absence of 
meaningful choice on the part of one of the parties together with contract 
terms which are unreasonably favorable to the other party.''\footnote{Casebook 
pp. 65--66.}

This contract probably still would have been unconscionable if the buyer had 
been a sophisticated lawyer, but it would have been a closer case. 
\begin{enumerate}
    \item Over five years, the plaintiffs bought furniture from Walker-Thomas 
    on its installment plan, which provided that if the buyer defaulted on any 
    item, Walker-Thomas could repossess every item purchased at the store. 
    Plaintiffs argued that the contracts were unconscionable and thus 
    unenforceable.
    \item The trial and appellate courts held for Walker-Thomas, finding ``no 
    ground upon which this court can declare the contracts in question 
    contrary to public policy.''\footnote{Casebook p. 65.}
    \item The D.C. Circuit disagreed, holding that unconscionability includes 
    ``an absence of meaningful choice on the part of one of the parties 
    together with contract terms which are unreasonably favorable to the other 
    party.''\footnote{Casebook pp. 65--65.}
    \item Remanded to determine whether the contracts were unconscionable.
    \item Many states later passed statutes to outlaw this type of installment 
    contract.
\end{enumerate}
 
\paragraph{UCC \S\ 2-302: Unconscionable Contract or Clause}

\begin{enumerate}
    \item Courts can strike or modify unconscionable contracts.
\end{enumerate}

\paragraph{Uniform Consumer Credit Code \S\ 5.108: Unconscionability}

\begin{enumerate}
    \item Courts can strike or modify unconscionable contracts.
    \item Factors in determining unconscionability include:
    \begin{enumerate}
        \item Seller's knowledge that the buyer will not benefit.
        \item Gross disparity between product/service and price.
        \item Seller takes advantage of buyer's infirmities (illiteracy, 
        etc.).
    \end{enumerate}
\end{enumerate}

\paragraph{Federal Trade Commission Regulations---Door-to-Door Sales}

\begin{enumerate}
    \item A detailed enumeration of unfair door-to-door sales practices.
\end{enumerate}

\paragraph{The Uniform Commercial Code}

\begin{enumerate}
    \item The UCC aims to unify commercial law among the states.
    \item The UCC applies only to ``goods.''
    \item Many states have adopted the UCC in whole or in 
    part.\footnote{Casebook pp. 72--74.}
\end{enumerate}

\paragraph{Applying the UCC: \emph{Pittsley v. Houser}}

\begin{enumerate}
    \item The UCC applies only to ``goods.'' This contract involved the 
    purchase of a carpet (a good) and its installation (a service). The 
    question was whether the UCC applied.
    \item The ``predominant factor'' test asks whether the bulk of the 
    contract involved goods or services.
    \item Another approach allows the goods and services portions of the 
    contract to be severed.
    \item The court here applied the predominant factor test, holding it was 
    advantageous to treat the contract as a whole, since the UCC aims to 
    simplify and clarify contract law. It found that the UCC applied.
\end{enumerate}

\paragraph{Comments to the UCC}

\begin{enumerate}
    \item In states that have adopted the UCC, the legal status of the 
    accompanying comments is unclear.
\end{enumerate}

\paragraph{UNIDROIT Principles of International Commercial Contracts Art. 
3.10: Gross Disparity}

\begin{enumerate}
    \item Taking unfair advantage of the other party's circumstances or 
    limitations can invalidate a contract.
\end{enumerate}

\paragraph{Principles of European Contract Law \S\S\ 4.109, 4.110}

\begin{enumerate}
    \item Largely the same as UNIDROIT \S\ 3.10 above.
\end{enumerate}

\paragraph{Commission of the European Communities, Council Directive 
93/13/EEC}

\begin{enumerate}
    \item Detailed specifications about what constitutes an unfair contract.
\end{enumerate}

\paragraph{Substantive and Procedural Unconscionability: \emph{Maxwell v. 
Fidelity Fin. Servs., Inc.}}
~\\\\
The court distinguished between \textbf{substantive unconscionability} 
(one-sidedness) and \textbf{procedural unconscionability} (fine-print 
surprises). It held that substantive unconscionability alone is sufficient to 
invalidate a contract.

\begin{enumerate}
    \item December 1984: a door-to-door salesman sold the Maxwells a water 
    heater for \$6,512 on an installment plan. The unit was never installed 
    properly and never worked properly. The total purchase price after all 
    installments was \$15,000. As part of the deal, the financing company, 
    Fidelity, placed a lean on the Maxwell house.
    \item 1988: Elizabeth Maxwell wanted to borrow \$800 from Fidelity for an 
    unrelated issue. Fidelity drew up a new contract that included the 
    outstanding water heater balance, the \$800 loan, and a life insurance 
    policy.
    \item Maxwell made payments until 1990, at which point she asserted that 
    the contract was unconscionable and therefore unenforceable.
    \item Fidelity argued that the 1988 contract ``worked a novation,'' 
    barring any action on the 1984 contract. The trial court granted summary 
    judgment for Fidelity and the appellate court affirmed.
    \item The court here distinguished between substantive unconscionability 
    (``an unjust or one-sided contract'') and procedural unconscionability 
    (fine print surprises, etc.).\footnote{Casebook p. 79.} It held that 
    substantive unconscionability alone was sufficient to invalidate a 
    contract, and that there was sufficient evidence to raise a question of 
    ``grossly-excessive price.''\footnote{Casebook p. 81.} Reversed.
\end{enumerate}

\paragraph{Classical and Modern Contract Law}

\begin{enumerate}
    \item Contract law reasoning:
    \begin{enumerate}
        \item \emph{Substantive legal reasoning}: validity of a doctrine turns 
        on normative considerations (morals, policy, etc.).
        \item \emph{Formal legal reasoning}: law consists of doctrines that 
        are autonomous from policy, morality, and experience.
        \item \emph{Axiomatic legal reasoning}: ``fundamental doctrines can be 
        established on the ground that they are 
        self-evident.''\footnote{Casebook p. 83.}
        \item \emph{Deductive legal reasoning}: most doctrines follow from 
        syllogisms beginning with more fundamental doctrines.
        \item Classical contract law coupled axiomatic and deductive 
        reasoning.
        \item By contrast, modern contract law reasoning justifies doctrines 
        on the basis of morality, policy, and experience.
    \end{enumerate}
    \item Contract law can be plotted along four axes:
    \begin{enumerate}
        \item Objectivity (directly observable state of the world) 
        $\leftrightarrow$ subjectivity (mental state).
        \item Standardization (depends on abstract variables) $\leftrightarrow$ 
        individualization (depends on situation-specific variables).
        \item Static (depends on what occurred at the moment the contract was 
        formed) $\leftrightarrow$ dynamic (depends on moving streams of events 
        before and after the contract).
        \item Binary (e.g., no damages or expectation damages) 
        $\leftrightarrow$ multifaceted (e.g., no damages, expectation damages, 
        reliance damages, restitution damages).
    \end{enumerate}
    \item Many other limits on contracts are surrogates for 
    unconscionability.\footnote{Casebook p. 86.}
\end{enumerate}

\subsubsection{Mutuality}

\begin{enumerate}
    \item ``As a contract defense, the mutuality doctrine has become a 
    faltering rampart to which a litigant retreats at his own 
    peril.''\footnote{Casebook p.  100.} \emph{Helle v. Landmark, Inc.}
\end{enumerate}

\paragraph{Corbin, ``The Effect of Options on Consideration}

\begin{enumerate}
    \item B must actually agree to limit his behavior as A requests for the 
    contract to be binding. Otherwise, it's an illusory contract.
\end{enumerate}

\paragraph{Williston on Contracts \S\ 103B}

\begin{enumerate}
    \item Illusory promises are not consideration.
\end{enumerate}

\paragraph{Restatement Second \S\ 77: Illusory and Alternative Promises}

\begin{enumerate}
    \item A promise is not consideration if the promisor reserves 
    alternatives, unless the alternatives would have been consideration or 
    would have been eliminated.
\end{enumerate}

\paragraph{Illusory Promises}

\begin{enumerate}
    \item There is no consideration without commitment.
    \item A valid promise shrinks the promisor's freedom of choice. An 
    illusory promise does not shrink freedom of choice. Since there is no 
    commitment, there is no consideration.
    \item Example: A offers to deliver up to 5,000 bushels of wheat at 
    \$2/bushel to B within 30 days. B has not promised to buy any wheat.
    \item Another example: A offers to buy 100 pounds of potatoes a month from 
    B, but only if A is in the mood for potatoes that month. A has not made a 
    commitment, so the promise is not enforceable.
    \item \textbf{The illusory promise rule has mostly 
    disappeared.}\footnote{Casebook p. 98.} The illusory promise rule is an 
    implication of the mutuality requirement. If one party can freely abandon 
    the agreement, there is no mutuality and therefore no consideration. 
    Today, courts only find a lack of mutuality if one party has 
    \emph{complete} discretion to abandon the contract, which is rare. In 
    addition, we require parties to perform in good faith. So, the illusory 
    promise rule has been confined to a very narrow range of scenarios.
\end{enumerate}

\paragraph{Unequal Terms: \emph{Lindner v. Mid-Continent Petroleum Corp.}}

As long as there has been sufficient consideration, unequal terms do not 
invalidate a contract.

\begin{enumerate}
    \item Linder leased a filling station to Mid-Continent for three years. 
    Mid-Continent retained the right to terminate the lease at any time with 
    ten days' notice. Lindner tried to cancel the lease on the ground that it 
    lacked mutuality. The court upheld the contract on the ground that ten 
    days' minimum rent constituted consideration, and that the unequal terms 
    of the contract did not invalidate the agreement.
\end{enumerate}

\paragraph{Reluctant Seller, Eager Buyer: \emph{Gurfein v. Werbelovsky}}

\begin{enumerate}
    \item A contract for the order for glass plates allowed the buyer to 
    cancel at any time. The buyer repeatedly requested performance. The seller 
    tried to get out of the contract, arguing that it was void because the 
    buyer's promise was illusory. The court held that there was consideration 
    because the seller could have bound the buyer by immediately shipping the 
    order when it was placed.
\end{enumerate}

\paragraph{Satisfaction Clauses: \emph{Mattei v. Hopper}}
~\\\\
Satisfaction clauses constitute consideration when courts read them as 
requiring good faith. Otherwise, the satisfaction clause would not involve a 
commitment, so the promise would be illusory.

\begin{enumerate}
    \item The plaintiff, a real estate developer, offered \$57,000 to buy the 
    defendant's land. The defendant accepted. The developer included a clause 
    in the contract that stipulated that the agreement was subject to 
    ``obtaining leases satisfactory to the purchaser.'' It wanted to make sure 
    it could find a tenant before it bought the land.
    \item The defendant then tried to back out of the deal. The plaintiff 
    indicated it had obtained leases and offered to pay the balance of the 
    full purchase price. Defendant declined. The plaintiff developer sued for 
    damages, and the defendant landowner argued that the plaintiff's promise 
    was illusory. The court held for the plaintiff. Such ``satisfaction'' 
    clauses are valid because the promisor's ``duty to exercise his judgment 
    in good faith is an adequate consideration to support the 
    contract~.~.~.~''\footnote{Casebook p. 100.}
\end{enumerate}

\paragraph{No Equivalence Required for Consideration: \emph{Harris v. Time, 
Inc.}}
~\\\\
Any bargained-for act or forbearance constitutes consideration. There is no 
mutuality requirement.

\begin{enumerate}
    \item The plaintiff received a letter from Time magazine offering a free 
    calculator watch if he opened the envelope. He did, and discovered he 
    could get the watch only if he also subscribed to Fortune magazine. He 
    brought a class action suit against Time, alleging that the envelope's 
    promise was a binding agreement. The court agreed, holding ``\emph{any} 
    bargained-for act or forbearance will constitute adequate consideration 
    for a unilateral contract~.~.~.~Courts will not require equivalence in the 
    values exchanged or otherwise question the adequacy of the 
    consideration.''\footnote{Casebook p. 101.}
    \item However, the court dismissed the action on the basis of \emph{de 
    minimis non curat lex.}
\end{enumerate}

\paragraph{``Primitive Stage of Formalism'': \emph{Wood v. Lucy, Lady 
Duff-Gordon}}
~\\\\
The intent of the parties outweighs ``primitive formalism.''

\begin{enumerate}
    \item Lady Duff-Gordon gave the plaintiff the exclusive to sell her 
    endorsements to put on others' designs. He brought suit because she put 
    her endorsement on other products without his knowledge and withheld 
    profits.
    \item Lady Duff-Gordon argued that the contract was invalid because it did 
    not bind the plaintiff to anything.
    \item Cardozo: ``The law has outgrown is primitive stage of formalism when 
    the precise word was the sovereign talisman and every slip was 
    fatal.''\footnote{Casebook p. 102.} The clear intent of the parties was to 
    give the plaintiff an exclusive right. The defendant cannot escape that 
    right on a formal technicality.
\end{enumerate}

\paragraph{UCC \S\ 2-306: Output, Requirements, and Exclusive Dealings}

\begin{enumerate}
    \item Terms dealing with expected output, requirements, and exclusive 
    dealings must be followed in good faith.
\end{enumerate}

\paragraph{Requirements and Output Contracts}

\begin{enumerate}
    \item \emph{Requirements contract}: a seller agrees to provide as much of 
    a good as the buyer requires.
    \item \emph{Output contract}: a buyer agrees to buy all of a seller's 
    output.
    \item Traditionally, ``courts often refused to enforce requirements 
    contracts where the buyer could choose to have no 
    requirements.''\footnote{Casebook p. 103.} But both parties shrunk their 
    realm of choice, so there was mutual consideration.
    \item UCC \S\ 2-306 requires good faith in requirements contracts.
    \item ``~.~.~.~a modern court would almost certainly hold that all 
    requirements and output contracts have consideration.''
\end{enumerate}

\subsubsection{Legal Duty, Modification, and Waiver}

\begin{enumerate}
    \item Promises to perform an act that the promisor was already obligated 
    to do are unenforceable.\footnote{Casebook p. 107.}
    \item Most companies say they wouldn't enforce strict compliance if the 
    other party requested modification of the contract.\footnote{Casebook pp. 
    126--27.}
    \item Initial contracts require consideration, but changes to existing 
    agreements do not. \emph{Watkins \& Son v. Carrig.}\footnote{Casebook pp. 
    141--42.}
\end{enumerate}

\paragraph{No Enforcement for Legal Duty: \emph{Slattery v. Wells Fargo 
Armored Serv. Corp.}}

``The performance of an existing duty does not amount to the consideration 
necessary to support a contract.''\footnote{Casebook p. 108.}

\begin{enumerate}
    \item Wells Fargo offered a \$25,000 reward for information on a robbery 
    and shooting. A polygraph operator discovered the perpetrator while 
    questioning him on an unrelated matter. The court denied him the reward 
    because he ``was under a pre-existing duty to furnish his employers with 
    all useful information revealed to him through interrogation of the 
    perpetrator.
\end{enumerate}

\paragraph{\emph{Shadwell v. Shadwell}}

\begin{enumerate}
    \item Affirming the unenforceability of legal duty as consideration.
\end{enumerate}

\paragraph{Restatement Second \S\ 73: Performance of Legal Duty}

\begin{enumerate}
    \item Performance of a legal duty is not consideration.
\end{enumerate}

\paragraph{N.Y. Penal Law \S\S\ 200.30, 200.35}

\begin{enumerate}
    \item It is illegal to give gratuities for a public servant, and for a 
    public servant to take them
\end{enumerate}

\paragraph{Thin Line Between Legal Duty and No Duty: \emph{Denney v. Reppert}}
~\\\\
A sheriff's deputy out of his jurisdiction was not under a legal duty to make 
an arrest.

\begin{enumerate}
    \item A bank association offered a reward for the apprehension of a bank 
    robber. Several bank employees and state policemen were ineligible because 
    they were performing their legal duty in apprehending the criminal. 
    However, a sheriff's deputy could get the reward because he made the 
    arrest while out of his jurisdiction.
\end{enumerate}

\paragraph{Extortion is not Consideration: \emph{Lingenfelder v. Wainwright 
Brewery Co.}}

\begin{enumerate}
    \item Lingenfelder was the executor of the estate of Jungenfeld, who had 
    contracted as an architect for the Wainwright Brewery Co. When Jungenfeld 
    learned that Wainwright had awarded the refrigeration contract to another 
    company, Jungenfeld called off his work. The brewery agreed to pay him 
    five percent of the cost of the competitor's refrigeration machine if 
    Jungenfeld would complete his contracted work.
    \item Wainwright argued that Jungenfeld's actions amounted to extortion. 
    The court held that Jungenfeld was already under a legal obligation to 
    finish the job. Performing his legal duty did not constitute 
    consideration.
\end{enumerate}

\paragraph{Extortion and Legal Duty: \emph{Austin Instrument, Inc. v. Loral 
Corp.}}
~\\\\
The legal duty rule did not prevent Austin from extorting higher prices from Loral, 
but the court held for Loral under the theory of duress.

\begin{enumerate}
    \item Loral contracted to provide radar equipment to the Navy. Loral 
    subcontracted with Austin to provide components. Before an order had been 
    filled, Austin stopped work and demanded a price increase. Austin 
    consented, but wrote a letter saying it did so only under duress.
    \item The New York Supreme Court, appellate division, held that Austin's 
    price increases were in good faith and did not constitute duress.
    \item The New York Court of Appeals reversed, holding that this was a 
    ``classic case'' of duress.\footnote{Casebook p. 121.} The possibility of 
    defaulting on its Navy contract posed a serious threat to Loral. 
\end{enumerate}

\paragraph{More on the Legal Duty Rule}

\begin{enumerate}
    \item The legal duty rule is based on a \emph{static} view of contract, 
    because the terms of a contract are entirely fixed when parties make the 
    agreement.
    \item A regime under which modifications are enforceable reflects a 
    \emph{dynamic} view of contract, in which contracts are constantly 
    evolving processes.
\end{enumerate}

\paragraph{Substitute Contracts: \emph{Schwartzreich v. Bauman-Basch, Inc.}}
~\\\\
If parties mutually agree to a new contract, the new contract is enforceable 
and the old contract is voided.

\begin{enumerate}
    \item The plaintiff contracted to work for the defendant at \$90/week. He 
    got another offer for \$100/week. He and the defendant agreed to sign a 
    new contract for \$100/week. The defendant argued that the new contract 
    was void, but the court held that as long as the parties mutually agreed 
    to enter into the new contract, it was enforceable.
\end{enumerate}

\paragraph{Restatement First \S\ 406, Illustration 1}

\begin{enumerate}
    \item In a bilateral contract, if ``each party surrenders something he 
    might have retained,'' there is consideration for the substitute contract.
\end{enumerate}

\paragraph{UCC \S\S\ 3-103(a)(4), 3-104, 3-311}

\begin{enumerate}
    \item 3-103(a)(4): definition of good faith.
    \item 3-104: definition of a negotiable instrument.
    \item 3-311: satisfaction of claims by instrument.
\end{enumerate}

\paragraph{Restatement Second \S\ 279: Substituted Contract}

\begin{enumerate}
    \item The substituted contract replaces the original contract.
\end{enumerate}

\paragraph{Restatement Second \S\ 281: Accord and Satisfaction}

\begin{enumerate}
    \item \emph{Accord}: ``a contract under which an obligee promises to 
    accept a stated performance in satisfaction of the obligor's existing 
    duty.''
\end{enumerate}

\paragraph{The Legal Effect of the ``Executory Accord''}

\begin{enumerate}
    \item \emph{Accord}: an agreement under which a new obligation replaces an 
    existing obligation.
    \item \emph{Executory accord}: an unperformed accord.
    \item \emph{Satisfaction}: performance of the accord.
    \item Historically, the executory accord was unenforceable---``one of the 
    major mysteries of the common law.''\footnote{Casebook p. 135.} The common 
    law rule was ``inconsistent with the bargain principle, basically 
    inexplicable, and wrong.''\footnote{Casebook p. 136.}
    \item \textbf{Substituted contract}: the earlier agreement is immediately 
    discharged.
    \item Courts are likely to find that an accord is a substituted contract 
    if the original duty ``was disputed, unliquidated, had not matured, and 
    involved a performance other than the payment of 
    money.''\footnote{Casebook pp. 136--37.} For instance, for agricultural 
    services, A agreed to pay B in cattle. They disagreed over how many cattle 
    were due. They agreed that A would pay three sheep instead. Courts will 
    likely treat the accord as a substituted contract.
    \item Courts are likely to find that an accord is \emph{not} a substituted 
    contract if the original duty ``was undisputed, liquidated, had matured, 
    and involved the payment of money.''\footnote{Casebook p. 137.} For 
    instance, A agreed to pay \$900 for B's services, and A does not dispute 
    B's claim. They agree that A will give two sheep instead of paying \$900. 
    Courts will likely \emph{not} find that the accord is a substituted 
    contract.
    \item Other rules:
    \begin{enumerate}
        \item \emph{Performance}: if an accord involving a different 
        performance is executed, the original contractual obligations are 
        discharged.
        \item \emph{Suspension of old contract}: ``an executory accord 
        operates to suspend A's rights under the original contract during the 
        period in which B is supposed to perform the 
        accord.''\footnote{Casebook p. 137.}
        \item If one party fails to perform under the accord, the other party 
        can sue under either the new accord or the old contract.
    \end{enumerate}
    \item ``~.~.~.~in many or most jurisdictions the traditional rule 
    concerning accord and satisfaction has been substantially or almost 
    completely eroded.''\footnote{Casebook p. 138.}
\end{enumerate}

\paragraph{Restatement Second \S\ 89: Modification of Executory Contract}

\begin{enumerate}
    \item A promise modifying an earlier contract is binding when:
    \begin{enumerate}
        \item It's fair;
        \item Provided by statute; or
        \item ``justice requires enforcement'' if a party relies on it.
    \end{enumerate}
\end{enumerate}

\paragraph{No Consideration for Modification: \emph{Angel v. Murray}}
~\\\\
Modifications do not require consideration as long as the parties agree 
voluntarily.

\begin{enumerate}
    \item Maher had a series of five-year contracts with the city for garbage 
    collection. In 1967, he requested an additional \$10,000 because the 
    city's population had grown unexpectedly fast. The council agreed. They 
    also granted an extra \$10,000 for the following year.
    \item Angel, a Newport resident, brought suit to recover the extra 
    \$20,000.
    \item The trial court found that Maher was not entitled to the extra 
    \$20,000 because the contract required him to collect all refuse within 
    the city, regardless of population increases---i.e., Maher had a 
    preexisting duty to collect all of the city's garbage.
    \item ``It is generally held that a modification of a contract is itself a 
    contract, which is unenforceable unless supported by 
    consideration.''\footnote{Casebook p. 139.}
    \item ``The modern trend appears to be to recognize the necessity that 
    courts should enforce agreements modifying contracts when unexpected or 
    unanticipated difficulties arise during the court of the performance of a 
    contract, even though there is no consideration for the modification, as 
    long as the parties agree voluntarily.''\footnote{Casebook p. 140.}
    \item In this case, since the modification was fair and voluntary, the 
    agreement was enforceable. Reversed.
\end{enumerate}

\paragraph{UCC \S\ 2-209: Modification, Rescission, and Waiver}

\begin{enumerate}
    \item Modifications do not require consideration.
\end{enumerate}

\paragraph{CISG Art. 29}

\begin{enumerate}
    \item CISG governs the sale of international goods.\footnote{UN Convention on Contracts for the International Sale 
    of Goods. See supplement p. 358.} 
    \item Art. 29: modification requires only agreement. Exchange is not 
    required.
\end{enumerate}

\paragraph{Waiver: \emph{Clark v. West}}
~\\\\
Clauses are waivable.

\begin{enumerate}
    \item Plaintiff contracted to write law books to be published by defendant 
    (West). The contract included a clause that gave the plaintiff \$4 per 
    page, but only \$2 per page if he did not abstain from ``intoxicating 
    liquors.''\footnote{Casebook p. 144.} Plaintiff did not abstain but 
    otherwise delivered on the contract. 
    \item Plaintiff argued that West knew about his drinking but did nothing 
    about it, which amounted to a waiver of that part of the contract.
    \item The court held that the abstinence clause was not consideration for 
    the contract. Rather, it was a condition precedent. ``It is not a contract 
    to write books in order that the plaintiff shall keep sober, but a 
    contract maintaining a stipulation that he shall keep sober so that he may 
    write satisfactory books.''\footnote{Casebook p. 146.}
    \item The court held that the clause was waivable as a matter of law. It 
    remanded it to the lower court to determine whether it was in fact waived.
\end{enumerate}

\paragraph{Restatement Second \S\ 84: Promise to Perform a Duty in Spite of 
Non-Occurrence of a Condition}

\begin{enumerate}
    \item A party can waive conditions of a contract. For instance, A hires B 
    to build a house. As part of the agreement, A will only pay after A's 
    architect, C, has signed off. C refuses to approve because of trivial 
    defects. A tells B he'll pay him anyway. A has waived the condition and 
    his promise is binding.
\end{enumerate}

\paragraph{\emph{Nassau Trust Co. v. Montrose Concrete Prods. Corp.}}

\begin{enumerate}
    \item Modification requires consideration, and is therefore binding. 
    Waiver does not require consideration, so it is not binding until 
    executed.
    \item Can this be reconciled with \emph{Angel v. Murray}, above?
\end{enumerate}

\paragraph{Waiver under the UCC: \emph{BMC Indus., Inc. v. Barth Indus., 
Inc.}}

\begin{enumerate}
    \item ``[T]he UCC does not require consideration or detrimental reliance 
    for waiver of a contract term.''\footnote{Casebook p. 149.}
\end{enumerate}

\subsection{Past Consideration}

\subsubsection{Restatement Second \S\ 82: Promise to Pay Indebtedness; Effect 
on the Statute of Limitations}

\begin{enumerate}
    \item A promise to pay earlier contractual indebtedness is binding, even 
    if the earlier debt would have been barred by the statute of limitations.
\end{enumerate}

\subsubsection{Restatement Second \S\ 83: Promise to Pay Indebtedness 
Discharged in Bankruptcy}

\begin{enumerate}
    \item If a debt was discharged during bankruptcy, a later promise to pay 
    it is binding.
\end{enumerate}

\subsubsection{Three Situations in Which a Promise to Discharge an Unenforceable 
Obligation is Binding}

\begin{enumerate}
    \item Traditionally, there were three cases when a promise to pay based on 
    a past event, rather than a present bargain, were 
    enforceable:\footnote{Casebook p. 151.}
    \begin{enumerate}
        \item A promise to pay a debt barred by the statute of limitations.
        \item A promise by an adult to pay a debt incurred when the person was 
        underage.
        \item A promise to pay a debt that has been discharged in bankruptcy.
    \end{enumerate}
    \item Explanations:
    \begin{enumerate}
        \item Infants and debtors have moral obligations to pay their debts. 
        When they recognize this moral obligation by making a promise, the 
        promise becomes binding.
        \item The creditor actually had a legal claim, but the debtor had a 
        valid defense (infancy or bankruptcy). When the debtor makes a 
        promise to pay the debt, he waives his defense.
        \begin{enumerate}
            \item Why not require consideration for the promise to pay the 
            debt?
        \end{enumerate}
    \end{enumerate}
\end{enumerate}

\subsubsection{No Past Consideration: \emph{Mills v. Wyman}}

Moral obligations do not make promises enforceable. Past consideration is 
required.

\begin{enumerate}
    \item Levy Wyman fell ill in Hartford on return from a voyage from sea. 
    The plaintiff, Daniel Mills, housed him at his expense for 15 days, at 
    which point Wyman died. Wyman's father promised to pay Mills's expenses.
    \item When Wyman's father failed to pay, Mills sued. The trial court 
    directed a nonsuit.
    \item There was no consideration in the defendant's promise to pay Mills. 
    Moral obligation is insufficient without consideraion. ``~.~.~.~if there 
    was nothing paid or promised for it, the law, perhaps wisely, leaves the 
    execution of it to the conscience of him who makes it. It is only when the 
    party making the promise gains something, or he to whom it is made loses 
    something, that the law gives the promise validity.''\footnote{Casebook p. 
    154.}
\end{enumerate}

\subsubsection{Protection as Consideration: \emph{Webb v. McGowin}}

Protecting McGowin from injury was sufficient consideration for McGowin's 
later promise to pay Webb to be binding.

\begin{enumerate}
    \item Webb worked in a lumber company. He was pushing heavy wooden blocks 
    off a ledge. As he was pushing a block, he noticed that McGowin was below. 
    The only way he could prevent the block from crushing McGowin was to fall 
    with it, so he did, and sustained serious injuries and suffered permanent 
    disabilities.
    \item McGowin promised to pay Webb \$15 every two weeks for the rest of 
    Webb's life, until McGowin died a few years later. McGowin's executors 
    refused to keep paying. Webb brought suit.
    \item ``Where the promisee cares for, improves, and preserves the property 
    of the promisor, though done without his request, it is sufficient 
    consideration for the promisor's subsequent agreement to pay for the 
    service, because of the material benefit received.''\footnote{Casebook p. 
    157.}
    \item McGowin's promise was therefore enforceable.
\end{enumerate}

\subsubsection{Saving from Decapitation: \emph{Harrington v. Taylor}}

Voluntary humanitarian acts are not consideration.

\begin{enumerate}
    \item The defendant assaulted his wife and she took refuge in the 
    plaintiff's house. The defendant came to the plaintiff's house. She tried 
    to decapitate him with an axe. The plaintiff caught it, saving the 
    defendant's life but badly mutilating her hand. The defendant agreed to 
    pay for the plaintiff's damages, but failed to pay.
    \item The court held that ``a humanitarian action of this kind, 
    voluntarily performed, is not such consideration as would entitle her to 
    recover at law.''\footnote{Casebook p. 160.}
\end{enumerate}

\subsubsection{Restatement Second \S\ 86: Promise for Benefit Received}

\begin{enumerate}
    \item Promises for previous benefits are ``binding to the extent necessary 
    to prevent injustice.'' But promises are not binding if they are gifts or 
    if th value of the promise is disproportionate to the benefit.
\end{enumerate}

\subsubsection{ALI 42d Annual Proceedings}

\begin{enumerate}
    \item Past consideration is \emph{sometimes} binding. Some cases are 
    ``gratuitous transactions'' (with no consideration) and 
    ``quasi-contracts'' (with consideration, if justice 
    requires).\footnote{Casebook p. 160--61.}
    \item The ALI seems to be struggling here to pin down a rule.
\end{enumerate}

\subsubsection{Note on Past Consideration}

\begin{enumerate}
    \item If A confers a benefit on B without B's prior request, the 
    subsequent relationship could fall into three categories:
    \begin{enumerate}
        \item B is legally obligated to compensate A under the law of 
        \emph{unjust enrichment}---for instance, if A paid B money by mistake.
        \item B is morally but not legally obligated to compensate A---for 
        instance, if B has suffered a loss on rescuing A. \emph{Mills v. 
        Wyman.}
        \item B is neither morally nor legally obligated to compensate 
        A---e.g., ordinary gifts.
    \end{enumerate}
    \item At common law, the general rule was that promises based on past 
    benefits were unenforceable.
    \item The Restatement Second makes promises based on past benefits 
    enforceable on the basis of unjust enrichment. Fuller and Eisenberg argue 
    that they should instead be enforceable on the basis of moral 
    obligation.\footnote{Casebook p. 162.}
\end{enumerate}

\subsection{The Limits of Contract}

\begin{enumerate}
    \item When are contracts inappropriate?
\end{enumerate}

\subsubsection{\emph{Balfour v. Balfour}}

\begin{enumerate}
    \item Agreements do not always create legally binding contracts---for 
    instance, two people agreeing to go on a walk. The parties did not intend 
    that legal consequences would attend the agreement.
\end{enumerate}

\subsubsection{In Vitro Fertilization: \emph{In Re the Marriage of Witten}}

How much judicial interference do we want in marital relations? How does 
public policy determine when contracts should be enforced?

The court here held that In vitro fertilization agreements are enforceable, 
but either party can change his or her mind up to the point of the use or 
destruction of the embryo.

Courts are generally hesitant to impose ``coercive parenting''---i.e., forcing 
someone to have a child.

\begin{enumerate}
    \item Trip and Tamera wanted a divorce. They had frozen fertilized 
    embryos. Tamera wanted to use the embryos to have a child. Trip objected.
    \item Before starting the in vitro fertilization process, Trip and Tamera 
    signed an ``Embryo Storage Agreement'' stipulating that any use or 
    transfer of the embryos required both of their consent. The one exception 
    allowed unilateral action if the other died. There was no provision for 
    divorce.
    \item Tamera raised three arguments for custody of the embryos: (1) the 
    storage agreement is silent on the possibility of divorce, (2) it would be 
    in the child's best interests, and (3) it would violate public policy to 
    allow Trip to back out of the agreement.
    \item \emph{Best interests}: no. There was no child to care for.
    \item \emph{Scope of storage agreement}: the agreement did not 
    specifically account for divorce, but release of the embryo nonetheless 
    required both parties' consent. The question was whether the agreement was 
    enforceable. The court tested three approaches:
    \begin{enumerate}
        \item \emph{Contractual approach}: contracts entered into at the time 
        of in vitro fertilization are enforceable if they are not contrary to 
        public policy. On the one hand, agreements would have little force if 
        they were not enforceable. On the other hand, parents may not know 
        their true feelings until the child is born.
        \item \emph{Contemporaneous mutual consent}: both partners should have 
        an equal say in how the embryo is used.
        \item \emph{Balancing test}: agreements are enforceable, but either 
        party can change his or her mind up to the point of use or destruction 
        of any stored embryos. In case of disagreement, the court must 
        evaluate both parties' interests. But should courts be the decision 
        makers in these kinds of situations?
    \end{enumerate}
    \item ``We think~.~.~.~it would be against the public policy of this state 
    to enforce a prior agreement between the parties in this highly personal 
    area of reproductive choice when one of the parties has changed his or her 
    mind concerning the disposition or use of the embryos.''\footnote{Casebook 
    p. 170.}
    \item Held: the agreements are enforceable, but any party can change his 
    or her mind up to the point of the use or destruction of the embryos. If 
    the parties cannot agree, the status quo remains (for up to 10 years).
\end{enumerate}

\subsubsection{Who is a Parent?}

\begin{enumerate}
    \item Domestic partners are responsible for child support.
\end{enumerate}

\subsubsection{Surrogacy Agreements: \emph{R.R. v. M.H.}}

Surrogacy agreements (in Massachusetts) are unenforceable if they involve 
money or if they require the mother to give up the child before four days 
after birth.

\begin{enumerate}
    \item A man paid a woman \$10,000 to be the surrogate mother of his child.
    \item ``Policies underlying our adoption legislation suggest that a 
    surrogate parenting agreement should be given no effect if the mother's 
    agreement was obtained prior to a reasonable time after the child's birth 
    or if her agreement was induced by the payment of 
    money.''\footnote{Casebook p. 183.} An agreement that violates these 
    policies is unenforceable.
    \item ``A surrogacy agreement judicially approved before conception may be 
    a better procedure~.~.~.~''\footnote{Casebook p. 185.}
\end{enumerate}

\subsubsection{Surrogate-Parenting Legislation}

\begin{enumerate}
    \item Seventeen states have surrogacy legislation. The statutes vary 
    widely.
\end{enumerate}

\subsubsection{42 U.S. Code \S\ 274e: Prohibition of Organ Purchases}

\begin{enumerate}
    \item Human organs cannot be transferred for ``valuable 
    consideration.''\footnote{Casebook p. 185.}
\end{enumerate}

\subsubsection{Radin, ``Market-Inalienability''}

\begin{enumerate}
    \item Radin is responding to Posner's argument that any bureaucratic 
    solution will be flawed, so we should allow organs to be sold on the open 
    market.
    \item ``In precluding sales but not gifts, market-inalienability places 
    some things outside the marketplace but not outside the realm of social 
    intercourse.''\footnote{Casebook p. 186.} Some things shouldn't be 
    commodified---children, organs, sexual services, and so on.
\end{enumerate}
