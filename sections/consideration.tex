\section{Consideration}

\subsection{Donative Promises, Form, and Reliance}

\subsubsection{Simple Donative Promises}

Made for affective reasons, usually informal, and not demonstrably relied 
upon.\footnote{Casebook p. 6.}

\paragraph{\emph{Dougherty v. Salt}}

\begin{enumerate}
    \item A boy's aunt gave him a promissory note for \$3,000, payable at her 
    death or before.
    \item The jury found for the plaintiff. The trial judge set aside the 
    verdict and dismissed the complaint. The appellate court reversed and 
    reinstated.
    \item The New York Court of Appeals (Cardozo) reversed, holding that the 
    note was ``the voluntary and unenforceable promise of an executory 
    gift.''\footnote{Casebook p. 7.}
\end{enumerate}

\paragraph{Restatement Second \S\ 1: Definition}

\begin{enumerate}
    \item ``A contract is a promise or a set of promises for the breach of 
    which the law gives a remedy, or the performance of which the law in some 
    way recognizes as a duty.''
\end{enumerate}

\paragraph{Restatement Second \S\ 17: Requirement of a Bargain}

\begin{enumerate}
    \item ``~.~.~.~a contract requires a bargain in which there is a 
    manifestation of mutual assent to the exchange and a consideration.''
\end{enumerate}

\paragraph{Restatement Second \S\ 71: Requirement of Exchange; Types of 
Exchange}

\begin{itemize}
    \item (1) Consideration must be bargained for. 
    \item (2) A performance or return promise is ``bargained for'' if sought and given in exchange.
    \item (3) Performance can be an act, forbearance, or creation, 
    modification, or destruction of a legal right.
    \item (4) Performance may be given to another person.
\end{itemize}

\paragraph{Restatement Second \S\ 79: Adequacy of Consideration; Mutuality 
of Obligation}

\begin{enumerate}
    \item If the consideration requirement is met, there are no requirements 
    of gain/loss, equivalence in value, or ``mutuality of obligation.''
\end{enumerate}

\paragraph{On the Restatement Second}

\begin{enumerate}
    \item Samuel Williston, Reporter for the Restatement First: contract law 
    is ``a set of axioms that were deemed to be self-evident, together with a 
    set of subsidiary rules that were purportedly deduced from the 
    axioms.''\footnote{Casebook p. 8.}
    \item Arthur Corbin, Consultant to the Restatement Second: father of 
    ``modern contract law.''\footnote{Casebook p. 8.}
\end{enumerate}

\paragraph{Consideration}

\begin{enumerate}
    \item Two conceptions: broad and narrow.
    \item \emph{Broad}: ``consideration'' refers collectively to the things 
    that make contracts legally enforceable---e.g., bargain or 
    reliance.\footnote{Casebook p. 8.}
    \item \emph{Narrow}: ``consideration'' is the same thing as ``bargain.'' 
    The Restatement Second adopts this approach, known as the \textbf{bargain 
    theory of consideration}.
    \item The bargain theory of consideration creates two kinds of distortion:
    \begin{enumerate}
        \item \emph{Terminological}: many other elements besides bargain can 
        make a contract enforceable. Therefore, ``under the terminology of the 
        Restatement Second, a promise needs consideration to be enforceable 
        unless it does not need consideration to be 
        enforceable.''\footnote{Casebook p. 9.}
        \item \emph{Substantive}: the bargain theory presumes that all 
        nonbargain promises are unenforceable. It does not allow the law to 
        develop other means of making promises enforceable.
    \end{enumerate}
\end{enumerate}

\paragraph{Gifts}

\begin{enumerate}
    \item The common law distinguishes between gifts and promises to make 
    gifts. A promise to make a gift is not enforceable. \emph{Dougherty}.
    \item A \textbf{deed of gift} (or \textbf{inter vivos document of 
    transfer}) transfers ownership via a written instrument.\footnote{Casebook 
    pp. 9--10.}
    \item Another way to make a gift is for the owner to declare herself a 
    \textbf{trustee} of her property for the benefit of another. The trustee 
    retains legal title but the beneficiary receives beneficial 
    ownership.\footnote{Casebook p. 10.}
\end{enumerate}

\paragraph{Donative Promises}

\begin{enumerate}
    \item Why make a donative promise, rather than wait and give the gift 
    later?
    \begin{enumerate}
        \item Instincts vary. The promisor might think she has better judgment 
        right now than she will in the future.
        \item The promisor wants to derive satisfaction from the promisee's 
        gratitude.
        \item \textbf{Beneficial reliance}: the promisee can rely on the 
        promise to her benefit---e.g., knowing that her Aunt will pay for 
        college, a niece decides to finish high school, rather than look for a 
        paying job.
    \end{enumerate}
    \item The ``basic fault line'' in classical contract law was between 
    bargain promises (enforceable) and gratuitous promises (unenforceable).
    \item Should contract law be based on the moral belief that breaking 
    promises is unethical, or should it promote utilitarian goals like 
    ''compensating injured promisees and increasing social 
    wealth''?\footnote{Casebook p. 11.}
    \item Lon Fuller introduced \textbf{substantive} and \textbf{process} 
    bases for enforcing promises.
    \begin{enumerate}
        \item The two process bases are evidentiary (making sure a promise 
        has actually been made) and cautionary (preventing inconsiderate 
        action by the promisor).
        \item Simple donative promises are problematic on both bases. They 
        raise problems of proof (evidence may be uncertain) and 
        deliberativeness (the promisor is likely to be emotionally involved 
        with the promisee).
        \item Substantive reasons for enforcing donative promises include 
        compensating the promisee's disappointment (a form of injury), move 
        assets from the wealthy to the less wealthy, and increase the 
        likelihood of beneficial reliance. But these substantive bases are 
        open to question.
    \end{enumerate}
\end{enumerate}

\paragraph{Conditional Donative Promises}

\begin{enumerate}
    \item In a bargain promise, the condition is the \textbf{price} of the 
    promise---e.g., I'll give you \$20 if you mow my lawn.
    \item In a donative promise, the condition is the \textbf{means} to make 
    the gift---e.g., I'll buy you a car if you pick one that costs less than 
    \$15,000. 
    \item We have to rely on a ``reasonable interpretation'' to determine 
    whether a condition is a means or a price.\footnote{Casebook p. 13.}
\end{enumerate}

%\subsubsection{The Element of Form}
% 
% \subsubsection{Von Mehren, ``Civil-Law Analogues to Consideration: An Exercise 
% in Comparative Analysis''}
% 
% \begin{enumerate}
%     \item % TODO 13-14
% \end{enumerate}
% 
% \subsubsection{Channeling Function of Contract-Law Rules}
% 
% \begin{enumerate}
%     \item % TODO 
% \end{enumerate}
% 
% \subsubsection{\emph{Schnell v. Nell}}
% 
% \begin{enumerate}
%     \item % TODO 14-16
% \end{enumerate}
% 
% \subsubsection{Form}
% 
% \begin{enumerate}
%     \item % TODO 21-23
% \end{enumerate}
% 
% \subsubsection{Reliance}
% 
% \begin{enumerate}
%     \item % TODO 23-24
% \end{enumerate}
% 
% \subsubsection{\emph{Kirksey v. Kirksey}}
% 
% \begin{enumerate}
%     \item % TODO 24-25
% \end{enumerate}
% 
% \subsubsection{Restatement Second \S\ 90}
% 
% \begin{enumerate}
%     \item % TODO supplement
% \end{enumerate}
% 
% \subsubsection{Estoppel in Pais and Promissory Estoppel}
% 
% \begin{enumerate}
%     \item % TODO 25-28
% \end{enumerate}
% 
% % TODO: hypo pp. 28-29
% 
% \subsubsection{\emph{Feinberg v. Pfeiffer}}
% 
% \begin{enumerate}
%     \item % TODO 29-34
% \end{enumerate}
% 
% \subsubsection{\emph{Hayes v. Plantations Steel Co.}}
% 
% \begin{enumerate}
%     \item % TODO 34-35
% \end{enumerate}
% 
% \subsubsection{Remedies and Consideration}
% 
% \begin{enumerate}
%     \item % TODO 35-39
% \end{enumerate}
% 
% \subsubsection{\emph{Goldstick v. ICM Realty}}
% 
% \begin{enumerate}
%     \item % TODO 39-40
% \end{enumerate}
% 
% \subsubsection{\emph{D \& G Stout, Inc. v. Bacardi Imports, Inc.}}
% 
% \begin{enumerate}
%     \item % TODO 40-45 + note p. 45
% \end{enumerate}
% 
% \subsubsection{\emph{Walters v. Marathon Oil Co.}}
% 
% \begin{enumerate}
%     \item % TODO 45-46
% \end{enumerate}
% 
% \subsection{The Bargain Principle}
% 
% \subsubsection{\emph{Westlake v. Adams, C.P.}}
% 
% \begin{enumerate}
%     \item % TODO 47
% \end{enumerate}
% 
% \subsubsection{Hamer v. Sidway}
% 
% \begin{enumerate}
%     \item % TODO 47-49
% \end{enumerate}
% 
% \subsubsection{State of Frauds}
% 
% \begin{enumerate}
%     \item % TODO 49
% \end{enumerate}
% 
% \subsubsection{\emph{Davies v. Martel Laboratory Services, Inc.}}
% 
% \begin{enumerate}
%     \item % TODO 49-50
% \end{enumerate}
% 
% \subsubsection{\emph{Hancock Bank \& Trust Co. v. Shell Oil Co.}}
% 
% \begin{enumerate}
%     \item % TODO 50-51
% \end{enumerate}
% 
% \subsubsection{Restatement Second \S\ 71}
% 
% \begin{enumerate}
%     \item % TODO supplement
% \end{enumerate}
% 
% \subsubsection{Restatement Second \S\ 72}
% 
% \begin{enumerate}
%     \item % TODO supplement
% \end{enumerate}
% 
% \subsubsection{Restatement Second \S\ 79}
% 
% \begin{enumerate}
%     \item % TODO supplement
% \end{enumerate}
% 
% \subsubsection{\emph{Batsakis v. Demotsis}}
% 
% \begin{enumerate}
%     \item % TODO 51-54
% \end{enumerate}
% 
% \subsubsection{Sweet-Escott, Greece---A Political and Economic Survey, 
% 1939--1953}
% 
% \begin{enumerate}
%     \item % TODO 54
% \end{enumerate}
% 
% \subsubsection{Consideration and Remedies}
% 
% \begin{enumerate}
%     \item % TODO 54-55
% \end{enumerate}
% 
% \subsubsection{Restatement Second \S\ 175}
% 
% \begin{enumerate}
%     \item % TODO supplement
% \end{enumerate}
% 
% \subsubsection{Restatement Second \S\ 176}
% 
% \begin{enumerate}
%     \item % TODO supplement
%     \item % TODO illustration 16, p. 55
% \end{enumerate}
% 
% \subsubsection{Unidroit Principles of International Commercial Contracts Art. 
% 3.9}
% 
% \begin{enumerate}
%     \item % TODO supplement
%     \item % TODO: note on unidroit -- 55-56
% \end{enumerate}
% 
% \subsubsection{Principles of European Contract Law \S\ 4.108}
% 
% \begin{enumerate}
%     \item % TODO supplement
%     \item % TODO note -- 56
% \end{enumerate}
% 
% \subsubsection{\emph{Chouinard v. Chouinard}}
% 
% \begin{enumerate}
%     \item % TODO 56-57
% \end{enumerate}
% 
% \subsubsection{\emph{Post v. Jones}}
% 
% \begin{enumerate}
%     \item % TODO 57-58
% \end{enumerate}
% 
% \subsubsection{Eisenberg, ``The Bargain Principle and Its Limits''}
% 
% \begin{enumerate}
%     \item % TODO 58-60
% \end{enumerate}
% 
% \subsubsection{New York Gen. Bus. Law \S\ 396-r: Price Gouging}
% 
% \begin{enumerate}
%     \item % TODO 60-62
% \end{enumerate}
% 
% \subsubsection{\emph{People v. Two Wheel Corp.}}
% 
% \begin{enumerate}
%     \item % TODO 62-63
% \end{enumerate}
% 
% \subsubsection{\emph{Williams v. Walker-Thomas Furniture Co.}}
% 
% \begin{enumerate}
%     \item % TODO 63-68
% \end{enumerate}
% 
% \subsubsection{Skilton \& Helstad, ``Protection of the Installment Buyer of 
% Goods under the Uniform Commercial Code''}
% 
% \begin{enumerate}
%     \item % TODO 68-69
% \end{enumerate}
% 
% \subsubsection{Pricing by Low-Income Market Retailers}
% 
% \begin{enumerate}
%     \item % TODO 69-70
% \end{enumerate}
