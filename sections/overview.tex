\section{Overview}

\subsection{Consideration}

% TODO

\subsection{Remedies}

\subsubsection{Overview}

\begin{enumerate}
    \item \textbf{Expectation}: puts promisee in the position he would have 
    been in if the contract had been \emph{performed}. You get what you 
    bargained for.
    \item \textbf{Reliance}: puts promisee in the position he would have been 
    in 
    if the contract had \emph{not been made}.
    \item \textbf{Restitution}: restores to the promisee any benefit he 
    conferred to the promisor.
    \item Expectation damages are the difference between promised performance 
    and actual performance---for instance, between a perfect hand and a hairy 
    hand. \emph{Hawkins v. McGee}.
    \item Damages compensate a party for actual losses. No punitive damages. 
    \emph{U.S. Naval Inst. v.  Charter Commc'ns, Inc.}
    \item Classical contract law did not permit a seller to profit from 
    breach. \emph{Coppola Enters., Inc. v. Alfone}. Some courts viewed it as a 
    breach of the covenant of good faith and fair dealing. \emph{Greer Props., 
    Inc. v. LaSalle Nat'l Bank}.
    \item Posner argues for efficient breach. Eisenberg responds that 
    efficient breach is ultimately not so efficient.
\end{enumerate}

\subsubsection{Expectation Damages}

\begin{enumerate}
    \item \textbf{Breach of contract to perform services}:
    \begin{enumerate}
        \item If a contractor leaves a job unfinished, the client can recover 
        \textbf{cost of completion} standards. \emph{Louise Caroline Nursing 
        Home, Inc. v. Dix Construction Co.}
        \item Where cost of performance exceeds \textbf{diminution in value}, 
        courts are likely to award damages only for diminution in value.  
        \emph{Peevyhouse v. Garland Coal \& Mining Co.} Courts are wary of 
        awarding cost of completion damages when the cost is \textbf{grossly 
        disproportionate} to the benefits (\emph{H.P. Droher \& Sons v. 
        Toushin}) or when it would cause waste (\emph{Eastern Steamship Lines, 
        Inc. v.  United States}).
        \item Courts are more likely to award cost of completion damages when 
        the defects are aesthetically important (e.g., a fancy swimming 
        pool---\emph{City School Dist. of the City of Elmira v.  McLane Const.  
        Co.}) or the project involves someone's home (\emph{Fox v. Webb}).  
        \item Calculating damages for partially performed services:
        \begin{enumerate}
            \item Contract price - value of performance completed - amount 
            paid by owner prior to breach. Restatement Second \S\ 346.
            \item Example: A contracts to build a house for B for \$100,000. B 
            breaches halfway through. A would have to spend \$60,000 to 
            finish.  A's damages are the contract price (\$100,000) - cost of 
            completion that the contractor saved (\$60,000) = \$40,000. A can 
            recover \$40,000 minus any payments already made.
        \end{enumerate}
    \end{enumerate}
    \item \textbf{Breach of a contract for the sale of goods}:
    \begin{enumerate}
        \item The UCC governs sale of goods.
        \item \textbf{Seller's breach}:
        \begin{enumerate}
            \item Two remedies: \textbf{specific relief} and \textbf{damages}.
            \item For defective goods, courts might award damages for the 
            \textbf{cost of repair} in excess of the purchase price.  UCC \S\ 
            2-714(2) sets the measure of direct damages for breach of warranty 
            as the difference between the value of the goods as warranted and 
            the value of the goods as accepted, often approximated as the cost 
            of repair.
            \item When the seller fails to deliver goods sold, the buyer has 
            two options:
            \begin{enumerate}
                \item \emph{Cover}: the buyer purchases a replacement. Damages 
                are the difference between the replacement price and the 
                contract price. UCC \S\ 7-212 and \emph{KGM Harvesting Co. v. 
                Fresh Network.}
                \item \emph{Hypothetical cover} or \textbf{market price} 
                damages: the buyer recovers the difference between the market 
                price at the time of the breach and the contract price. UCC 
                \S\ 7-213.
            \end{enumerate}
            \item When a buyer covers, can he still recover market price 
            damages under UCC \S\ 7-213, rather than the actual cost of cover 
            under \S\ 2-712?
            \begin{enumerate}
                \item Example: a buyer agrees to buy a steamroller for 
                \$1,000. The market price for steamrollers is \$1,300 at the 
                time of the breach. The buyer waits six months to cover, when 
                the market price has dropped to \$500. Under \S\ 2-712, he can 
                recover nothing, but under 2-713, he can recover \$300.
                \item The UCC has been updated to force covering buyers to use 
                \S\ 7-212, but most states have not yet adopted the change.
                % TODO add Delchi Carrier SpA
                % TODO resume at nehri
            \end{enumerate}
        \end{enumerate}
        \item \textbf{Buyer's breach}:
        \begin{enumerate}
            \item % TODO
        \end{enumerate}
    \end{enumerate}
    \item \textbf{Mitigation and contracts for employment}:
    \begin{enumerate}
        \item % TODO
    \end{enumerate}
    \item \textbf{Foreseeability}:
    \begin{enumerate}
        \item % TODO
    \end{enumerate}
    \item \textbf{Certainty}:
    \begin{enumerate}
        \item % TODO
    \end{enumerate}
    \item \textbf{Liquidated damages}:
    \begin{enumerate}
        \item % TODO
    \end{enumerate}
\end{enumerate}

\subsubsection{Reliance Damages}
% TODO

\subsubsection{Restitution Damages}
% TODO

\subsection{Assent}

% TODO

\subsection{Form Contracts}

% TODO

\subsection{Mistake}

% TODO
